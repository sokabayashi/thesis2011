\documentclass[oneside]{myumnStatThesis}
%\usepackage{epsfig}
\graphicspath{{Figures/}}
\usepackage{graphicx}
\usepackage{epsfig,color}

\DeclareMathOperator{\E}{E}
\DeclareMathOperator{\I}{I}
\DeclareMathOperator{\Var}{Var}
\DeclareMathOperator{\Cov}{Cov}
\DeclareMathOperator{\logit}{logit}
\DeclareMathOperator{\dom}{dom}
\DeclareMathOperator{\cl}{cl}
\DeclareMathOperator{\bd}{bd}
\DeclareMathOperator{\intr}{int}
\DeclareMathOperator{\rintr}{rint}
\DeclareMathOperator{\con}{con}
\DeclareMathOperator{\aff}{aff}
\DeclareMathOperator{\epi}{epi}
\DeclareMathOperator{\lev}{lev}

\def\RR{{\mathbb R}}
\def\ZZ{{\mathbb Z}}
\def\D{{\mathcal D}}
\def\XX{{\mathcal X}}
\def\YY{{\mathcal Y}}
\def\NN{{\mathcal N}}
\newcommand{\deriv}[2]{\frac{d #1}{d #2}}
\newcommand{\dderiv}[2]{\frac{d^2 #1}{d #2^2}}
\newcommand{\pderiv}[2]{\frac{\partial #1}{\partial #2}}
\newcommand{\ppderiv}[2]{\frac{\partial^2 #1}{\partial #2^2}}
\newcommand{\ppmderiv}[3]{\frac{\partial^2 #1}{\partial #2 \partial #3}}
\newcommand{\fatdot}{\,\cdot\,}
\newcommand{\inner}[1]{\langle #1 \rangle}
\newcommand{\set}[1]{\{\, #1 \,\}}
\newcommand{\abs}[1]{\lvert #1 \rvert}
\newcommand{\norm}[1]{\lVert #1 \rVert}
\newcommand{\etaMLE}{\hat{\eta}_{\textrm{MLE}}}
\newcommand{\betaMLE}{\hat{\beta}_{\textrm{MLE}}}



\author{Saisuke Okabayashi}
\adviser{Charles J. Geyer}
%\coadviser{Co-Adviser Name Here}
\title{Maximum Likelihood in Exponential Families with Complex Dependence}
\month{April}
\year{2011}
% Month and Year of Degree Clearance, NOT necessarily when you defended

\begin{document}

%\makesignaturepage % required
\maketitlepage % required
%\makecopyrightpage % recommended, required if registering copyright
%\frontmatter
%\begin{acknowledgementspage} % optional
%I owe all of my success to Alicia who is awesome.
%\end{acknowledgementspage}

\begin{abstract}
A social network is an example of a phenomenon with complex stochastic dependence 
that is commonly modeled with a class of exponential families called exponential random graph models (ERGM).
Maximum likelihood estimators (MLE) for such  
exponential families
can be difficult to estimate when the likelihood is difficult to compute.
Most methodologies rely on iterated estimates and are sensitive
to the starting value, often unable to converge if started too far from the
solution.  
% 4/7 Sai
%In addition, these methods may require significant trial-and-error to 
%work effectively.  
% old comment out
%Markov chain Monte Carlo (MCMC) methods based on the MCMC-MLE algorithm in \citep
%{Geyer:1992} are guaranteed to converge in theory under certain conditions when 
%starting from any value, but in practice such an algorithm may labor to converge when 
%given a poor starting value.  
Even more problematic is that the MLE may not exist, 
a situation that occurs with positive probability for ERGMs.
% for discrete state space exponential families like ERGMs.
In such a case, the MLE 
is actually ``at infinity" in some direction of the parameter space.  
% 4/7 Sai
%\citet{Geyer:gdor} illustrated an algorithm to detect non-existent MLEs 
%and construct one-sided confidence intervals for how close
%the parameter is to infinity in the case of generalized linear 
%models, where the convex support is known.  

Here we present a simple line search algorithm to find the MLE of a regular exponential 
family when the MLE exists and is unique.  
The algorithm can be started from any 
initial value and avoids trial-and-error experimentation.  
% 4/7 Sai
%We show convergence of the algorithm for the case where the gradient can be 
%calculated exactly.  When it cannot, it has a particularly convenient form that is 
%easily estimable with Markov chain Monte Carlo (MCMC), making the algorithm still useful to a practitioner.  
%Finally, w
When the MLE does not exist, our algorithm adapts
\citeauthor{Geyer:gdor}'s \citeyearpar{Geyer:gdor} approach
to detect non-existent MLEs and construct one-sided confidence intervals
for how close the parameter is to infinity.
%to a setting where the convex support is not
%known in advance, which is typically the case for network models.  
%We are able to then calculate one-sided confidence intervals
%for the parameters.
\end{abstract}

\tableofcontents % required

\newpage
\chapter*{List of Tables}
\addcontentsline{toc}{chapter}{List of Tables}
{\def\chapter*#1{}
\listoftables}

\newpage
\chapter*{List of Figures}
\addcontentsline{toc}{chapter}{List of Figures}
{\def\chapter*#1{}
\listoffigures}


\mainmatter

\chapter{Introduction}
%\input{Introduction}
Exponential families are commonly used to model phenomena with dependent structure, 
where the outcomes of the response variable of interest are in fact dependent on one another.  For example, the Ising 
model \citep{Ising,Potts} is an exponential family model that has been used to model ferromagnetism.  A realized 
sample from this model is depicted in Figure~\ref{F:pottsimage}, where neighboring pixels (representing atoms in a crystal lattice)
are more likely to have the same color.  We explore this model further in Section~\ref{S:Examples:Ising}.
Other examples of phenomena with dependent structure modeled with exponential families include
plant ecology \citep{Besag:1974,Besag:1975}, 
friendship networks \citep{Wasserman:1996,advancesp*,Goodreau:2009}, 
protein-protein interaction networks \citep{Saul:2007},
and the lifetime fitness of plants \citep{Shaw:2008}.
  

\begin{figure}
\begin{center}
\includegraphics[width=4.5in,keepaspectratio]{potts}
\end{center}
\caption{A realized sample from an Ising model on a $32 \times 32$ lattice with $\eta = \left(0, \log(1 + \sqrt{2}) 
\right)^T$.  This value of $\eta$ corresponds to the phase transition point, where the sample images are mostly one 
color with small but significant portions of the other color.  There is no preference for the dominant color to be 
white or black.}
\label{F:pottsimage}
\end{figure}

The appeal of exponential families in these settings stems from their simplicity and maximum entropy property \citep{Jaynes:1978,Geyer:1992}.
By choosing statistics of interest on the data, one fully specifies a model that gives the 
most reasonable inference possible derived solely from those statistics.  Furthermore, exponential families have been 
well-studied \citep{Barndorff,Brown:1986} and utilized over the decades and have desirable properties such as a 
strictly concave likelihood function.

\section{Parameter estimation methods in exponential families}
Calculating the maximum likelihood estimators (MLE) for exponential families when dependence is complex, however, 
remains a challenging problem because the likelihood function may be computationally infeasible.  In particular, the 
form of the likelihood is determined by the chosen statistics up to a normalizing constant, but this normalizing 
constant may involve a summation over an astronomical number of terms.  Evaluating the likelihood function---let alone 
maximizing it---presents a significant challenge.
  
Two commonly used parameter estimation methods to circumvent this issue in exponential families are the \emph{pseudo-likelihood}
approach \citep{Besag:1975,Strauss:1990,Composite}, which finds parameter values that maximize the pseudo-likelihood function,
and the \emph{Markov chain Monte Carlo maximum likelihood estimate} (MCMC-MLE) approach \citep{Geyer:1992,Geyer:1994},
which uses MCMC to approximate the log likelihood so that it can subsequently be maximized.  
The pseudo-likelihood approach is computationally expedient, but has been shown to produce unreliable results when 
dependence is strong \citep*{Geyer:1992,Duijn:2009}.  

The MCMC-MLE approach is theoretically guaranteed to converge to the MLE if it exists and is the default algorithm in 
software packages such as \texttt{statnet} \citep{statnet:R} in the R platform for network models.  
However, this approach has been shown in practice to be sensitive to initial parameter values when used without the 
trust region methodology recommended in \citep{Geyer:1992}, and the algorithm may require many iterations and enormous 
(sometimes infeasibly large) Monte Carlo sample sizes when the starting value is far from the MLE \citep*{ergm}.  
Improvement to the MCMC-MLE approach is an active area of research \citep{Hummel}.     

Variations on the Robbins-Monro \emph{stochastic approximation} (SA) algorithm \citep{Robbins-Monro} have 
been applied to find in the MLE similar contexts: \citet{Younes:1988,Younes:1989,Moyeed:1991,Gu:2001}
applied MCMC stochastic approximation to spatial models and \citet{Snijders:2002} to social network 
models (exponential random graph models).
%Our approach shares a similar recursive mechanism with SA, so we will describe this method further.  
SA procedures for finding the MLE of a parameter $\eta$ generate iterated estimates $\eta_k$ to find the 
root of a gradient function $h(\eta)$:
\begin{align} \label{E:eta SA update}
	\eta_{k+1} = \eta_k + \alpha_k Y_k,
\end{align}
where $\alpha_k$ is a step size and is typically a member of a decreasing sequence of positive numbers, and $Y_k$ is a 
random variable from the distribution specified by $\eta_k$ that noisily estimates the gradient function $h(\eta_k)$.  

Restrictive conditions are required of $\alpha_k$ and $Y_k$ to establish convergence of the sequence $\eta_k$.  
In Robbins-Monro SA \citep{Robbins-Monro}, the step size $\alpha_k$ must be a sequence of positive constants 
that satisfies 
\begin{align*}
\sum \alpha_k^2 < \infty
%	\sum \alpha_k = \infty, \qquad \sum \alpha_k^2 < \infty.
\end{align*}
for which the choice of
\begin{align} \label{E:SA step size}
\alpha_k = \frac{A}{B + k}
\end{align}
 is commonly used, where $A$ and $B$ are constants that must be specified by the user.  This specification requires experimentation and care.  There can be significant variation in performance depending on choice of these constants. 
A large body of more recent research presents evidence for a sequence that goes to 0 more slowly than $1/k$ 
for faster convergence \citep[Chapter 11]{Kushner:1997}.  
%Regardless of the exact form, there is no guarantee the likelihood function will increase at each update. 

The conditions on $Y_k$ are more restrictive.  Popular approaches include constraining the sequence of estimators $
\eta_k$ to a compact set specified \emph{a priori}, or assuming that the noise component of $Y_k$ be a martingale 
difference sequence.  As commonly observed \citep{Chen:2002,Andrieu:2005,Liang:2010} these may be 
difficult to satisfy in practice.  
See \citep{Andrieu:2005,Liang:2010} for recent developments that impose less restrictive conditions using truncated 
updates.

An issue for any recursive search algorithm including stochastic approximation is the choice of starting point.  It is 
often the case that algorithms are good at finding the MLE when the starting point is close to it, but of course the 
location of the MLE is unknown.  For any exponential family with bounded support, Fisher information 
becomes singular as the canonical parameter $\eta$ goes to $\infty$ \citep{Rinaldo:2009}.  Hence methods which rely on 
the Fisher information matrix may fail when the starting point for $\eta$ is far from the MLE \citep{Younes:1989,Gu:2001}.
Of course, one may try different starting points until a ``good'' one is found, but this can be cumbersome in 
practice.

\section{Algorithm overview} 
In this article, we propose a simple and practical line search algorithm that converges to the MLE of any regular 
exponential family when the MLE exists and is unique and the first derivative of the log likelihood can be calculated 
exactly.  
When it cannot, the first derivative still has a particularly convenient form that is easily estimable with MCMC, 
making the algorithm still useful in application.  We also show how to construct and apply confidence intervals in such 
a setting to increase the probability of convergence.  

The appeal of this algorithm is its usability: no trial and error is needed.  
No experimentation with multiple starting points or tuning parameters is necessary and
no unrealistic \emph{a priori} information about the problem need be specified.  
It is currently used in the \texttt{aster2} contributed R package 
\citep{aster:R} as the safeguard for steepest ascent and Newton-Raphson iterations in finding the MLE for aster models.

Our algorithm generates iterated estimates $\eta_k$ of the MLE $\hat{\eta}$ of the form 
\begin{align} \label{E:eta update}
	\eta_{k+1} = \eta_k + \alpha_k p_k
\end{align}
where $\alpha_k$ is a step size and $p_k$ is a \emph{search direction} and is restricted to be an ascent direction of 
the log likelihood.  
Despite the visual similarity between \eqref{E:eta SA update} and \eqref{E:eta update}, the line search approach treats 
the search direction $p_k$ in \eqref{E:eta update} as constant whereas in SA the corresponding $Y_k$ in \eqref{E:eta SA 
update} is random.
Furthermore, line search algorithms have more restrictions on the step size $\alpha_k$.  
The step size 
conditions in the classical gradient ascent algorithm, which is the basis for our algorithm, force  a sufficiently 
large increase in the objective function at every step, guaranteeing convergence to the global maximum.
%By contrast, there may be steps in SA that result in a decrease of the objective function.   

Theorem 3.2 in \citep{NW} implies the global convergence of the steepest ascent 
algorithm for a continuously differentiable function, $\ell(\eta)$.  It requires the step length $\alpha_k$ to satisfy 
the Wolfe conditions for \emph{sufficient increase} and \emph{curvature}:
\begin{equation} \label{eq:wolfe}
\begin{split}
	\ell(\eta_k + \alpha_k \eta_k) \geq \ell(\eta_k) + c_1 \alpha_k \nabla \ell (\eta_k)^T p_k \\
	\nabla \ell( \eta_k + \alpha_k p_k)^T p_k \leq c_2 \nabla \ell( \eta_k)^T p_k
\end{split}
\end{equation}
where $\nabla$ is the gradient operator and $0 < c_1 < c_2 < 1$.   
Variations of these conditions exist in the numerical optimization literature \citep{Fletcher,NW,Sun:2006}, but all 
require evaluating the objective function.

Exponential families we consider are an unusual case in optimization in that the objective function 
is harder to compute than its derivatives and hence not previously considered by optimization theorists.
In our algorithm, we replace \eqref{eq:wolfe} with a single modified curvature condition:
\begin{align} \label{E:curvature mod}
	 0 & \leq \nabla \ell( \eta_k + \alpha_k p_k)^T p_k \leq c \nabla \ell(\eta_k)^T p_k
\end{align}
for some $0 < c < 1$.  This replacement is possible while still guaranteeing sufficient increase and convergence 
because we have the additional property that the exponential family log likelihood function we consider is strictly 
concave.  The restrictions on the step size $\alpha_k$ along a particular direction $p_k$ and the resulting values for 
$\ell(\eta_{k+1})$ are depicted in Figure~\ref{F:alpha_region}.  



\begin{figure}
\centering
    \scalebox{.4}{\input{Figures/alphamax.pstex_t}}
	\caption{The acceptable region for step size $\alpha_k$ along a particular search direction $p_k$ according to the 
modified curvature condition \eqref{E:curvature mod}.  The step sizes $\alpha_{c}$ and $\alpha_{\textrm{max}}$ 
correspond to values of $\nabla \ell( \eta_k + \alpha p_k)^T p_k$ equaling $c \nabla \ell(\eta_k)^T p_k$ and $0$, 
respectively.  The condition ensures sufficient increase in the log likelihood along the search direction $p_k$.}
\label{F:alpha_region}
\end{figure}
 
The desire to avoid calculation of higher order derivatives is motivated not just by computational considerations, but 
also by how much useful information can be extracted from them.   As noted previously, if $\eta$ is far from the MLE,  
the Fisher information matrix may be near-singular and algorithms like (unsafeguarded) Newton-Raphson algorithm may fail.  For this 
reason, the best use of our algorithm may be from ``long range,'' filling a gap in the MLE estimation toolbox.  It may 
be expedient to switch to another algorithm like Newton-Raphson after significant progress is made and 
the Fisher information matrix becomes useful.  Our line search algorithm with $p_k$ the Newton direction provides a
safeguard for Newton-Raphson that makes it safe (not necessarily efficient) for use from any range.
%, though it is difficult to quantify what is significant progress since the 
%limitations of these other algorithms are not well-defined.  
The \texttt{aster2} contributed R package \citep{aster:R} switches $p_k$ from steepest ascent direction to Newton direction
after a fixed number of steps ($d / 2$ where $d$ is the dimension $\eta$) but always finds a step length $\alpha_k$ satisfying
\eqref{E:curvature mod}, iterating until the (unsafeguarded) Newton step satisfies \eqref{E:curvature mod}.\\


%%% Can't just put a \newpage just anywhere in the text
\pagebreak[3]

Our algorithm can be outlined as follows.  Let $\lVert \, \cdot \, \rVert$ denote the Euclidean norm function, and $\epsilon$ 
a small value greater than 0.  \\

% ALGORITHM 
\noindent Get an initial value, $\eta_0$.\\ 
Set $k=0$. \\
Calculate $\nabla \ell( \eta_k)$, the direction of steepest ascent. \\
Set $p_k = \nabla \ell( \eta_k)$. \\
\textbf{while}  $\lVert \nabla \ell( \eta_k) \rVert > \epsilon$ \\ 
\hspace{4mm} \indent	 \textbf{Find} a step size $\alpha_k$ that satisfies the modified curvature condition
\begin{align*}
	 0 & \leq \nabla \ell( \eta_k + \alpha_k p_k)^T p_k \leq c \nabla \ell(\eta_k)^T p_k
\end{align*}
\indent for some $0 < c < 1$.  
%This condition requires $\alpha_k$ to fall within \\
%\indent the acceptable region in Figure \ref{F:alpha_region}. \\


$\eta_{k+1} = \eta_k + \alpha_k p_k$.\\
\indent Calculate $\nabla \ell( \eta_{k+1})$.\\
\indent \textbf{Find} the new search direction $p_{k+1}$, which must be an ascent direction. \\
\indent $k = k + 1$.  \\
\textbf{end while}\\


   

%Finally, a future application of this algorithm may be to discrete state space exponential families 
%where the MLE for $\eta$ does not exist with positive probability.  
%In these cases, the MLE is ``at infinity'' and the concave log likelihood continues to increase as 
%$\eta \to \infty$ along certain directions of recession \citep{Rinaldo:2009, Geyer:gdor}.  
%Finding ``the'' MLE then becomes a very different problem, and conditions to restrict $\eta$ to a 
%compact set as utilized in MCMC stochastic approximation \citep{Andrieu:2005, Liang:2007, Liang:2010} 
%would not be relevant.    Our proposed algorithm may be applied to this setting to climb the log likelihood until it 
%``flattens out''  and the gradient approaches zero.  
%This in turn may be useful in identifying the directions of recession. 

%%%%%%%%%%%%%%%%%%%%%%%%%%%%%%%%%%%%%%%%%%%%%%%%%%%%%%%%
\chapter{Background Exponential Family Theory}\label{Section:Background}
%\input{Background}

An exponential family of distributions \citep{Barndorff,Geyer:gdor}
on a sample space $\YY$ has log likelihood
\begin{align} \label{E:loglike}
   \ell(\eta) = \inner{g(y), \eta} - c(\eta)
\end{align}
where $g(y)$ is a $d$-dimensional vector of canonical statistics, $\eta$ a $d$-dimensional vector of
canonical parameters, and $\inner{\fatdot, \fatdot}$ denotes the bilinear form
$$
   \inner{g, \eta} = \sum_{i =1}^d g_i \eta_i.
$$
So that the probability function integrates to 1,
the cumulant function $c$ must have the form
\begin{align} \label{E:cumulant}
      c(\eta) = \log \left(\int e^{\inner{g(y), \eta}} \, d\mu(y) \right),
\end{align}
where $\mu$ is a measure on $\YY$.
Define
\begin{align} \label{E:fullparam}
	\Xi = \{ \eta \in \RR^q: c(\eta) < \infty \}.
\end{align}
The exponential family is \emph{full} if the natural parameter space is \eqref{E:fullparam}, and \emph{regular} if, in 
addition, $\Xi$ is an open set.  We say an exponential family is \emph{minimal} if $g(y)$ is not concentrated on a 
hyperplane. Minimality guarantees that if an MLE exists, it is unique \citep{Geyer:gdor}.

In finite state space models with complicated dependence like an Ising model or exponential random graph model,  \eqref
{E:cumulant} is a sum which may have no simple expression and can only be evaluated by explicitly doing the sum.
When the sample space $\YY$ is even moderately large, this can be prohibitively expensive.  For example, an Ising model 
defined on a $32\times 32$ square lattice where each entry takes values of 0 or 1, there are $2^{1024} \approx 10^
{300}$ elements in $\YY$.  A loop with this many iterations takes too long no matter how programmed.

A useful property of all exponential families \cite[p.~27]{TPE2} on which we rely heavily is that 
\begin{align*}
	\E_\eta(g(Y)) &= \nabla c(\eta)	\\
	\Var_\eta(g(Y)) &= \nabla^2 c( \eta ).
\end{align*}

Thus we can express first and second derivatives of the log likelihood \eqref{E:loglike} and Fisher information, $I
(\eta)$, as
\begin{align}
	\nabla \ell( \eta ) &= g(y) - \E_\eta g(Y) \label{E:nabla ell} \\
	\nabla^2 \ell( \eta ) &=  - \Var_\eta g(Y) \label{E:nabla2 ell} \\
	\I(\eta) &= -\E_\eta \nabla^2 \ell (\eta ) = \Var_\eta g(Y) \label{E:FI}
\end{align}
and thereby avoid evaluation of the problematic cumulant function $c$.



\chapter{Algorithm}
%%%%%%%%%%%%%%%%%%%%%%%%%%%%%%%%%%%%%%%%%%%%%%%%%%%%%%%%
\section{Long range search algorithm for MLEs} \label{Section:Algorithm}
We now present our search algorithm, which will converge to the optimum for any continuously differentiable, strictly 
convex (or concave) function.  The algorithm and requirements are presented in Theorem~\ref{Thm:Line Search}.  Proofs 
are in Appendix~\ref{Section:Proofs}. 


We apply the algorithm in Theorem~\ref{Thm:Line Search works} to the specific setting of finding the MLE in a regular 
exponential family when the MLE is known to exist and be unique and the gradient can be calculated exactly.

%\section{Line search convergence}
In order to be consistent with the general optimization literature \citep{Fletcher,NW}, we state our algorithm in this 
section from the perspective of a minimization problem.  Thus we wish to minimize a real-valued objective function $f$ 
defined on $\RR^n$.  


%%%%%%%%%%% BEGIN THEOREM %%%%%%%%%%%%%%
\begin{theorem}[Convex function root search] \label{Thm:Line Search}
Consider any line search of the form 
\begin{align}
	x_{k+1} &= x_k + \alpha_k p_k \label{E:x_update}
\end{align}
used to minimize the objective function $f$, which satisfies the following assumptions:
\begin{enumerate}
	\item The objective function $f$ is bounded below in $\RR^n$. \label{ass:one}
	\item The objective function $f$ is proper, lower semicontinuous, and strictly convex.
	\item The objective function $f$ is differentiable in an open set $\NN$ containing the level set $\lev_{\leq f
(x_0)} f$, which is bounded, where $x_0$ is the starting point of the iteration.
%\item The \emph{step length} $\alpha_k$ is greater than $0$ unless $\nabla f(x_k) = 0$, 
% in which case $x_k$ is already the solution and the search is complete.
\item The \emph{search direction} $p_k$ is a non-zero \emph{descent direction} \label{ass:four}
such that the angle $\theta_k$ between the search direction $p_k$ and steepest descent direction $-\nabla f(x_k)$ is 
restricted to be less than 90 degrees by
\begin{align*}
\cos \theta_k \geq \delta > 0
\end{align*}
 for some fixed $\delta > 0$.  
\end{enumerate}

Then, unless $\nabla f(x_k) = 0$, in which case $x_k$ is already the solution and the search is complete, it is 
possible to find a step length $\alpha_k$ that satisfies the \emph{curvature condition}
\begin{align}
	c \nabla f(x_k)^T p_k &\leq \nabla f( x_k + \alpha_k p_k)^T p_k \leq 0 \label{E:Wolfe-mod}
\end{align}
for some fixed $0 < c < 1$.

Furthermore, repeated iterations of \eqref{E:x_update} satisfying assumptions~\ref{ass:one} through~\ref{ass:four} and \eqref{E:Wolfe-mod} will produce 
a sequence, $x_1, x_2, \ldots$ such that
\begin{align*}
	\lim_{k \to \infty} || \nabla f(x_k) || = 0.
\end{align*}
\end{theorem}

%\section{Line search for MLE estimation}
\begin{theorem}[Exponential family log likelihood maximization] \label{Thm:log like max}
Consider any line search of the form 
\begin{align}
	\eta_{k+1} &= \eta_k + \alpha_k p_k \label{E:eta_update}
\end{align}
used to minimize the negative log likelihood function $-\ell(\cdot)$ of a regular exponential family, where the \emph{search direction} $p_k$ is a non-zero \emph{descent direction}
such that the angle $\theta_k$ between the search direction $p_k$ and steepest descent direction $-\nabla \ell(\eta_k)$ is 
restricted to be less than 90 degrees by
\begin{align*}
\cos \theta_k \geq \delta > 0
\end{align*}
 for some fixed $\delta > 0$.  

Then, unless $\nabla \ell(\eta_k) = 0$, in which case $\eta_k$ is already the solution and the search is complete, it is 
possible to find a step length $\alpha_k$ that satisfies the \emph{curvature condition}
\begin{align}
	0 \leq \nabla \ell( \eta_k + \alpha_k p_k)^T p_k  \leq c \nabla \ell(\eta_k)^T p_k  \label{E:Wolfe-ll}
\end{align}
for some fixed $0 < c < 1$.

Furthermore, repeated iterations of \eqref{E:eta_update} along a descent direction satisfying \eqref{E:Wolfe-ll} will produce a sequence, $\eta_1, \eta_2, \ldots$ such that
\begin{align*}
	\lim_{k \to \infty} || \nabla \ell(\eta_k) || = 0.
\end{align*}
\end{theorem}



We apply Theorem~\ref{Thm:Line Search} to the setting of exponential families to find the MLE when it exists.  

\begin{theorem}[] \label{Thm:Line Search works}
For a regular exponential family with minimal representation where the MLE exists, the line search described in 
Theorem~\ref{Thm:Line Search} can be applied to the negative log likelihood function $-\ell(\eta)$ so that a search 
starting at any $\eta_0 \in \Xi$ will converge to the MLE of $\eta$.
\end{theorem}

The issue of MLE existence is a problem in computational geometry, not an optimization problem, so we do not address it 
here.  See \citep{Geyer:gdor,Rinaldo:2009} and references cited therein.

%%%%%%%% ADD CONJUGATE GRADIENT SECTION IN NEXT VERSION
\chapter{Refinements of algorithm}

In Theorem~\ref{Thm:Line Search}, we restricted our search direction $p_k$ to be a descent direction, so that $\nabla f
(x_k)^T p_k < 0$ or, alternatively, the angle $\theta_k$ between the search direction $p_k$ and steepest descent 
direction $-\nabla f(x_k)$ is less than 90 degrees.  However, this still leaves many possibilities for the choice of 
$p_k$ other than steepest descent.  In addition, we have specified restrictions on the step size $\alpha_k$ in the 
curvature condition \eqref{E:Wolfe-mod} with $0 < c < 1$, but it would be useful to know if certain values of $c$ are 
better than others.

\section{Search directions}
In our examples in Section~\ref{S:Examples}, we default to steepest descent directions in our implementation for 
transparency.  Although often effective in early steps, steepest descent directions can result in a zigzagging 
trajectory of the sequence $x_k$ \citep{Sun:2006}.  Conjugate gradient methods address this phenomena and cover the 
sample space more efficiently \citep{NW}.  It is easy to implement a variant of the Polak-Ribi\`{e}re method
\citep[pp.~120--122]{NW} here, requiring little more in terms of calculation or storage.  The search direction $p_k$ would update 
with an extra intermediate step as follows:
\begin{align*}
	\gamma_{k+1}^{PR} &= \max \left( 0, \frac{ [ \nabla f( x_{k+1}) ]^T( \nabla f( x_{k+1} ) - \nabla f( x_k) )  }
{ \lVert \nabla f( x_k) \rVert^2 } \right )\\
	p_{k+1} &= -\nabla f( x_{k+1}) + \gamma_{k+1}^{PR} \, p_k.
\end{align*}
Note that when $\gamma_{k+1}^{PR} = 0$, $p_{k+1}$ will be just $-\nabla f( x_{k+1})$, the direction of steepest 
descent, and thus serves as a ``reset''.  The curvature condition \eqref{E:Wolfe-mod} guarantees that this method always 
yields a descent direction for $p_{k+1}$ and thus Theorem~\ref{Thm:Line Search} still holds.  

\section{Step size}
We now turn our attention to the optimal step size $\alpha_k$ when our objective function is the log likelihood of an 
exponential family.  Taking the derivative of $\ell( \eta_k + \alpha_k p_k)$ with respect to $\alpha_k$ shows that the 
log likelihood is maximized as a function of $\alpha_k$ along the direction $p_k$  when 
\begin{align*}
	\nabla \ell( \eta_{k+1} )^T p_k = 0.
\end{align*}

By choosing $c$ to be small, say 0.2, we ensure that the step taken is close to maximizing the log likelihood along the 
search direction.  This is also apparent in Figure~\ref{F:alpha_region}. 

Making $c$ too small, however, may make it difficult to find an $\alpha_k$ that meets the curvature condition \eqref
{E:curvature mod} since this search must be done numerically.  In fact, as the line search nears the MLE and $\nabla 
\ell( \eta_k)$ gets smaller, the rightmost term in \eqref{E:curvature mod} gets smaller in magnitude (it equals $c 
\lVert \nabla \ell(\eta_k) \rVert^2$ if using steepest ascent directions), making a numerical search for $\alpha_k$ 
more challenging.  

%Finally, while the choice of 0.2 for $c$ worked well in the problems we explored regardless of search 
%directions used, it follows from our discussion in the previous section that it may make sense to use slightly larger 
%values of $c$ when using steepest ascent directions, thereby reducing the zigzagging phenomenon, but smaller values for 
%$c$ when using conjugate gradient methods.


\section{MCMC approximations} \label{section:MCMC approx}
Our algorithm requires us to be able to calculate $\nabla \ell(\eta)$ using \eqref{E:nabla ell}.  For many 
applications, we will need to approximate $\E_{\eta}g(Y)$ using MCMC.  That is,
\begin{align}
 	\nabla \ell (\eta) = g(y) - \E_\eta g(Y) \approx g(y) - \frac{1}{m}\sum_{i=1}^m g(Y_i), \label{E:nabla ell approx}
\end{align}
where $Y_1, \ldots, Y_m$ are MCMC draws from the distribution with parameter $\eta$.  There are many MCMC algorithms 
such as Metropolis-Hastings or Swensen-Wang (used for the Ising model example in Section~\ref{S:Examples:Ising}); see 
\citep{Brooks} and references cited therein.
We show examples in the next section where $\nabla \ell(\eta)$ can be calculated exactly and where it must be 
approximated.

The accuracy of the approximation in \eqref{E:nabla ell approx} increases with Monte Carlo sample size $m$. 
When the current estimate is far away from the MLE, we can use smaller $m$ to save time and work with a 
fairly noisy approximation of the gradient.  However, when the current estimate approaches the MLE, larger $m$ are necessary.

Our algorithm relies on the computed values of $\nabla \ell(\eta)$ in the curvature condition \eqref{E:curvature mod}, 
as well as the stop condition for the algorithm, $\lVert \nabla \ell( \eta_k ) \rVert < \epsilon$.  Given that we may 
only have approximations of $\nabla \ell(\eta)$, we cannot know for certain if either of these conditions are truly 
met.  We can ameliorate this by constructing confidence intervals for each of the inequalities.  

For the inequalities in \eqref{E:curvature mod}, we can estimate asymptotic standard errors of $\nabla \ell( \eta_k + 
\alpha_k p_k)^T p_k$  and $c \nabla \ell(\eta_k)^T p_k - \nabla \ell( \eta_k + \alpha_k p_k)^T p_k$ by appealing to the 
Markov chain Central limit theorem \citep{Chan:1994,Jones:2004,Roberts:1997,Roberts:2004}.
The \texttt{initseq} function from the R package \texttt{mcmc} \citep{mcmc:R} can be used to estimate asymptotic 
standard errors for univariate functionals of reversible Markov chains: given an MCMC sample for a univariate 
quantity, \texttt{initseq}
returns a value (divided by sample size) that is an estimate of the asymptotic variance in the Markov chain central 
limit theorem.  Both of the quantities in \eqref{E:curvature mod} are univariate.  In the second expression, $c \nabla \ell(\eta_k)^T 
p_k - \nabla \ell( \eta_k + \alpha_k p_k)^T p_k$, the MCMC sample generated for $\nabla \ell( \eta_k + \alpha_k p_k)^T 
p_k$ is independent of the sample generated for $c \nabla \ell(\eta_k)^T p_k$.  Thus \texttt{initseq} can be applied 
to each sample separately and the results summed for an estimated variance.  
We can then be approximately 95\% confident (non-simultaneously) that $\alpha_k$ satisfies \eqref{E:curvature 
mod} if
\begin{align*}
	 \nabla \ell( \eta_k + \alpha_k p_k)^T p_k - 1.645 \cdot \text{se}_1 > 0 \\
	 c \nabla \ell(\eta_k)^T p_k - \nabla \ell( \eta_k + \alpha_k p_k)^T p_k - 1.645 \cdot \text{se}_2 > 0 
\end{align*}
where $\text{se}_1$ and $\text{se}_2$ are the asymptotic standard errors for $\nabla \ell( \eta_k + \alpha_k p_k)^T p_k
$  and $c \nabla \ell(\eta_k)^T p_k - \nabla \ell( \eta_k + \alpha_k p_k)^T p_k$, respectively, calculated as described.

The delta method can be applied to estimate a standard error for $\lVert \nabla \ell( \eta_k ) \rVert$. The asymptotic 
variance is calculated by
\begin{align*}
	V \left( \lVert \nabla \ell( \eta_k ) \rVert \right )= \frac{1}{\lVert \nabla \ell( \eta_k ) \rVert^2} \nabla \ell
( \eta_k )^T \, \Sigma \,  \nabla \ell( \eta_k ),
\end{align*}
where $\Sigma$ is the variance matrix of $\nabla \ell( \eta_k )$ and can be estimated by the sample variance matrix of 
the batch mean vectors of $g(Y_1), \ldots, g(Y_n)$ divided by the number of batches (the \texttt{initseq} function 
requires a univariate vector and so cannot be used here).  We can be approximately 95\% confident that $\lVert 
\nabla \ell( \eta_k ) \rVert > \epsilon$ if 
\begin{align*}
	\lVert \nabla \ell( \eta_k ) \rVert - 1.645 \sqrt{ V \left( \lVert \nabla \ell( \eta_k ) \rVert \right )} > 
\epsilon.
\end{align*}
%To be conservative, we suggest using a larger quantile, say 2, 
%in the confidence intervals instead of the 1.645 used above.

In practice, however, use of confidence intervals does not appear necessary with  Monte Carlo sample sizes that are set 
large enough so that these standard errors are initially small relative to the point estimates.  The ratio of point 
estimate to standard error of course decreases as the algorithm progresses and the estimate of the parameter nears the 
MLE, reflected in $\nabla \ell( \eta_k )$ nearing 0.  Thus these confidence intervals are most useful as a guide for
when to increase the MCMC sample size, or when to switch methods, or when to terminate the algorithm.



\section{Combining with other algorithms}
We believe the best use of this algorithm is in combination with other faster methods like MCMC-MLE \citep{Geyer:1992}
or Newton-Raphson safeguarded by our line search algorithm.  Our algorithm with steepest ascent or conjugate gradient search direction
should be used initially from ``long range'', when one has no good intuition for an initial value.
It is well known that when the objective function is quadratic the conjugate gradient method with exact arithmetic converges to the solution
in at most $d$ steps, where $d$ is the dimension of the problem \citep{NW}.  As a rule of thumb, we think using our 
algorithm for $d$ steps before switching seems reasonable.
 


\chapter{Examples} \label{S:Examples}
\section{Example: logistic regression}
We illustrate the application of our algorithm in the case of a logistic regression with a starting point far from the 
solution.  In such a case, the Hessian matrix is often near-singular and algorithms such as Newton-Raphson which rely 
on it will fail.  For classical SA with step size $1/k$, the magnitudes of the updates diminishes too quickly for 
the parameter estimates to approach the MLE in a reasonable amount of time.

The response vector $Y$ has components that are Bernoulli trials with mean vector $p$.  The natural parameter is $
\theta_i = \log \left( \frac{p_i}{1-p_i} \right )$, which is modeled componentwise as a linear function of the 
predictors $1, x_1, \ldots, x_{q}$, so that
\begin{align*}
	\theta_i &= \beta_0 + \beta_1 x_{1i} + \beta_2 x_{2i} + \cdots + \beta_{q} x_{q\,i} = \beta^T x_i \qquad
i = 1, \ldots, n
\end{align*}
where $\beta = (\beta_0, \ldots, \beta_{q} )^T$ and $x_i = ( 1, x_{1i}, \ldots, x_{q i})^T$.  

Defining the model matrix $M$ to be the $n \times (q+1)$ matrix with the $x_i$ as rows, we can express $\theta = M \beta
$.  This in turn allows us to reparameterize the exponential family  as one with $\beta$ as the natural parameter 
vector and $M^T y$ the vector of statistics with log likelihood
\begin{align*}
		 \ell(\beta) &=  \beta^T (M^T y) - c(\beta),
\end{align*}
where $y$ is the vector of observed Bernoulli responses.
By \eqref{E:nabla ell}, the gradient is
\begin{align*}
	\nabla \ell( \beta ) &=  M^T y - \E_{\beta}(M^T Y) = M^T( y - \E_{\beta}(Y) ),
\end{align*}
where $\E_{\beta}(Y) = \frac{1}{1 + \exp(-M \beta)}$ can be calculated exactly.  This allows us to directly apply 
Theorem~\ref{Thm:Line Search works}.

Suppose we specify our true parameter value to be $\beta = (0, 2, 2, 1, 1, 0, 0, 0)^T$ and use 100 independent draws 
from a correlated multivariate normal distribution centered at 0 as our predictors to generate 100 independent 
Bernoulli trials.  
Fitting these data using the R function \texttt{glm}, we find the MLE of $\beta$ to be
\begin{align*}
	\betaMLE = (  0.635,  5.949, 1.273, 0.180, 1.006, 1.536, -2.252, -0.472 )^T,
\end{align*}
where the disparity to the true value of $\beta$ results from a relatively small sample size of $n=100$.
We use $\beta_0 = ( 5, 4, 3, 2, 1, 0, -1, -2)^T$ for the starting point for our line search algorithm, a point for 
which Newton-Raphson fails due to a nearly singular Hessian matrix.  

We measure the performance of our algorithm in terms of the total number of iterations used, where each iteration 
requires the evaluation of the gradient, $\nabla \ell( \beta_k + \alpha_k p_k )$.  Typically, several iterations take 
place in an inner loop to find a step size $\alpha_k$ that meets the curvature condition \eqref{E:curvature mod}, a 
process that grows increasingly difficult as the estimates near the MLE since the rightmost term in \eqref{E:curvature 
mod} gets smaller in magnitude.  Once an acceptable step size is found, the parameter estimate $\beta_k$ is updated and 
a new search direction is determined, requiring another evaluation of the gradient.

Our algorithm took 54 iterations over 20 different search directions to get $\lVert \nabla \ell( \beta_k ) \rVert < 
0.01$ and arrive at an estimate for the MLE that differs from the \texttt{glm} result by 0.0117 in Euclidean distance 
(See Table~\ref{Table:Logistic}).  
Using the Polak-Ribi\`{e}re conjugate gradient method described in the previous section  resulted in comparably sharp 
MLE estimates (see Table~\ref{Table:Logistic}) in fewer iterations---28 over 11 search directions---a noticeable 
improvement. 

We also applied  SA with step size $1/k$ (setting $A=1$, $B=0$ in \eqref{E:SA step size}) from the same starting point 
$\beta_0$.  The choice of constants $A$ and $B$ in the step size is of course not likely to be optimal;
however, we want to apply SA without trial and error experimentation.  
After 10,000 iterations, the parameter estimates look nothing at all like the MLE (See Table~\ref{Table:Logistic}).  
The starting point $\beta_0$ is so far from the MLE and the step sizes so small that the algorithm does not converge in a reasonable amount of time.
Table~\ref{Table:step size} shows the first 20 step sizes used by SA and our line search. Our line search continues to 
use step sizes of relatively large magnitude even well into the process.  It should be noted that these 20 step sizes 
correspond to the first 20 iterations of SA but all 54 iterations of our line search algorithm since it spends several 
iterations finding an acceptable step size for each update.


% latex table generated in R 2.10.1 by xtable 1.5-6 package
% Tue May  4 17:37:01 2010
\begin{table}
\caption{Comparison of MLEs of $\beta$ for Example 1: MLE = \texttt{glm}, Steep = line search using steepest ascent, 
CG = line search using conjugate gradient, and SA =  SA with step size = $1/k$, terminated at 10,000 iterations,
$n$ = number of iterations.  Our 
proposed algorithm arrives at nearly identical MLE estimates to \texttt{glm}.}
\begin{center}
\begin{tabular}{rrrrrrrrrr}
  \hline
 & $n$ & $\beta[1]$ & $\beta[2]$ & $\beta[3]$ & $\beta[4]$ & $\beta[5]$ & $\beta[6]$ & $\beta[7]$ & $\beta[8]$ \\ 
True $\beta$ & & 0.000 & 2.000 & 2.000 & 1.000 & 1.000 & 0.000 & 0.000 & 0.000 \\ 
  $\hat{\beta}_{\textrm{MLE}}$ & & 0.635 & 5.949 & 1.273 & 0.180 & 1.006 & 1.536 & $-2.252$ & $-0.472$ \\ 
  $\hat{\beta}_{\textrm{Steep}}$ & 54 & 0.633 & 5.938 & 1.272 & 0.181 & 1.005 & 1.535 & $-2.249$ & $-0.470$ \\ 
  $\hat{\beta}_{\textrm{CG}}$ & 28 & 0.631 & 5.936 & 1.272 & 0.181 & 1.003 & 1.532 & $-2.244$ & $-0.470$ \\    
  $\hat{\beta}_{\textrm{SA}}$ & $10^4$ & 1.280 & 10.619 & 5.588 & 4.005 & 2.478 & $-7.153$ & 1.255 & 0.264 \\ 
  \hline
\end{tabular}
\end{center}
\label{Table:Logistic}
\end{table}

\begin{table}
\caption{The first 20 step sizes used by  SA (with step size $1/k$) and our algorithm for Example 1.  The step sizes 
used by our algorithm do not diminish like $1/k$.
}
\begin{center}
\begin{tabular}{rrrrrrrrr}
  \hline
  $k$ & $\alpha_{\textrm{SA}} = 1/k$  & $\alpha_{\textrm{Steep}}$ & $\alpha_{\textrm{CG}}$ \\ 
  \hline
1	&	1.000	&	0.192	&	0.192	\\
2	&	0.500	&	0.319	&	0.319	\\
3	&	0.333	&	0.403	&	0.416	\\
4	&	0.250	&	0.353	&	0.561	\\
5	&	0.200	&	0.380	&	0.491	\\
6	&	0.167	&	0.333	&	1.092	\\
7	&	0.143	&	0.420	&	0.359	\\
8	&	0.125	&	0.307	&	0.314	\\
9	&	0.111	&	0.442	&	0.275	\\
10	&	0.100	&	0.283	&	0.318	\\
11	&	0.091	&	0.483	&	0.278	\\
12	&	0.083	&	0.241	&	-	\\
13	&	0.077	&	0.745	&	-	\\
14	&	0.071	&	0.203	&	-	\\
15	&	0.067	&	1.224	&	-	\\
16	&	0.063	&	0.173	&	-	\\
17	&	0.059	&	2.510	&	-	\\
18	&	0.056	&	0.195	&	-	\\
19	&	0.053	&	0.944	&	-	\\
20	&	0.050	&	0.173	&	-	\\
  \hline
\end{tabular}
\end{center}
\label{Table:step size}
\end{table}



\section{Example: Ising model} \label{S:Examples:Ising}
In this example, we apply our gradient-based line search algorithm to an Ising model \citep{Ising} on a toroidal square 
lattice.  Ising models are exponential families where each entry in the square lattice takes the value of either zero 
or one.  A realized sample is shown in Figure~\ref{F:pottsimage}.  The sufficient statistic vector is two-dimensional, 
comprising the number of entries with value one and the number of ``neighbor'' entries with the same value.  Entries are 
considered ``neighbors'' if they are adjacent to one another horizontally or vertically (but not diagonally).  

Here we describe the toroidal square lattice as an $n \times n$ matrix $Y$ and each entry as $Y_{ij}$, where $i$ and $j
$ take values in $1, \ldots, n$ considered as a cyclical set (addition is done modulo $n$).  The sufficient statistic, 
$g(y)$, has components:
\begin{align*}
	g_1(y) &= \sum_{i=1}^n \sum_{j=1}^n I(Y_{ij}=1), \\
	g_2(y) &= \frac{1}{2} \sum_{i=1}^n \sum_{j=1}^n 
				\bigl[ I(Y_{ij}=Y_{i-1,j}) + I(Y_{ij}=Y_{i,j-1}) \\
							&\qquad \qquad \qquad + I(Y_{ij}=Y_{i+1,j}) + I(Y_{ij}=Y_{i,j+1}) \bigr ]
				,
\end{align*}  
where $I(\cdot)$ denotes the indicator function taking logical expressions to the numbers zero and one, false 
expressions to zero and true expressions to one.  

Because of the interdependence of neighboring entries in the lattice, there is no closed form expressing $\nabla \ell
( \eta)$ as in the logistic example.  Instead, we need to approximate $\nabla \ell( \eta)$ using MCMC as described by 
\eqref{E:nabla ell approx}.  As discussed in Section~\ref{section:MCMC approx}, Theorem~\ref{Thm:Line Search works} 
cannot be applied directly, but as we demonstrate here, satisfactory estimates are still attained.  The MCMC draws are 
performed here using the Swendsen-Wang algorithm \citep{Swendsen-Wang:1987,Swendsen-Wang:1990}, available in the contributed R 
package \texttt{potts} \citep{Geyer:potts}.

We choose $\eta = \left(0, \log(1 + \sqrt{2} ) \right)^T$ to generate a $32 \times 32$ lattice, which we will use as 
our observed data (Figure~\ref{F:pottsimage}).  This value for $\eta$ is of particular interest because it corresponds 
to the phase transition point \citep{Potts} and has been shown to be difficult to estimate \citep{Geyer:1990}.  In 
order to get a good estimate of the MLE to which we can compare our algorithm's results, we use Newton-Raphson starting 
at the true value of $\eta$ so that it will converge.

We apply our line search algorithm to this data using an arbitrary initial value of $\eta^{(0)} = ( 2, 0.001)$ and a 
fixed MCMC sample size of 10,000.  Our algorithm used 62 iterations (gradient evaluations) over 17 search directions to 
get  $\lVert \nabla \ell( \eta_k ) \rVert < 0.005$ and arrive at an estimate of the MLE that differs from Newton-Raphson
by 0.0037 (see Table~\ref{Table:Potts}).   Using the Polak-Ribi\`{e}re conjugate gradient method resulted in 
comparably sharp MLE estimates using 45 iterations over 7 search directions.  The total MCMC sample sizes used were $62\times10,0000 = 620,000$ and $45\times10,0000 = 450,000$, respectively.

%> cat( "Total iterations:",i.total, "\n")
%Total iterations: 62 
%> cat( "Outer iterations:",i, "\n")
%Outer iterations: 18 
%> cat( "Inner iterations:",i.total - i, "\n")
%Inner iterations: 44 



\begin{table}
\caption{Comparison of MLEs for $\eta$ for Example 2: MLE = Newton-Raphson starting from the true $\eta$, Steep = 
line search using steepest ascent, CG = line search using conjugate gradient, and SA = SA with step size = $1/k$.  All 
algorithms converged.}
\begin{center}
\begin{tabular}{rrrrrrlrr}
  \hline
%    &  &  &  & \multicolumn{1}{c}{inner}\\
  \multicolumn{1}{c}{} & 
  \multicolumn{1}{c}{MC Samples} &
  \multicolumn{1}{c}{$\eta[1]$} &
  \multicolumn{1}{c}{$\eta[2]$} \\
%  & \multicolumn{1}{c}{loop }\\
    &  \multicolumn{1}{c}{(thousands)} &  &  & \\
  \hline
True $\eta$  & & 0.000 & 0.881 \\ 
  $\hat{\eta}_{\textrm{MLE}}$ & & $-0.007$ & 0.896 \\ 
  $\hat{\eta}_{\textrm{Steep}}$ & 620 & $-0.011$ & 0.895 \\ 
  $\hat{\eta}_{\textrm{CG}}$ & 450 & $-0.008$ & 0.895 \\ 
  $\hat{\eta}_{\textrm{SA}}$ & 1368 & $-0.010$ & 0.895 \\ 
   \hline
\end{tabular}
\end{center}
\label{Table:Potts}
\end{table}

We also applied SA, again with step size $1/k$ from the same starting point $\eta^{(0)}$, and used
a MCMC sample size of 1,000 for gradient calculation.  
Here SA converged in 1368 iterations or 1,368,000 MC samples, comparable to our algorithm (see Table~\ref{Table:Potts}).  Table~\ref{Table:Potts step 
size} shows the first 17 step sizes used by SA and our line search.  The step sizes used by our line search are 
initially very small compared to $1/k$, but stay in a range of about $1/300$ to $1/3000$.  So, the $1/k$ step size used 
by SA in fact occasionally satisfies our curvature condition when $k$ is large. 

\begin{table}
\caption{The first 17 step sizes used by SA (with step size $1/k$) and our algorithm for Example 2.  The step sizes 
used by our algorithm are initially much smaller than $1/k$.
}
\begin{center}
\begin{tabular}{rrrrrrrrr}
  \hline
  $k$ & $\alpha_{\textrm{SA}} =1/k$  & $\alpha_{\textrm{Steep}}$ & $\alpha_{\textrm{CG}}$ \\ 
  \hline
1	&	1.0000	&	0.0029	&	0.0029	\\
2	&	0.5000	&	0.0005	&	0.0005	\\
3	&	0.3333	&	0.0017	&	0.0017	\\
4	&	0.2500	&	0.0013	&	0.0045	\\
5	&	0.2000	&	0.0017	&	0.0007	\\
6	&	0.1667	&	0.0011	&	0.0002	\\
7	&	0.1429	&	0.0021	&	0.0015	\\
8	&	0.1250	&	0.0009	&		\\
9	&	0.1111	&	0.0020	&		\\
10	&	0.1000	&	0.0007	&		\\
11	&	0.0909	&	0.0018	&		\\
12	&	0.0833	&	0.0006	&		\\
13	&	0.0769	&	0.0013	&		\\
14	&	0.0714	&	0.0006	&		\\
15	&	0.0667	&	0.0007	&		\\
16	&	0.0625	&	0.0003	&		\\
17	&	0.0588	&	0.0013	&		\\
  \hline
\end{tabular}
\end{center}
\label{Table:Potts step size}
\end{table}

%\section{Example: Social Network}
%Use the new \texttt{simulate.formula()} from \texttt{statnet}.
%In this example, we apply our algorithm to a simple social network model for which MCMC-MLE has been shown 
%Social networks are typically modeled as a random network represented by a matrix $X$, an $n \times n$ matrix where $n
%$ is the number of actors.
%Each entry $X_{ij}$ in the random matrix $X$ is a random variable representing a relation from actor $i$ to actor $j$, 
%such that:
%\[
%	X_{ij} = 
%	\begin{cases}
%		1 & \text{if a relationship exists \textit{from} actor $i$ \textit{to} actor $j$ (notation: $i \to j$)}\\
%		0 & \text{otherwise}
%	\end{cases}
%	\
%\]
%where $i$ and $j$ take values in $1, \ldots, n$, $i \neq j$, for a network with $n$ actors.  
%Note that $X_{ij}$ take only values of $0$ or $1$, reflecting our restriction on networks to those 
%with dichotomous relations, that is, the relation between a pair of actors is either present or absent.  
%In addition, we do not allow the possibility of $i \to i$ and always denote $X_{ii} = 0$.  
%In the special case that $X_{ij} = X_{ji}$ and thus the matrix $X$ is symmetric, the network is referred to as a 
%\textit{undirected} network or graph.  A network is \textit{directed} if it is not undirected.  
%   1: -0.666     NA 16.520  0.241  0.126   0.012 2288.79100 3770.25800     8.0
%   2: -0.888     NA  1.270  0.019  0.079   0.013   12.88560   54.58208     2.0
%   3: -0.904     NA  1.480  0.003  0.160   0.013    0.06350    0.32258    10.0
%> 
%> # summary
%> cat( "Precision:",cutoff.len, "\n")
%Precision: 0.01 
%> theta.frame <- data.frame( theta.MLE, theta.current )
%> theta.frame
%       theta.MLE theta.current
%edges -0.9071582    -0.9040951
%> cat( "Total iterations:",i.total, "\n")
%Total iterations: 24 
%> cat( "Outer iterations:",i, "\n")
%Outer iterations: 4 
%> cat( "Inner iterations:",i.total - i, "\n")
%Inner iterations: 20 

\chapter{Discussion}
We have presented a simple line search algorithm for finding the MLE of a regular exponential family when the MLE 
exists.  The algorithm avoids the trial and error experimentation of tuning parameters and starting points commonly associated with optimization routines
not invented by optimization specialists.  Our algorithm is modeled after algorithms discussed in optimization textbooks \citep{Fletcher,NW,Sun:2006},
all of which are safeguarded to ensure rapid automatic convergence.
%Because it only relies on first order derivatives, this approach avoids problems with near-singular Fisher information 
%matrices that plague methods like Newton-Raphson.  The reliant on a curvature condition for step size makes it less 
%sensitive to poor initial values that are problematic for MCMC-MLE and  SA in practice.

Convergence is guaranteed when the gradient can be calculated exactly.  Even when the gradient cannot be calculated 
exactly and is only estimable via MCMC, the algorithm is still useful in practice, as demonstrated by the Ising model 
example.  We have also described a way to construct and use confidence intervals to make convergence highly probable.

The algorithm can be computationally demanding.  When the current iteration approaches the solution, the 
curvature condition for step size becomes more difficult to satisfy and the method may require several iterations of 
MCMC sampling and perhaps an increase in MCMC sample size.  Eventual increase in MCMC sample size is unavoidable,
because the achievable accuracy is inversely proportional to the square root of the MCMC sample size, as in all Monte Carlo.
Thus we believe the best use of this algorithm is in combination with other faster methods like MCMC-MLE \citep{Geyer:1992}
or Newton-Raphson safeguarded by our line search algorithm.  Our 
algorithm should be used from ``long range'', when one has no good intuition for an initial value and is concerned about 
picking one that is far from the MLE.  The switch between types of search direction (steepest ascent, conjugate gradient,
or Newton) within our algorithm or the switch to another algorithm (such as MCMC-MLE \citep{Geyer:1992})
need not require manual intervention.  When used in combination in this
manner, we do not think the confidence intervals are necessary as the curvature condition is quite easily satisfied 
when the current iteration is far from the MLE.

One way to improve performance is to use conjugate gradient search directions rather than steepest ascent.  In our 
examples, this reduced the number of iterations by over 25\%.  However, in other problems we tried with different 
dimensionality, this performance varied significantly and it appears that no guarantee can be made about quantity of 
improvement in performance, though in all cases we examined, it never did worse.  This is no surprise, because the
necessity of ``preconditioning'' for good performance of the conjugate gradient algorithm is well known (but no
good ``preconditioner'' is available for maximum likelihood in exponential families).

There are several outstanding issues.  Most notably, we have not showed convergence of the algorithm when the gradient 
is approximated via MCMC.  This is a more difficult theoretical problem and is the motivation for stochastic 
approximation research.  
Further work is necessary to determine if one can adapt our restrictive curvature condition \eqref{E:Wolfe-mod} to the 
approach of \citet{Andrieu:2005} or \citet{Liang:2010} in MCMC stochastic approximation.  

Another remaining issue is the stopping criteria: what value should be chosen for $\epsilon$ in the exit condition
$\lVert  \nabla \ell( \eta_k ) \rVert < \epsilon$?  Because the value of $\lVert  \nabla \ell( \eta_k ) \rVert$ can only 
be approximated via MCMC, one cannot be certain if this condition is actually satisfied.  Here again, the switch to 
another methodology may be appropriate, though at least in our Ising model example, our use of 10,000 for the MCMC 
sample size and 0.005 for $\epsilon$ were successful in obtaining a reasonable parameter estimate. 

 A final remaining issue is estimation of Monte Carlo error of the estimates.  Here too we recommend switching to another
algorithm at the end.  The MCMC-MLE procedure gives accurate error estimates \citep{Geyer:1994}.
For very small steps these are essentially the same as the Monte Carlo error of a single unsafeguarded Newton-Raphson step,
so the method in \citep{Geyer:1994} can be used for either.

%A natural extension of this algorithm is to the case where the MLE may not exist.  
%This occurs with positive probability for discrete state space exponential families and 
%is a practical concern when estimating parameters \citep{Rinaldo:2009, Geyer:gdor}.  
%In such instances, the concave log likelihood continues to increase as $\eta \to \infty$ along 
%certain directions of recession.  Our algorithm can still be applied to such a setting to climb the 
%log likelihood until the gradient approaches zero and help identify the directions of recession. 




% References don't have to be double spaced either
\setstretch{1.3}
\bibliographystyle{ims}
%\bibliographystyle{imsart-nameyear}
\bibliography{References}
%\bibliography{/Users/saipuck/Tako/THESIS/References}


% text in appendices may be single spaced, if desired
\setstretch{1.3} % to 1.3 spacing
\appendix
\chapter{Proofs} \label{Section:Proofs}
Our algorithm minimizes the objective function $f$ by performing repeated one-dimensional updates.  We need the 
following lemma to transfer global properties of the objective function to the objective function restricted to a 
search direction.

%%%%%%%%%%% BEGIN LEMMA %%%%%%%%%%%%%%
\begin{lemma} \label{Lemma:f min} 
Suppose a function $f:\RR^n \to \RR$ is proper, lower semicontinuous, and strictly convex.  Then the minimum for $f$ 
exists and is unique if and only if every nonempty level set $\lev_{\leq \alpha} f$ is bounded.
\end{lemma}

%%%%%%%%%%% PROOF - LEMMA %%%%%%%%%%%%%%

\begin{proof}[Proof of Lemma~\ref{Lemma:f min}]
Assume every non-empty level set is bounded.  Then by Theorem~1.9 in \citet{Rockafellar}, the minimum of $f$ is finite 
and thus exists.  By strict convexity, this minimum is also unique.

Now assume the minimum for $f$ exists, denoted by $\min f$.  By assumption, the level set $\lev_{\leq \min f} f$ 
contains exactly one point.  By Corollary~8.7.1 in \citet{Rockafellar:1970}, the level sets $\lev_{\leq \alpha} f$ are 
bounded for every $\alpha$.  
\end{proof}

%%%%%%%%%%% PROOF - LINE SEARCH %%%%%%%%%%%%%%
\begin{proof}[Proof of Theorem~\ref{Thm:Line Search}]
The objective function $f$ is bounded below, strictly convex, and lower semicontinuous by assumption, and so by Lemma~
\ref{Lemma:f min}, the global minimum exists.  Then by Lemma~\ref{Lemma:f min}, all level sets of type $\lev_{\leq a} f
$, $a \in \RR$ are bounded in $\RR^n$.  Restricting the set to be along a search direction $p_k$ maintains the 
boundedness of these sets.  By Lemma~\ref{Lemma:f min} again, the minimum in this restriction exists and is unique.     

Then, unless $\nabla f( x_k ) = 0$ in which case $x_k$ is already the solution, for each $k$, we can uniquely define $
\alpha_{c_k}$ and $\alpha_{min_k}$ as follows: 
\begin{align}
%	\theta_k &= \cos^{-1} \left( \frac{ -\nabla f_k^T p_k }{ ||\nabla f_k|| \, ||p_k||} \right) \label{E:cosine} \\
	\nabla f( x_k + \alpha_{c_k} p_k)^T p_k &= c \nabla f(x_k)^T p_k \label{E:alphac} \\
	\nabla f( x_k + \alpha_{min_k} p_k)^T p_k &= 0. \label{E:alphamin} 
\end{align}
The point $\alpha_{c_k}$ is uniquely defined because it is the minimizer of $\alpha \mapsto f( x_k + \alpha p_k) - 
\alpha c \nabla f( x_k )^T p_k$.
These values appear on the $\alpha$-axis in Figure \ref{F:Wolfe-mod}.
%%%%%%%%%%% FIGURE %%%%%%%%%%%%%%
\begin{figure}
\centering
\scalebox{.4}{\input{Figures/Wolfe-mod.pstex_t}}
\caption{The acceptable region for $\alpha$ according to the curvature condition \eqref{E:Wolfe-mod} when restricting 
$f$ to direction $p_k$.}
\label{F:Wolfe-mod}
\end{figure}

%These values are illustrated on the $\alpha$-axis in Figure \ref{F:Wolfe-mod}.  
%Equation \eqref{E:alphamin} defines $\alpha_{min_k}$ to be the step size that would make the 
%gradient at $x_{k+1}$ equal to zero and hence minimizes $f(x_{k+1})$, equation \eqref{E:alphac} 
%defines \alpha_{c_k} to be the step size that would make the gradient at $x_{k+1}$ equal to 

By the strict convexity of $f$ and Theorem~2.14 in \citet{Rockafellar}, 
\begin{align}
	f( x_k + \alpha_{c_k} p_k ) &< f(x_k) +  \bigl[ \nabla f(x_k + \alpha_{c_k} p_k) \bigr]^T \alpha_{c_k} p_k. \notag 
\\
	\intertext{Applying \eqref{E:alphac} to the right hand side of the above gives}
	f( x_k + \alpha_{c_k} p_k ) &< f(x_k) + \alpha_{c_k} c \nabla f(x_k)^T p_k. \label{E:b-less-a}
	\end{align}	
(See points $a$ and $b$ in Figure \ref{F:Wolfe-mod}).

The subproblem $\alpha \mapsto f(x_k + \alpha p_k)$ is strictly convex and hence monotonically decreasing at $\alpha_k$ 
such that $\alpha_{c_k} \leq \alpha_k \leq \alpha_{min_k}$ (in Figure \ref{F:Wolfe-mod}, see points $b$ and $c$).  That 
is,
\begin{align}
	f( x_k + \alpha_{min}p_k) &\leq f( x_k + \alpha_k p_k) \leq f( x_k + \alpha_{c_k} p_k). \label{E:f-sandwich}
\end{align}
	
Combining the second inequality of \eqref{E:f-sandwich} with \eqref{E:b-less-a}, we have	
\begin{align}
	f( x_k + \alpha_k p_k ) &< f(x_k) + \alpha_{c_k} c \nabla f(x_k)^T p_k,  \label{E:decrease}
	\intertext{which can be rearranged as}
	f(x_k)-f( x_k + \alpha_k p_k ) &>  -\alpha_{c_k} c \nabla f(x_k)^T p_k. \label{E:f-lb}
\end{align}
This last inequality \eqref{E:f-lb} expresses a lower bound for the amount of decrease in our objective function at 
each step (the right-hand side is positive since $\nabla f(x_k)^T p_k < 0$ by assumption that $p_k$ is a descent 
direction).  It is this lower bound that we will use to cover the distance to the minimum of the objective function.  

We now turn our attention to \eqref{E:alphac}.  Define $x_{c_k} = x_k + \alpha_{c_k} p_k$.  Then
\begin{align}
	\nabla f( x_{c_k} )^T p_k &= c \nabla f(x_k)^T p_k. \notag
\end{align}
Subtracting $\nabla f(x_k)^T p_k$ from both sides gives
\begin{align}
	\left( \nabla f( {x_{c_k}} ) - \nabla f(x_k) \right )^T p_k &= ( c - 1 ) \nabla f(x_k)^T p_k.  \label{E:c-1}
\end{align}

%\textbf{NEW: SAI 12/05/10}\\
%Since we are considering a finite state space for $| \nabla^2 \ell(\eta) | \leq K$ for some constant K for all $\eta$ 
%\textbf{By Theorems 9.2 and 9.7 in \citet{Rockafellar}, since $\nabla^2 f(x)$ is assumed to be finite, 
%$\nabla f(x)$ is Lipschitz continuous relative to the convex set $\RR^n$.}

By Corollary~25.5.1 in \citet{Rockafellar:1970}, since $f$ is convex and differentiable on the open convex set $\NN$, 
it is actually continuously differentiable on $\NN$.  It is then Lipschitz continuously differentiable relative to any 
compact subset of $\NN$. 

Applying this to the compact level set $\lev_{\leq f(x_k)} f$, which is contained in $\NN$ by assumption, there exists 
a constant $L < \infty$ such that
	\begin{align}
		|| \nabla f(x) - \nabla f(\tilde{x}) || \leq L || x - \tilde{x} || \quad \text{for all $x, \tilde{x} \in \lev_
{\leq f(x_0)} f$}. \label{E:Lipschitz}
	\end{align} 

Applying \eqref{E:Lipschitz} to $x_{c_k}$ and $x_k$, we have
\begin{align}
|| \nabla f(x_{c_k}) - \nabla f(x_k) || &\leq L || x_{c_k} - x_k || \notag
\intertext{or}
|| \nabla f(x_{c_k}) - \nabla f(x_k) || &\leq L || \alpha_{c_k} p_k ||. \notag	
\end{align}

Multiplying both sides by $\lVert p_k \rVert$ gives
\begin{align}
|| \nabla f(x_{c_k}) - \nabla f(x_k) || \cdot ||p_k || &\leq \alpha_{c_k} L || p_k ||^2 \notag \\
\intertext{and by Cauchy-Schwarz this implies}
( \nabla f(x_{c_k}) - \nabla f(x_k) )^T p_k & \leq || \nabla f(x_{c_k}) - \nabla f(x_k) || \cdot ||p_k || \label
{E:from-Lipschitz} \\ 
	&\leq \alpha_{c_k} L || p_k ||^2. \notag
\end{align}
Substituting \eqref{E:c-1} into the left-hand side of this last inequality \eqref{E:from-Lipschitz} gives
\begin{align}
( c - 1 ) \nabla f(x_k)^T p_k &\leq \alpha_{c_k} L || p_k ||^2 \notag\\
\intertext{or}
-\alpha_{c_k} &\leq \frac{( 1 - c )}{L} \frac{ \nabla f(x_k)^T p_k}{ || p_k ||^2}. \label{E:-alpha}
\end{align}

%%%%%%
%%%%%%
Write out the first $k+1$ inequalities of \eqref{E:f-lb}:
\begin{align}
\begin{split}
	f( x_1 ) &< f(x_0) + \alpha_{c_0} c \nabla f(x_0)^T p_0 \\
	f( x_2 ) &< f(x_1) + \alpha_{c_1} c \nabla f(x_1)^T p_1  \\
	\ldots  \\
	f( x_{k} ) &< f(x_{k-1}) + \alpha_{c_{k-1}} c \nabla f(x_{k-1})^T p_{k-1}  \\
	f( x_{k+1} ) &< f(x_k) + \alpha_{c_k} c \nabla f(x_k)^T p_k 
\end{split}
	\label{E:to telescope}
\end{align}
Telescoping the right-hand side of \eqref{E:to telescope},
\begin{align*}
	f( x_{k+1} ) &< f(x_0) + c \sum_{j=0}^{k} \alpha_{c_j} \nabla f(x_j)^T p_j. \notag
\end{align*}
Noting that $\nabla f(x_j)^T p_j < 0$, we can substitute our upper bound \eqref{E:-alpha} for $-\alpha_{c_j}$ in the 
right-hand side above,
\begin{align}
	f( x_{k+1} ) &< f(x_0) - c \sum_{j=0}^{k} \frac{( 1 - c )}{L} \frac{ \nabla f(x_j)^T p_j}{ || p_j ||^2 } \nabla f
(x_j)^T p_j \notag \\
	\intertext{which simplies to}
	f( x_{k+1} ) &< f(x_0) - c \sum_{j=0}^{k} \frac{( 1 - c )}{L} \frac{ (\nabla f(x_j)^T p_j)^2}{ || p_j ||^2 }. 
\notag
\end{align}

Because $f(x)$ is bounded below by assumption, there exists some $M < \infty$ such that $f(x_0) - f(x_{k+1}) < M$ for 
all $k$. Then rearranging the above yields,

\begin{align*}
	\frac{c( 1 - c )}{L} \sum_{j=0}^{k}   \frac{ ( \nabla f(x_j)^T p_j )^2}{ || p_j ||^2 } &< M < \infty.
 \end{align*}
The angle $\theta_j$ between the search direction $p_k$ and steepest descent direction $-\nabla f_k$ can be expressed 
by $\cos \theta_j = \frac{ -\nabla f(x_j)^T p_j}{||\nabla f_j|| \cdot || p_j||}$.  Substituting this into the equation 
above and taking $k \to \infty$,
\begin{align*}
	\frac{c( 1 - c )}{L} \sum_{j=0}^{\infty}  \cos^2 \theta_j ||\nabla f(x_j)||^2 &< \infty.
\end{align*}
Since $0 < c < 1$,
\begin{align}
	\sum_{j=0}^{\infty}  \cos^2 \theta_j ||\nabla f(x_j)||^2 &< \infty. \label{E:Z's}
\end{align}

The convergent series in \eqref{E:Z's} implies that 
\begin{align*}
	\cos^2 \theta_k || \nabla f(x_k) ||^2 &\to 0 \text{ as } k \to \infty.
\end{align*}
With the additional restriction on the search direction $p_k$ such that $\cos \theta_k \geq \delta > 0$ for some choice 
of $\delta$, for all choices of $k$, we get the desired convergence result of
\begin{align*}
	\lim_{k \to \infty} || \nabla f(x_k) || &= 0.
\end{align*}
\end{proof}
%%%%%%%%%%% END PROOF %%%%%%%%%%%%%%
The inequality \eqref{E:Z's} has been referred to as \emph{Zoutendijk's condition} \citep{NW}, though we arrive at this 
result via different assumptions.\\  












\newpage
\section{Proof of Exponential family log likelihood maximization}
\begin{proof}[Proof of Theorem~\ref{Thm:log like max}]
Let $f(\cdot)$ represent the negative log likelihood $- \ell(\cdot)$, the objective function to be minimized.  We proceed from the perspective of a minimization of a function $f(\cdot)$ since this is the convention in the optimization literature \citep{NW}.  \citet{Okabayashi:longrange} prove a very similar result for when the minimum exists, and this proof mimics that one.

The negative log likelihood function $-\ell(\cdot)$ is strictly convex by \eqref{E:nabla2 ell}, and continuous since it is infinitely differentiable by LEHMAN.

It is bounded below by Corollary CHARLIE.

The objective function $f$ is bounded below, strictly convex, and lower semicontinuous by assumption, and so by Lemma~
\ref{Lemma:f min}, the global minimum exists.  
Then by Lemma~\ref{Lemma:f min}, all level sets of type $\lev_{\leq a} f
$, $a \in \RR$ are bounded in $\RR^n$.  Restricting the set to be along a search direction $p_k$ maintains the 
boundedness of these sets.  By Lemma~\ref{Lemma:f min} again, the minimum in this restriction exists and is unique.     

Then, unless $\nabla f( x_k ) = 0$ in which case $x_k$ is already the solution, for each $k$, we can uniquely define $
\alpha_{c_k}$ and $\alpha_{min_k}$ as follows: 
\begin{align}
%	\theta_k &= \cos^{-1} \left( \frac{ -\nabla f_k^T p_k }{ ||\nabla f_k|| \, ||p_k||} \right) \label{E:cosine} \\
	\nabla f( x_k + \alpha_{c_k} p_k)^T p_k &= c \nabla f(x_k)^T p_k \label{E:alphac} \\
	\nabla f( x_k + \alpha_{min_k} p_k)^T p_k &= 0. \label{E:alphamin} 
\end{align}
The point $\alpha_{c_k}$ is uniquely defined because it is the minimizer of $\alpha \mapsto f( x_k + \alpha p_k) - 
\alpha c \nabla f( x_k )^T p_k$.
These values appear on the $\alpha$-axis in Figure \ref{F:Wolfe-mod}.
%%%%%%%%%%% FIGURE %%%%%%%%%%%%%%
\begin{figure}
\centering
\scalebox{.4}{\input{Figures/Wolfe-mod.pstex_t}}
\caption{The acceptable region for $\alpha$ according to the curvature condition \eqref{E:Wolfe-mod} when restricting 
$f$ to direction $p_k$.}
\label{F:Wolfe-mod}
\end{figure}

%These values are illustrated on the $\alpha$-axis in Figure \ref{F:Wolfe-mod}.  
%Equation \eqref{E:alphamin} defines $\alpha_{min_k}$ to be the step size that would make the 
%gradient at $x_{k+1}$ equal to zero and hence minimizes $f(x_{k+1})$, equation \eqref{E:alphac} 
%defines \alpha_{c_k} to be the step size that would make the gradient at $x_{k+1}$ equal to 

By the strict convexity of $f$ and Theorem~2.14 in \citet{Rockafellar}, 
\begin{align}
	f( x_k + \alpha_{c_k} p_k ) &< f(x_k) +  \bigl[ \nabla f(x_k + \alpha_{c_k} p_k) \bigr]^T \alpha_{c_k} p_k. \notag 
\\
	\intertext{Applying \eqref{E:alphac} to the right hand side of the above gives}
	f( x_k + \alpha_{c_k} p_k ) &< f(x_k) + \alpha_{c_k} c \nabla f(x_k)^T p_k. \label{E:b-less-a}
	\end{align}	
(See points $a$ and $b$ in Figure \ref{F:Wolfe-mod}).

The subproblem $\alpha \mapsto f(x_k + \alpha p_k)$ is strictly convex and hence monotonically decreasing at $\alpha_k$ 
such that $\alpha_{c_k} \leq \alpha_k \leq \alpha_{min_k}$ (in Figure \ref{F:Wolfe-mod}, see points $b$ and $c$).  That 
is,
\begin{align}
	f( x_k + \alpha_{min}p_k) &\leq f( x_k + \alpha_k p_k) \leq f( x_k + \alpha_{c_k} p_k). \label{E:f-sandwich}
\end{align}
	
Combining the second inequality of \eqref{E:f-sandwich} with \eqref{E:b-less-a}, we have	
\begin{align}
	f( x_k + \alpha_k p_k ) &< f(x_k) + \alpha_{c_k} c \nabla f(x_k)^T p_k,  \label{E:decrease}
	\intertext{which can be rearranged as}
	f(x_k)-f( x_k + \alpha_k p_k ) &>  -\alpha_{c_k} c \nabla f(x_k)^T p_k. \label{E:f-lb}
\end{align}
This last inequality \eqref{E:f-lb} expresses a lower bound for the amount of decrease in our objective function at 
each step (the right-hand side is positive since $\nabla f(x_k)^T p_k < 0$ by assumption that $p_k$ is a descent 
direction).  It is this lower bound that we will use to cover the distance to the minimum of the objective function.  

We now turn our attention to \eqref{E:alphac}.  Define $x_{c_k} = x_k + \alpha_{c_k} p_k$.  Then
\begin{align}
	\nabla f( x_{c_k} )^T p_k &= c \nabla f(x_k)^T p_k. \notag
\end{align}
Subtracting $\nabla f(x_k)^T p_k$ from both sides gives
\begin{align}
	\left( \nabla f( {x_{c_k}} ) - \nabla f(x_k) \right )^T p_k &= ( c - 1 ) \nabla f(x_k)^T p_k.  \label{E:c-1}
\end{align}

%\textbf{NEW: SAI 12/05/10}\\
By \eqref{E:nabla2 ell}, $\nabla^2 \ell(\eta)$ is bounded for finite state space $g(\YY)$, which is true by assumption (and the only case of interest to us).  Thus $| \nabla^2 f(x) | \leq K$ for some constant K for all $x$. 
\textbf{Then by Theorems 9.2 and 9.7 in \citet{Rockafellar}, 
%since $\nabla^2 f(x)$ is assumed to be finite, 
$\nabla f(x)$ is Lipschitz continuousLY DIFFABLE? relative to the convex set $\RR^n$.}

%By Corollary~25.5.1 in \citet{Rockafellar:1970}, since $f$ is convex and differentiable on the open convex set $\NN$, 
%it is actually continuously differentiable on $\NN$.  It is then Lipschitz continuously differentiable relative to any 
%compact subset of $\NN$. 

Applying this to the compact level set $\lev_{\leq f(x_k)} f$, 
%which is contained in $\NN$ by assumption, 
there exists a constant $L < \infty$ such that
	\begin{align}
		|| \nabla f(x) - \nabla f(\tilde{x}) || \leq L || x - \tilde{x} || \quad \text{for all $x, \tilde{x} \in \lev_
{\leq f(x_0)} f$}. \label{E:Lipschitz}
	\end{align} 

Applying \eqref{E:Lipschitz} to $x_{c_k}$ and $x_k$, we have
\begin{align}
|| \nabla f(x_{c_k}) - \nabla f(x_k) || &\leq L || x_{c_k} - x_k || \notag
\intertext{or}
|| \nabla f(x_{c_k}) - \nabla f(x_k) || &\leq L || \alpha_{c_k} p_k ||. \notag	
\end{align}

Multiplying both sides by $\lVert p_k \rVert$ gives
\begin{align}
|| \nabla f(x_{c_k}) - \nabla f(x_k) || \cdot ||p_k || &\leq \alpha_{c_k} L || p_k ||^2 \notag \\
\intertext{and by Cauchy-Schwarz this implies}
( \nabla f(x_{c_k}) - \nabla f(x_k) )^T p_k & \leq || \nabla f(x_{c_k}) - \nabla f(x_k) || \cdot ||p_k || \label
{E:from-Lipschitz} \\ 
	&\leq \alpha_{c_k} L || p_k ||^2. \notag
\end{align}
Substituting \eqref{E:c-1} into the left-hand side of this last inequality \eqref{E:from-Lipschitz} gives
\begin{align}
( c - 1 ) \nabla f(x_k)^T p_k &\leq \alpha_{c_k} L || p_k ||^2 \notag\\
\intertext{or}
-\alpha_{c_k} &\leq \frac{( 1 - c )}{L} \frac{ \nabla f(x_k)^T p_k}{ || p_k ||^2}. \label{E:-alpha}
\end{align}

%%%%%%
%%%%%%
Write out the first $k+1$ inequalities of \eqref{E:f-lb}:
\begin{align}
\begin{split}
	f( x_1 ) &< f(x_0) + \alpha_{c_0} c \nabla f(x_0)^T p_0 \\
	f( x_2 ) &< f(x_1) + \alpha_{c_1} c \nabla f(x_1)^T p_1  \\
	\ldots  \\
	f( x_{k} ) &< f(x_{k-1}) + \alpha_{c_{k-1}} c \nabla f(x_{k-1})^T p_{k-1}  \\
	f( x_{k+1} ) &< f(x_k) + \alpha_{c_k} c \nabla f(x_k)^T p_k 
\end{split}
	\label{E:to telescope}
\end{align}
Telescoping the right-hand side of \eqref{E:to telescope},
\begin{align*}
	f( x_{k+1} ) &< f(x_0) + c \sum_{j=0}^{k} \alpha_{c_j} \nabla f(x_j)^T p_j. \notag
\end{align*}
Noting that $\nabla f(x_j)^T p_j < 0$, we can substitute our upper bound \eqref{E:-alpha} for $-\alpha_{c_j}$ in the 
right-hand side above,
\begin{align}
	f( x_{k+1} ) &< f(x_0) - c \sum_{j=0}^{k} \frac{( 1 - c )}{L} \frac{ \nabla f(x_j)^T p_j}{ || p_j ||^2 } \nabla f
(x_j)^T p_j \notag \\
	\intertext{which simplies to}
	f( x_{k+1} ) &< f(x_0) - c \sum_{j=0}^{k} \frac{( 1 - c )}{L} \frac{ (\nabla f(x_j)^T p_j)^2}{ || p_j ||^2 }. 
\notag
\end{align}

Because $f(x)$ is bounded below by assumption, there exists some $M < \infty$ such that $f(x_0) - f(x_{k+1}) < M$ for 
all $k$. Then rearranging the above yields,

\begin{align*}
	\frac{c( 1 - c )}{L} \sum_{j=0}^{k}   \frac{ ( \nabla f(x_j)^T p_j )^2}{ || p_j ||^2 } &< M < \infty.
 \end{align*}
The angle $\theta_j$ between the search direction $p_k$ and steepest descent direction $-\nabla f_k$ can be expressed 
by $\cos \theta_j = \frac{ -\nabla f(x_j)^T p_j}{||\nabla f_j|| \cdot || p_j||}$.  Substituting this into the equation 
above and taking $k \to \infty$,
\begin{align*}
	\frac{c( 1 - c )}{L} \sum_{j=0}^{\infty}  \cos^2 \theta_j ||\nabla f(x_j)||^2 &< \infty.
\end{align*}
Since $0 < c < 1$,
\begin{align}
	\sum_{j=0}^{\infty}  \cos^2 \theta_j ||\nabla f(x_j)||^2 &< \infty. \label{E:Z's}
\end{align}

The convergent series in \eqref{E:Z's} implies that 
\begin{align*}
	\cos^2 \theta_k || \nabla f(x_k) ||^2 &\to 0 \text{ as } k \to \infty.
\end{align*}
With the additional restriction on the search direction $p_k$ such that $\cos \theta_k \geq \delta > 0$ for some choice 
of $\delta$, for all choices of $k$, we get the desired convergence result of
\begin{align*}
	\lim_{k \to \infty} || \nabla f(x_k) || &= 0.
\end{align*}

\end{proof}


%%%%%%%%%%% BEGIN PROOF - Line Search Works  %%%%%%%%%%%%%%
Theorem 3.1 shows that the gradient of the objective function converges to 0.  The proof for Theorem 3.2 is concerned 
with the conditions for mapping this convergence to the convergence of the iterated parameter estimates $\eta_k$ to the 
unique MLE.  In particular, the mapping from $\eta_k$ to the gradient must be globally invertible.

\begin{proof}[Proof of Theorem~\ref{Thm:Line Search works}]

The Fisher information for a regular exponential family is non-singular by \eqref{E:FI} and thus invertible.  If we 
consider the map defined by
\begin{align*}
	h(\eta) = \nabla c(\eta)
\end{align*}
where $c$ is the cumulant function \eqref{E:cumulant}, its first derivative matrix is
\begin{align}
	\nabla h(\eta) = \nabla^2 c(\eta) = I(\eta) \label{E:nabla h eta}
\end{align}
which is again non-singular.  Since this is true for any $\eta$, by the inverse function theorem, $h$ is everywhere 
locally invertible.

In fact, $h$ is globally invertible. For any $\mu$ in the range of $h$, consider the function
\begin{align*}
	q(\eta) = \mu^T\eta - c(\eta).
\end{align*}
Since $\nabla^2 q(\eta) = - I(\eta)$ by \eqref{E:nabla h eta}, $q$ is strictly concave.  Therefore, a maximizer for $q
$, call it $\hat{\eta}$, is unique if it exists and satisfies the first-order condition
\begin{align*}
	\nabla q( \hat{\eta} ) = 0.
\end{align*}
This in turn implies that
\begin{align*}
	\mu - h(\hat{\eta}) = 0
\end{align*} 
or
\begin{align*}
	\mu = h( \hat{\eta} ). \label{E:eta hat}
\end{align*}
Because of the assumption that $\mu$ is in the range of $h$, this means that $\hat{\eta}$ in fact exists, and by the 
strict concavity of $q$, is unique.  This implies that $h$ must be one-to-one and hence globally invertible.


Since $c$ is infinitely differentiable by Theorem~2.7.1 in \citet{TSH}, so is $h$, and by the inverse function theorem, 
so is $h^{-1}$ (even if we do not know the form of $h^{-1}$).  The first derivative of $h^{-1}$ can be expressed as
\begin{align*}
	\nabla h^{-1}(\mu) = \left [ \nabla h(\eta) \right ]^{-1} = \left [ I(\eta) \right ]^{-1}
\end{align*}
when $\mu = h(\eta)$ and is thus non-singular everywhere, including at the MLE of $\eta$, $\etaMLE$.

Thus our algorithm, which concludes that $ || \nabla \ell( \eta_k) ||  = || g(y) - h(\eta_k) || \to 0$, implies that 
\begin{align*}
	\mu_k = h(\eta_k) \to g(y), 
\end{align*}
or
\begin{align*}
	h^{-1}(\mu_k)  \to h^{-1}\left (g(y) \right),
\end{align*}
or
\begin{align*}
	\eta_k  \to  \etaMLE.
\end{align*}

\end{proof}
%%%%%%%%%%% END PROOF %%%%%%%%%%%%%%

%\chapter{Proofs of Chapter \ref{sec:gibbs} Results}\label{sec:app1}
%\input{GibbsProofs}
%
%\chapter{Proofs of Chapter \ref{sec:examples} Results}\label{sec:app2}
%\input{ExampleProofs}




\end{document}