\section{Long range search algorithm for MLEs} \label{Section:Algorithm}
We now present our search algorithm, which will converge to the optimum for any 
continuously differentiable, strictly 
convex (or concave) function.  The algorithm and requirements are presented in 
Theorem~\ref{Thm:log like max}.  Proofs 
are in Appendix~\ref{Section:Proofs}. 


We apply the algorithm in Theorem~\ref{Thm:Line Search works} to the specific setting 
of finding the MLE in a regular 
exponential family when the MLE is known to exist and be unique and the gradient can 
be calculated exactly.

%\section{Line search convergence}
In order to be consistent with the general optimization literature \citep
{Fletcher,NW}, we state our algorithm in this 
section from the perspective of a minimization problem.  Thus we wish to minimize a 
real-valued objective function $-\ell(\eta)$ 
defined on $\RR^d$.  


%%%%%%%%%%%% BEGIN THEOREM %%%%%%%%%%%%%%
%\begin{theorem}[Convex function root search] \label{Thm:Line Search}
%Consider any line search of the form 
%\begin{align}
%	x_{k+1} &= x_k + \alpha_k p_k \label{E:x_update}
%\end{align}
%used to minimize the objective function $f$, which satisfies the following 
%assumptions:
%\begin{enumerate}
%	\item The objective function $f$ is bounded below in $\RR^n$. \label{ass:one}
%	\item The objective function $f$ is proper, lower semicontinuous, and strictly 
%convex.
%	\item The objective function $f$ is differentiable in an open set $\NN$ containing 
%the level set $\lev_{\leq f
%(x_0)} f$, which is bounded, where $x_0$ is the starting point of the iteration.
%%\item The \emph{step length} $\alpha_k$ is greater than $0$ unless $\nabla f(x_k) = 0$, 
%% in which case $x_k$ is already the solution and the search is complete.
%\item The \emph{search direction} $p_k$ is a non-zero \emph{descent direction} \label
%{ass:four}
%such that the angle $\theta_k$ between the search direction $p_k$ and steepest descent 
%direction $-\nabla f(x_k)$ is 
%restricted to be less than 90 degrees by
%\begin{align*}
%\cos \theta_k \geq \delta > 0
%\end{align*}
% for some fixed $\delta > 0$.  
%\end{enumerate}
%
%Then, unless $\nabla f(x_k) = 0$, in which case $x_k$ is already the solution and the 
%search is complete, it is 
%possible to find a step length $\alpha_k$ that satisfies the \emph{curvature 
%condition}
%\begin{align}
%	c \nabla f(x_k)^T p_k &\leq \nabla f( x_k + \alpha_k p_k)^T p_k \leq 0 \label
%{E:Wolfe-mod}
%\end{align}
%for some fixed $0 < c < 1$.
%
%Furthermore, repeated iterations of \eqref{E:x_update} satisfying assumptions~\ref
%{ass:one} through~\ref{ass:four} and \eqref{E:Wolfe-mod} will produce 
%a sequence, $x_1, x_2, \ldots$ such that
%\begin{align*}
%	\lim_{k \to \infty} || \nabla f(x_k) || = 0.
%\end{align*}
%\end{theorem}

%\section{Line search for MLE estimation}
\begin{theorem}[Exponential family log likelihood maximization] \label{Thm:log like 
max}
Consider any line search of the form 
\begin{align}
	\eta_{k+1} &= \eta_k + \alpha_k p_k \label{E:eta_update}
\end{align}
used to minimize the negative log likelihood function $-\ell(\cdot)$ of a regular 
exponential family on a finite sample space, where the \emph{search direction} $p_k$ 
is a non-zero \emph{descent direction}
such that the angle $\theta_k$ between the search direction $p_k$ and steepest descent 
direction $-\nabla \ell(\eta_k)$ is 
restricted to be less than 90 degrees by
\begin{align*}
\cos \theta_k \geq \delta > 0
\end{align*}
 for some fixed $\delta > 0$.  

Then, unless $\nabla \ell(\eta_k) = 0$, in which case $\eta_k$ is already the solution 
and the search is complete, it is 
possible to find a step length $\alpha_k$ that satisfies the \emph{curvature 
condition}
\begin{align}
	0 \leq \nabla \ell( \eta_k + \alpha_k p_k)^T p_k  \leq c \nabla \ell(\eta_k)^T p_k  
\label{E:Wolfe-ll}
\end{align}
for some fixed $0 < c < 1$.

Furthermore, repeated iterations of \eqref{E:eta_update} along a descent direction 
satisfying \eqref{E:Wolfe-ll} will produce a sequence, $\eta_1, \eta_2, \ldots$ such 
that
\begin{align*}
	\lim_{k \to \infty} || \nabla \ell(\eta_k) || = 0.
\end{align*}
\end{theorem}



We apply Theorem~\ref{Thm:log like max} to the setting of exponential families to find 
the MLE when it exists.  

\begin{theorem}[] \label{Thm:Line Search works}
For a regular exponential family with minimal representation where the MLE exists, the 
line search described in 
Theorem~\ref{Thm:log like max} can be applied to the negative log likelihood function 
$-\ell(\eta)$ so that a search 
starting at any $\eta_0 \in \Xi$ will converge to the MLE of $\eta$.
\end{theorem}

The issue of MLE existence is a problem in computational geometry, not an optimization 
problem, so we do not address it 
here.  See \citep{Geyer:gdor,Rinaldo:2009} and references cited therein.


\chapter{Algorithm pseudocode}
The algorithm is summarized on the next page.  Below are a few comments about notation 
and equations:
\begin{itemize}
\item $C$ is the convex support.
\item $Y$ is the random variable in the sample space.
\item $g(\cdot)$ is the function that maps $y$ to its natural statistics.
\item $\yobs$ is the observed observed, $g(\yobs)$ the natural statistics of the observed data.
\item $\con()$ is the convex hull of a set of points.
\item $\bd()$ is the boundary of a set of points.
\item $\norm{\cdot}$ is the Euclidean norm.
\item The curvature condition \eqref{E:Wolfe-ll} used here comes from the search 
algorithm of \citet{Okabayashi:longrange} and is necessary for guaranteeing 
convergence to the MLE when it exists.
%\item $\bar{Y}_m$ is the sample mean, $\frac{1}{m}\sum_{i=1}^m g(Y_i)$
\end{itemize}
\newpage
{\small
\noindent \textbf{LCM MLE Algorithm}

\noindent \begin{algorithmic}[1]
\State Get an initial value, $\eta_1 = (0, \ldots 0)$.
\State \textbf{Sample} $Y_1, \ldots, Y_m$ from the distribution with parameter $\eta_{1}$.  
\State Set $ty.hull = \con(g(Y_1), \ldots, g(Y_m) )$.
\State Approximate 
\begin{align} \label{E:nabla ell approx}
\nabla \ell( \eta_1) \approx g(\yobs) - \frac{1}{m}\sum_{i=1}^m g(Y_i).
\end{align}
\State Set $p_1 = \nabla \ell( \eta_1)$, $k=1$, $LCM.flag$ = FALSE, $c=0.2$.\\
%, $LCM.k=1$, 
%\State , $on.boundary$ = FALSE, $on.interior$ = FALSE.
%\State Set $face.cutoff = 0.30$.

\While{$\lVert \nabla \ell( \eta_k) \rVert > \epsilon$}
\State \textbf{Find} a step size $\alpha_k$ that satisfies the \textit{curvature condition}
\begin{align*}%\label{E:curvature}
	 0 & \leq \nabla \ell( \eta_k + \alpha_k p_k)^T p_k \leq c \nabla \ell(\eta_k)^T p_k.
\end{align*}
%\State for some $0 < c < 1$. % (Calculating $\nabla \ell(\cdot)$ requires more sampling)

\State $\eta_{k+1} = \eta_k + \alpha_k p_k$.
\State \textbf{Sample} $Y_1, \ldots, Y_m$ from the distribution with parameter $\eta_{k+1}$.
\If{$LCM.flag=$TRUE}
	\State Restrict sample points to those in empirical face.
\EndIf
\State Call the resulting sample points $g(Y_{(1)}), \ldots, g(Y_{(k)})$.
\State \textbf{Update} $ty.hull$ to reflect new sample points, $g(Y_{(1)}), \ldots, g(Y_{(k)})$.\\
\State \textbf{Question: $g(\yobs) \in \bd( \con(g(Y_{(1)}), \ldots, g(Y_{(k)}) ))$? }
%\If{Yes, $g(\yobs)$ is outside the convex hull}
%	\State  Keep sampling.  
%%	\State $on.boundary$ = FALSE.
%	\State Skip to estimating $\nabla \ell( \eta_{k+1})$.
%\ElsIf{No, $\yobs$ is inside the convex hull}
%	\State \textbf{The MLE exists}.  Finding it should be straightforward, (except 
%when 
%	\State it isn't ...) 
%	\State $on.boundary$ = FALSE, $on.interior$ = TRUE.
%	\State Skip to estimating $\nabla \ell( \eta_{k+1})$.
%\Else 
\If{Yes} %\Comment{$g(\yobs)$ is on the boundary of convex hull}
	\State \textbf{Either:}
	\State (1) the MLE exists but the sample just touches $g(\yobs)$, 
	\State (2) the MLE does \emph{not} exist; both $g(\yobs)$ and our sample points  
	\State are touching the boundary of $C$.
	\State \textbf{Find} the empirical face $F$ of $ty.hull$ on which $g(\yobs)$ lies.
%	\State \textbf{Calculate} $face.prop$, the proportion of the sample that falls on
%	\State this face.\\
%\newpage
	\If{$>60\%$ of the sample points are on $F$}
		\State Conclude that we are in case (2); case (1) is very unlikely.
		\State \textbf{Set} $LCM.flag$ = TRUE.
	\EndIf
\EndIf\\
%\State \textbf{Calculate} $\nabla \ell( \eta_{k+1})$ as follows: \label{Calc:nabla}
%\If{ $LCM.flag$ == FALSE }
%	\State $\nabla \ell( \eta_{k+1}) \approx g(\yobs) - \frac{1}{m}\sum_{i=1}^m g(Y_i)$
%\Else
	\State Approximate
	\begin{align} \label{E:nabla ell approx LCM}
	\nabla \ell( \eta_{k+1}) \approx g(\yobs) - \frac{1}{k}\sum_{i=1}^k g(Y_{(i)}).
	\end{align}

%, where the sample mean is restricted 
%	\State to empirical face points.
%\EndIf\\

\State \textbf{Find} the new search direction $p_{k+1}$, which must be an ascent 
direction.
\Statex This may involve using a direction normal to empirical face, or a regression 
\Statex over previous $\eta_k$ values, or simply using steepest ascent, $\nabla \ell
( \eta_{k+1})$.

\State $k = k + 1$.
\EndWhile
\end{algorithmic}
}
\chapter{Refinements of algorithm}

In Theorem~\ref{Thm:log like max}, we restricted our search direction $p_k$ to be a 
descent direction, so that $\nabla f
(x_k)^T p_k < 0$ or, alternatively, the angle $\theta_k$ between the search direction 
$p_k$ and steepest descent 
direction $-\nabla f(x_k)$ is less than 90 degrees.  However, this still leaves many 
possibilities for the choice of 
$p_k$ other than steepest descent.  In addition, we have specified restrictions on the 
step size $\alpha_k$ in the 
curvature condition \eqref{E:Wolfe-ll} with $0 < c < 1$, but it would be useful to 
know if certain values of $c$ are 
better than others.

\section{Search directions}
In our examples in Section~\ref{S:Examples}, we default to steepest descent directions 
in our implementation for 
transparency.  Although often effective in early steps, steepest descent directions 
can result in a zigzagging 
trajectory of the sequence $x_k$ \citep{Sun:2006}.  Conjugate gradient methods address 
this phenomena and cover the 
sample space more efficiently \citep{NW}.  It is easy to implement a variant of the 
Polak-Ribi\`{e}re method
\citep[pp.~120--122]{NW} here, requiring little more in terms of calculation or 
storage.  The search direction $p_k$ would update 
with an extra intermediate step as follows:
\begin{align*}
	\gamma_{k+1}^{PR} &= \max \left( 0, \frac{ [ \nabla f( x_{k+1}) ]^T( \nabla f( x_{k+1} ) - \nabla f( x_k) )  }
{ \lVert \nabla f( x_k) \rVert^2 } \right )\\
	p_{k+1} &= -\nabla f( x_{k+1}) + \gamma_{k+1}^{PR} \, p_k.
\end{align*}
Note that when $\gamma_{k+1}^{PR} = 0$, $p_{k+1}$ will be just $-\nabla f( x_{k+1})$, 
the direction of steepest 
descent, and thus serves as a ``reset''.  The curvature condition \eqref{E:Wolfe-ll} 
guarantees that this method always 
yields a descent direction for $p_{k+1}$ and thus Theorem~\ref{Thm:log like max} still 
holds.  


\subsection{Further Refinement on Search Directions}
We require that the search direction, $p_k$, be an ascent direction, but make no 
further restrictions.
If we use only steepest ascent directions, using $p_k = \nabla \ell( \eta_k)$, our 
algorithm 
may not be efficient in scaling the log likelihood due to excessive ``zig-zagging" as 
illustrated in Figure 
\ref{F:zigzag}.  This is especially problematic when the MLE does not exist in the 
conventional sense---the MLE is actually off at infinity and a zig-zagging route may 
take an especially long time to realize this.  \citet{Okabayashi:longrange} made the 
case for using search directions chosen according to Polak-Ribiere conjugate gradient 
updates.  

Here we suggest another alternative that is especially useful when it appears that the 
observed statistic might fall on the boundary of the convex support.  Because our 
algorithm computes the normal cone when it finds an empirical face, we suggest using 
the average normal cone vector as a search direction, checking first that it is an 
ascent direction to ensure the algorithm proceeds uphill.  As theory in the previous 
sections show, a GDOR, if it exists, is any vector in the relative interior of the 
normal cone at the observed statistic.  So, the search direction chosen in this manner 
may in fact be a GDOR of the original model and hence result in large steps when 
meeting the curvature condition.  Alternatively, we also consider using search 
directions resulting from a regression through the previous few parameter values to 
break the zig-zagging pattern.

\begin{figure}[!ht]
\centering
\includegraphics[height=2.6in,width=2.6in]{Figures/zigzag-eta}
\includegraphics[height=2.6in,width=2.6in]{Figures/zagplusnorm-eta}
\caption{Contour plots of the log likelihood when the MLE does not exist.  The surface 
of the log 
likelihood tends to flatten, though technically it is still concave.  This can cause 
the steepest ascent 
algorithm to zigzag, which is inefficient (top).  However, by periodically using 
search directions 
determined by normal vectors  derived from the empirical face or regression through 
previous points, the algorithm can make much larger steps (bottom).}
\label{F:zigzag}
\end{figure}


\section{Step size}
We now turn our attention to the optimal step size $\alpha_k$ when our objective 
function is the log likelihood of an 
exponential family.  Taking the derivative of $\ell( \eta_k + \alpha_k p_k)$ with 
respect to $\alpha_k$ shows that the 
log likelihood is maximized as a function of $\alpha_k$ along the direction $p_k$  
when 
\begin{align*}
	\nabla \ell( \eta_{k+1} )^T p_k = 0.
\end{align*}

By choosing $c$ to be small, say 0.2, we ensure that the step taken is close to 
maximizing the log likelihood along the 
search direction.  This is also apparent in Figure~\ref{F:alpha_region}. 

Making $c$ too small, however, may make it difficult to find an $\alpha_k$ that meets 
the curvature condition \eqref
{E:curvature mod} since this search must be done numerically.  In fact, as the line 
search nears the MLE and $\nabla \ell( \eta_k)$ gets smaller, the rightmost term in \eqref{E:curvature mod} gets 
smaller in magnitude (it equals $c \lVert \nabla \ell(\eta_k) \rVert^2$ if using steepest ascent directions), making a 
numerical search for $\alpha_k$ 
more challenging.  

%Finally, while the choice of 0.2 for $c$ worked well in the problems we explored 
%regardless of search 
%directions used, it follows from our discussion in the previous section that it may 
%make sense to use slightly larger 
%values of $c$ when using steepest ascent directions, thereby reducing the zigzagging 
%phenomenon, but smaller values for 
%$c$ when using conjugate gradient methods.


\section{MCMC approximations} \label{section:MCMC approx}
Our algorithm requires us to be able to calculate $\nabla \ell(\eta)$ using \eqref
{E:nabla ell}.  For many 
applications, we will need to approximate $\E_{\eta}g(Y)$ using MCMC.  That is,
\begin{align}
 	\nabla \ell (\eta) = g(y) - \E_\eta g(Y) \approx g(y) - \frac{1}{m}\sum_{i=1}^m g
(Y_i), \label{E:nabla ell approx}
\end{align}
where $Y_1, \ldots, Y_m$ are MCMC draws from the distribution with parameter $\eta$.  
There are many MCMC algorithms 
such as Metropolis-Hastings or Swensen-Wang (used for the Ising model example in 
Section~\ref{S:Examples:Ising}); see \citep{Brooks} and references cited therein.
We show examples in the next section where $\nabla \ell(\eta)$ can be calculated 
exactly and where it must be 
approximated.

The accuracy of the approximation in \eqref{E:nabla ell approx} increases with Monte 
Carlo sample size $m$. 
When the current estimate is far away from the MLE, we can use smaller $m$ to save 
time and work with a 
fairly noisy approximation of the gradient.  However, when the current estimate 
approaches the MLE, larger $m$ are necessary.

Our algorithm relies on the computed values of $\nabla \ell(\eta)$ in the curvature 
condition \eqref{E:curvature mod}, 
as well as the stop condition for the algorithm, $\lVert \nabla \ell( \eta_k ) \rVert 
< \epsilon$.  Given that we may 
only have approximations of $\nabla \ell(\eta)$, we cannot know for certain if either 
of these conditions are truly 
met.  We can ameliorate this by constructing confidence intervals for each of the 
inequalities.  

For the inequalities in \eqref{E:curvature mod}, we can estimate asymptotic standard 
errors of $\nabla \ell( \eta_k + 
\alpha_k p_k)^T p_k$  and $c \nabla \ell(\eta_k)^T p_k - \nabla \ell( \eta_k + 
\alpha_k p_k)^T p_k$ by appealing to the 
Markov chain Central limit theorem 
\citep{Chan:1994,Jones:2004,Roberts:1997,Roberts:2004}.
The \texttt{initseq} function from the R package \texttt{mcmc} \citep{mcmc:R} can be 
used to estimate asymptotic 
standard errors for univariate functionals of reversible Markov chains: given an MCMC 
sample for a univariate 
quantity, \texttt{initseq}
returns a value (divided by sample size) that is an estimate of the asymptotic 
variance in the Markov chain central 
limit theorem.  Both of the quantities in \eqref{E:curvature mod} are univariate.  In 
the second expression, $c \nabla \ell(\eta_k)^T 
p_k - \nabla \ell( \eta_k + \alpha_k p_k)^T p_k$, the MCMC sample generated for $
\nabla \ell( \eta_k + \alpha_k p_k)^T 
p_k$ is independent of the sample generated for $c \nabla \ell(\eta_k)^T p_k$.  Thus 
\texttt{initseq} can be applied 
to each sample separately and the results summed for an estimated variance.  
We can then be approximately 95\% confident (non-simultaneously) that $\alpha_k$ 
satisfies \eqref{E:curvature 
mod} if
\begin{align*}
	 \nabla \ell( \eta_k + \alpha_k p_k)^T p_k - 1.645 \cdot \text{se}_1 > 0 \\
	 c \nabla \ell(\eta_k)^T p_k - \nabla \ell( \eta_k + \alpha_k p_k)^T p_k - 1.645 
\cdot \text{se}_2 > 0 
\end{align*}
where $\text{se}_1$ and $\text{se}_2$ are the asymptotic standard errors for $\nabla 
\ell( \eta_k + \alpha_k p_k)^T p_k
$  and $c \nabla \ell(\eta_k)^T p_k - \nabla \ell( \eta_k + \alpha_k p_k)^T p_k$, 
respectively, calculated as described.

The delta method can be applied to estimate a standard error for $\lVert \nabla \ell
( \eta_k ) \rVert$. 
%%%%%%%%%%%%% 1/25/11 -Newly added from Algorithm.tex
The multivariate version of the delta method states that for a sequence of r.v. $B_n$ 
such that
\begin{align*}
	\sqrt{n} ( B_n - \beta) \stackrel{\DD}{\longrightarrow} N( 0, \Sigma ),
\end{align*}
then for a function $h(B)$, where $h$ such that $\nabla h$ is defined and non-zero,
\begin{align*}
	\sqrt{n} \left ( h(B_n) - h(\beta) \right ) \stackrel{\DD}{\longrightarrow} N 
\left ( 0, \nabla h( \beta)^T \Sigma\nabla h( \beta)  \right ).
\end{align*}

We set $B_n = \nabla \ell_n( \theta)$ and $\beta =\nabla \ell( \theta)$, where we 
know that $B_n \stackrel{a.s.}{\longrightarrow} B$ by SLLN.  
We also do not know $\Sigma$, the variance of $\nabla \ell( \theta)$, but this we will 
approximate with $
\hat{\Sigma}$, the scaled sample variance-covariance matrix of our MCMC batches of 
the canonical statistic (the \texttt{initseq} function 
requires a univariate vector and so cannot be used here).  That is,
\begin{align*}
	\hat{\Sigma} = \frac{1}{\text{nbatch}}\frac{1}{m-1}\sum_{i=1}^{m} (g(Y_i) - \overline{g
(Y)})( g(Y_i) - 
\overline{g(Y)})^T.
\end{align*}

%%%%%%%%%%%%%%%
Thus the asymptotic variance is calculated by
\begin{align*}
	V \left( \lVert \nabla \ell( \eta_k ) \rVert \right )= \frac{1}{\lVert \nabla \ell
( \eta_k ) \rVert^2} \nabla \ell
( \eta_k )^T \, \hat{\Sigma} \,  \nabla \ell( \eta_k ),
\end{align*}% added hat to Sigma
%where $\Sigma$ is the variance matrix of $\nabla \ell( \eta_k )$ and can be estimated 
%by the sample variance matrix of 
%the batch mean vectors of $g(Y_1), \ldots, g(Y_n)$ divided by the number of batches 
%(the \texttt{initseq} function 
%requires a univariate vector and so cannot be used here).  
We can be approximately 95\% confident that $\lVert 
\nabla \ell( \eta_k ) \rVert > \epsilon$ if 
\begin{align*}
	\lVert \nabla \ell( \eta_k ) \rVert - 1.645 \sqrt{ V \left( \lVert \nabla \ell
( \eta_k ) \rVert \right )} > 
\epsilon.
\end{align*}
%To be conservative, we suggest using a larger quantile, say 2, 
%in the confidence intervals instead of the 1.645 used above.

In practice, however, use of confidence intervals does not appear necessary with  
Monte Carlo sample sizes that are set 
large enough so that these standard errors are initially small relative to the point 
estimates.  The ratio of point 
estimate to standard error of course decreases as the algorithm progresses and the 
estimate of the parameter nears the 
MLE, reflected in $\nabla \ell( \eta_k )$ nearing 0.  Thus these confidence intervals 
are most useful as a guide for
when to increase the MCMC sample size, or when to switch methods, or when to terminate 
the algorithm.



\section{Combining with other algorithms}
We believe the best use of this algorithm is in combination with other faster methods 
like MCMC-MLE \citep{Geyer:1992}
or Newton-Raphson safeguarded by our line search algorithm.  Our algorithm with 
steepest ascent or conjugate gradient search direction
should be used initially from ``long range'', when one has no good intuition for an 
initial value.
It is well known that when the objective function is quadratic the conjugate gradient 
method with exact arithmetic converges to the solution
in at most $d$ steps, where $d$ is the dimension of the problem \citep{NW}.  As a rule 
of thumb, we think using our 
algorithm for $d$ steps before switching seems reasonable.
