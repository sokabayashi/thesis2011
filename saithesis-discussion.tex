The goal of this dissertation is to present a generally applicable approaching
to finding MLEs to discrete exponential families.

We have presented a simple line search algorithm for finding the MLE of a regular 
exponential family when the MLE 
exists.  The algorithm avoids the trial and error experimentation of tuning parameters 
and starting points commonly associated with optimization routines
not invented by optimization specialists.  Our algorithm is modeled after algorithms 
discussed in optimization textbooks \citep{Fletcher,NW,Sun:2006},
all of which are safeguarded to ensure rapid automatic convergence.
%Because it only relies on first order derivatives, this approach avoids problems with 
%near-singular Fisher information 
%matrices that plague methods like Newton-Raphson.  The reliant on a curvature 
%condition for step size makes it less 
%sensitive to poor initial values that are problematic for MCMC-MLE and  SA in 
practice.

Convergence is guaranteed when the gradient can be calculated exactly.  Even when the 
gradient cannot be calculated 
exactly and is only estimable via MCMC, the algorithm is still useful in practice, as 
demonstrated by the Ising model 
example.  We have also described a way to construct and use confidence intervals to 
make convergence highly probable.


\section{Areas for Improvement}
\subsection{Step size search}
The algorithm can be computationally demanding.  When the current iteration approaches 
the solution, the 
curvature condition for step size becomes more difficult to satisfy and the method may 
require several iterations of 
MCMC sampling and perhaps an increase in MCMC sample size.  Eventual increase in MCMC 
sample size is unavoidable,
because the achievable accuracy is inversely proportional to the square root of the 
MCMC sample size, as in all Monte Carlo.
Thus we believe the best use of this algorithm is in combination with other faster 
methods like MCMC-MLE \citep{Geyer:1992}
or Newton-Raphson safeguarded by our line search algorithm.  Our 
algorithm should be used from ``long range'', when one has no good intuition for an 
initial value and is concerned about 
picking one that is far from the MLE.  The switch between types of search direction 
(steepest ascent, conjugate gradient,
or Newton) within our algorithm or the switch to another algorithm (such as MCMC-MLE 
\citep{Geyer:1992})
need not require manual intervention.  When used in combination in this
manner, we do not think the confidence intervals are necessary as the curvature 
condition is quite easily satisfied 
when the current iteration is far from the MLE.

One way to improve performance is to use conjugate gradient search directions rather 
than steepest ascent.  In our 
examples, this reduced the number of iterations by over 20\%.  However, in other 
problems we tried with different 
dimensionality, this performance varied significantly and it appears that no guarantee 
can be made about quantity of 
improvement in performance, though in all cases we examined, it never did worse.  This 
is no surprise, because the
necessity of ``preconditioning'' for good performance of the conjugate gradient 
algorithm is well known (but 
we know of no literature about ``preconditioners'' for 
maximum likelihood in exponential families).

\subsection{Convergence}
There are several outstanding issues.  Most notably, we have not showed convergence of 
the algorithm when the gradient 
is approximated via MCMC.  This is a more difficult theoretical problem and is the 
motivation for stochastic 
approximation research.  
Further work is necessary to determine if one can adapt our restrictive curvature 
condition \eqref{E:Wolfe-ll} to the 
approach of \citet{Andrieu:2005} or \citet{Liang:2010} in MCMC stochastic 
approximation.  

\subsection{Switch criteria}
Another remaining issue is the stopping criteria: what value should be chosen for $
\epsilon$ in the exit condition
$\lVert  \nabla \ell( \eta_k ) \rVert < \epsilon$?  Because the value of $\lVert  
\nabla \ell( \eta_k ) \rVert$ can only 
be approximated via MCMC, one cannot be certain if this condition is actually 
satisfied.  Here again, the switch to 
another methodology may be appropriate, though at least in our Ising model example, 
our use of 10,000 for the MCMC 
sample size and 0.005 for $\epsilon$ were successful in obtaining a reasonable 
parameter estimate. 

\subsection{Monte Carlo standard errors}
 A final remaining issue is estimation of Monte Carlo error of the estimates.  Here 
too we recommend switching to another
algorithm at the end.  The MCMC-MLE procedure gives accurate error estimates 
\citep{Geyer:1994}.
For very small steps these are essentially the same as the Monte Carlo error of a 
single unsafeguarded Newton-Raphson step,
so the method in \citep{Geyer:1994} can be used for either.

\section{non-existent MLE}
ALSO.  simulated tempering for faster convergence.

MLE.  What does it mean to ``find the MLE"?
