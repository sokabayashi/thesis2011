\documentclass[12pt]{article}

\usepackage{amsbsy,amsmath,amsthm,amssymb,verbatim,color}

\usepackage{geometry,graphicx,subfigure}
\graphicspath{{../Figures/}}

\geometry{hmargin=1in,vmargin={1in,1in},footskip=.5in}
\usepackage{natbib}
\usepackage{setspace}
%\usepackage{hyperref}
%\usepackage{fancyhdr,lastpage}
%\pagestyle{fancy}
%\fancyhead{}
%\rhead{Saisuke Okabayashi}

%%%%%%%%%%%%%%%%%%%%%%%%%%%%%%%%%%%%%%%%%%%%%%%%%%%%%%%%
\begin{document}

\begin{center}
{\normalsize{\textbf{Parameter Estimation in Social Network Models---one page summary}} }

\vspace{0.15in}
{Saisuke Okabayashi} \\
%\href{http://www.stat.umn.edu/~sai}{www.stat.umn.edu/\textasciitilde sai}\\
www.stat.umn.edu/\textasciitilde sai\\
%\vspace{0.05in}
{Department of Statistics}\\
\end{center}

\setstretch{2}
%a section on the significance of the research 
The friendship between individuals, strategic alliances
between corporations, and trade between countries
are all phenomena that involve complex dependence, where the 
decision for one entity to relate with another depends not only on the attributes of 
the target, but also on the presence of other relations.  
%These other relations in turn
% dependent on the presence of yet other relations, including the one we originally considered.  
 Traditional statistical methodologies like linear regression 
ignore any influence these relations may 
have on one another, relying only on the attributes of autonomous individual to explain outcomes.
Such analysis may lead to wrong conclusions.  For example, 
one may conclude that ethnicity is an important determinant in friendship formation when
in fact there is a strong propensity of individuals to form relationships with their friends' friends,
regardless of ethnicity.

Statistical network models are a powerful tool that have been developed 
precisely to analyze such dependent phenomena.
These models make it possible for a researcher to then identify the 
factors, including those that capture dependency, that contribute to the overall structure 
of outcomes.  
However, such statistical inference can only be performed with a network model that is ``best" fit 
to a dataset, and the methods for fitting these models are still problematic.

In my dissertation, we propose a new algorithm to fit network models to datasets.
Our approach  is designed to work particularly well in settings that occur frequently but are 
problematic for existing methods.  As such, we believe our approach fills a gap in the
existing toolbox of methodologies.

%We have motivated this discussion with a friendship network example.  However, networks are
%a more general concept and can be thought of as a conduit for flow.  That flow
%may be advice, needle-sharing, or diseases between individuals, association between
%terrorists, the negotiating between politicians,
%transactions between companies,
%airplanes between cities,
%trade between nations,
%or binding between proteins.
%A statistical network model is particularly useful for explaining the mechanism of this flow
%when there are complex interdependencies present.


\newpage
\bibliographystyle{/Users/saipuck/Tako/THESIS/ims}
\bibliography{/Users/saipuck/Tako/THESIS/References}




\end{document}
%:
