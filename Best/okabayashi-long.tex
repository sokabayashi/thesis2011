\documentclass[12pt]{article}

\usepackage{amsbsy,amsmath,amsthm,amssymb,verbatim,color}

\usepackage{geometry,graphicx,subfigure}
\graphicspath{{../Figures/}}

\geometry{hmargin=1in,vmargin={1in,1in},footskip=.5in}
\usepackage{natbib}
\usepackage{setspace}
%\usepackage{hyperref}
\usepackage{fancyhdr,lastpage}
\pagestyle{fancy}
\fancyhead{}
\rhead{Saisuke Okabayashi}

%%%%%%%%%%%%%%%%%%%%%%%%%%%%%%%%%%%%%%%%%%%%%%%%%%%%%%%%
\begin{document}

\begin{center}
{\normalsize{Parameter Estimation in Social Network Models}} 

\vspace{0.15in}
{Saisuke Okabayashi} \\
%\href{http://www.stat.umn.edu/~sai}{www.stat.umn.edu/\textasciitilde sai}\\
www.stat.umn.edu/\textasciitilde sai\\
%\vspace{0.05in}
{Department of Statistics}\\
\end{center}

\setstretch{2}

\section{Background}
Is it possible to build a mathematical model that captures the behavioral tendencies of individuals in how they form relationships?

\begin{figure}[h!]
\centering
\includegraphics[width=4.8in]{fmh-gradesex2}
\caption{A friendship network of 1,461 students in grade 7--12 derived 
from the National Longitudinal Study of Adolescent Health \citep{Resnick:1997}.  
Individuals, represented by nodes, are colored by grade, and boys
are depicted by squares, girls by circles.  A line is only present between a pair of 
nodes if a friendship exists.  The data is available in the 
\texttt{statnet} package \citep{statnet:R} on the R platform \citep{R}.}
\label{F:fmh}
\end{figure}

Studies of social networks such as the high school friendship network depicted 
in Figure~\ref{F:fmh} typically look to identify gender, age, ethnicity, and 
other individual-specific attributes to explain which relationships form.  
However, relationships form in an interdependent manner---for example, Alicia 
and Christina are more likely to become friends if separately Alicia and Christina 
are each friends with Brad. Indeed, sociologists have long observed that people 
exhibit a strong tendency to form such triangular structures in friendships.
This complicates any analysis that attempts to understand the mechanism of 
relationship formation.   
In particular, is gender still an important factor in determining whether or not 
Alicia and Christina become friends if they also share a common friend in Brad? 
The issue necessitates a more sophisticated statistical methodology that 
does not treat potential relationships as independent units and instead 
takes into consideration relationship structures---such as triangles---as 
important factors in the analysis.

A probabilistic network model can be constructed for exactly this setting
to clarify which of the underlying forces are important in shaping the 
overall structure of the network.  
When appropriately fit to the observed data, such a model can be used to 
simulate new random networks which retain essential characteristics of 
the original network.  This allows a researcher to 
test hypotheses about the process of relationship formation and identify the
significant contributing factors.  In the example with Alicia and Christina, it 
may be possible that both gender and the tendency to form triangular 
friendships are significant factors.

Our research is in methodologies of fitting network models to data.  
In fact, specifying a network model with factors of interest---both 
individual-specific attributes as well as network structures---has become 
relatively straightforward to do \citep{Wasserman:1996}; it is fitting
the network model to data that is problematic.  This entails calibrating 
the \emph{parameters}, the values which dictate how the model's predictions 
are affected by changes in the selected factors.  The science of finding 
the parameter values that ``best fit" network models to 
data is a research problem that is still in its infancy.  
We focus on two particular areas in the current approaches that we look 
to improve upon:
\begin{enumerate}
\item Short range.  Most methodologies rely on iterated updates of the model, 
where parameters are  adjusted incrementally and the improved model evaluated 
for fit.  When the initial guesses for the parameters are far away from the 
``best fit" values (which are not known in advance), however, current 
methodologies perform poorly and do not arrive at satisfactory values.
\item Non-existent ``best fit" values.  For particular models and data, it is 
in fact possible that the ``best fit" parameter values do not exist, a situation 
that is known as \emph{model degeneracy} \citep{Handcock:degeneracy,Rinaldo:2009}.  
This been cause for considerable concern in the network literature 
\citep{advancesp*,recentp*,statnet-tutorial}.  In such a case, current 
methodologies may return nonsensical values, yielding a model that generates 
random networks that show no resemblance to the original data and causing 
confusing for the researcher.
\end{enumerate}
In my dissertation, I propose a new algorithm that addresses both of these 
issues: our algorithm is designed with the goal of working best when the initial 
guess for parameter values is far from the ``best fit" values.  In addition, its 
update mechanism utilizes computer simulations  in such a way as 
to detect the conditions that lead to non-existent ``best fit" models.

%%%%%%%%%%%%%%%%%%%%%%%%%%%%%%%%%%%%%%%%%%%
\section{Network models}
A probability model assigns probabilities to outcomes of an event.  For example, 
the probability model for flipping a coin that is possibly unfair assigns some 
probability to getting a head, and one minus that probability to getting a tail.  
In the case of network models, this concept is simply 
extended to assign probabilities to each of the different networks that could 
occur for the number of individuals of interest.
The most commonly used probability model for networks is the 
\textit{exponential random graph model} (ERGM) 
\citep{Wasserman:1996,Pattison:1999,logit,Snijders:2002,introp*,ergm} and 
is the focus of our research.  It has a simple and flexible form:
to specify a model, a researcher need only handpick \textit{network statistics}, 
or relationship structures 
like the triangle described earlier, to accompany the other actor-specific 
factors like gender or age.  

%This is analogous to picking the predictors for a linear regression model, with the difference being that although we may think of a relationship structure such as a triangle as a predictor in our model, it is in fact a function of the very response variable that we are trying to model.
In the example of the high school friendship network depicted in 
Figure~\ref{F:fmh} with 1,461 students in grades 7--12, a researcher may 
be interested in whether being in the same grade, of the same gender,
and of the same ethnicity are important factors in friendship formation.  
However, because of the inherent dependency in friendship formation, the 
researcher should also include network structures like triangles
into the analysis.  The inclusion of these network structures in the 
model ideally prevents a researcher from concluding
that a factor, say gender, is important in friendship formation when in 
fact it was due to the interdependence of individuals behaviors.  In order 
to carry out such an analysis, however, the model needs to be fit the 
the observed data set.

\subsection{Parameter estimation}
Our first area of research is to develop a new algorithm for fitting a 
social network model to data by determining appropriate values for the 
parameters.  The \emph{likelihood function} is the function that assigns 
probability to the observed network for a specified parameter value.
The goal is to find the \textit{maximum likelihood estimates} (MLE), the 
parameter values that maximize the likelihood function, that is, they 
maximize the probability of the observed data.
The model outfitted with these parameter values can then be used to simulate 
new random networks that resemble the original data and make it possible 
to test hypotheses about friendship formation.  

%The MLEs are what we earlier referred to as the ``best fit" model parameters.

\subsubsection{Our algorithm}
%We pursue a simple iterated approach for finding the MLEs for the model parameters: 
An iterated algorithm for finding the MLE for the model parameters starts 
from any initial guess.  The algorithm then take a step in a direction that 
yields a parameter value that assigns higher probability to the observed data.    
When no higher probability can be found, the MLE has been attained.  
So far, this is no different than any other optimization algorithm for finding 
a maximum of a function.
The particular complication here, however, is that actually calculating a 
probability using an ERGM---and hence evaluating the likelihood 
function---is computationally infeasible even with advances in computing: 
it involves a summation over all the different possible networks which could occur, 
an astronomical number when there are even 40 individuals in the network 
(there are 6.36$\times10^{234}$ networks).  
This inability to calculate probabilities with a probability model may seem 
like an insurmountable drawback,
but in fact the analysis of interest---the hypothesis testing of factors---relies 
only on being able to simulated networks from the ``best fit" model and not 
actually calculate probabilities themselves.  Thus this is 
only an issue in finding the MLEs for model parameters and not in the 
subsequent inference.

In our implementation, we avoid this obstacle by using only the slope of 
the likelihood function, which can in fact be well-approximated by 
a computation simulation technique called \emph{Markov chain Monte Carlo} (MCMC).  
By making sure that
\begin{enumerate}
\item the direction of steps are always uphill on the likelihood function
\item the steps taken in that direction are sufficiently large,
\end{enumerate}
our algorithm will converge to the MLE in practice.  The first point can be met 
simply by using directions that have a positive slope.  The second point is the 
focus of the first part of my research.
Figure~\ref{F:curvature} illustrates the likelihood function values 
for different choices of step sizes along a direction that has already 
met the first condition.  

\begin{figure}[!h]
\centering
\includegraphics[width=5.2in]{curvature-layman}
\caption{The likelihood function for different choices of step sizes along 
a specified direction.  Our curvature condition forces a step size in the 
region marked, with end points corresponding to particular slope values.  
The tangent lines corresponding to these slopes are the dotted lines.
The smoothness of this function and absence of more than one maximum is guaranteed 
by theoretical properties of the model.}
\label{F:curvature}
\end{figure}

We devised a \emph{curvature condition} which, using only the slope, ensures 
that the steps taken are large enough to make significant progress up the 
likelihood function.  The condition operates on just 
a cross section of the likelihood function in a specified direction, and 
thus even if the function is maximized in this specified direction, the attained 
value is typically not the global maximizer.  Hence these updates will need to be 
repeated several times.

The simplicity of this approach in only using the slope allows it avoid 
many of the issues related to poor initial guess that plague other 
more complicated methodologies.  For this reason, we have called our 
algorithm a ``long range" algorithm.
The tradeoff is that our methodology is less efficiently computationally
since it may require many MCMC simulations (in friendship example, about 150), 
which may take a few hours, and is particularly slower when the maximum 
is close.  However, this is exactly the area
where existing methodologies excel, and we advocate combining methodologies, 
so that our algorithm is used at the outset and it nears to maximum, it 
switches to a faster approach that can attain the MLE in one or two iterations.

\subsection{Model degeneracy}
The setting described so far assumed the ``best fit" model parameters---the 
maximizers of the likelihood function---exist.
But it is in fact possible that the likelihood function increases in perpetuity;
that is, unlike the case depicted in Figure~\ref{F:curvature} where 
the likelihood function bends back downward, it can instead continue 
to increase to infinity.  This situation, known as model degeneracy, 
causes significant problems for any parameter estimation method since 
no maximizer of the likelihood function exists---the MLE
is ``at infinity".  Model degeneracy occurs frequently enough to be an impediment to 
researchers using ERGMs.  As a countermeasure, network researchers have been developed
 new network statistics such as more sophisticated forms of triangles that are 
less prone to degeneracy though they do not make it impossible 
\citep{Handcock:degeneracy,advancesp*,recentp*,statnet-tutorial}.
The second part of my research concerns adapting our algorithm to detect 
when degeneracy occurs and how to find a new model that can still 
be used for hypothesis testing when it does.

The theoretical underpinnings for why these instances occur has been 
understood for many years \citep{Barndorff},
but these conditions were difficult to check in practice until recently.  
\citet{Geyer:gdor} showed a to check for this condition using 
advances in computing \citep{Fukuda:2008} in the case
of regression models, and demonstrated that hypothesis testing can still be 
done using a new model, called a \emph{limiting conditional model} (LCM), 
that is a relative of the original.
The second part of my research has been to apply similar methodology to 
the setting of ERGMs.

Model degeneracy is closely related to the value ranges of the factors 
included in the model.  In the case of ERGMs, the flexibility of 
choice over the network statistics makes it particularly difficult
to detect.  However, we showed that the very MCMC simulations used to estimate
the slope as described earlier can in fact be used to 
\begin{enumerate}
\item check for model degeneracy,
\item if the model is degenerate, define the framework for the LCM.
\end{enumerate}
So, our algorithm can be applied in a setting where a maximizer may not exist.  
In the very process of seeking the maximizer, which requires MCMC simulations 
to estimate the slope, those same simulation
results can also be used to check if the model may be degenerate.  If the 
model is degenerate, these simulation results can then be used to characterize the LCM.

We applied this to the case of a small 9-individual network model, 
where the setting is still simple enough to confirm the degeneracy conditions.


\section{CONCLUSIONS}

%\begin{figure}[!h]
%\centering
%\includegraphics[width=5in]{g9-hull-bw}
%\caption{Convex support?}
%\label{F:g9-hull}
%\end{figure}
\section{Next steps}
\texttt{statnet} \citet{statnet:R}
\section{Extension of our research}
While the motivation for our research is rooted in \emph{social} network models, 
network models are in fact more general and can be applied to any setting when there is 
interdependent flow between entities.  This flow may be
diseases between individuals, the connected of computers, airlines between cities, 
or the binding between proteins.  

Furthermore, the methodology we have developed to find ``best fit" parameter 
values for ERGMs and detect when they do not exist can be applied to a 
broader class of finite \emph{exponential family models}.  These
can be applied to more general settings of dependency, such as the geographic 
variation in crop yields \citep{Besag:1974,Besag:1975},
DNA fingerprint data \citep{Geyer:1992}, or the spin of neighboring 
atoms in a ferromagnetic model \citep{Ising,Potts}.


\newpage
\bibliographystyle{apalike}
\bibliography{/Users/saipuck/Tako/THESIS/References}

%\begin{thebibliography}{77}

%
%\bibitem{kend}
%Kendall, W. S. 2004.
%\newblock Geometric ergodicity and perfect simulation,
%\newblock {\sl Electronic Communications in Probability}, 9:140--151.

%

%\bibitem{jones}
%Jones, G. L. (2004)
%\newblock On the {M}arkov chain central limit theorem,
%\newblock {\sl Probability Surveys}, 1:299--320.

%
%\bibitem{jone:hobe:2004}
%Jones, G. L. and Hobert, J. P. (2004).
%\newblock Sufficient burn-in for {G}ibbs samplers for a hierarchical random effects model, 
%\newblock {\sl The Annals of Statistics}, 32:784--817.

%\bibitem{prop:wils:1996} 
%Propp, J. G. and Wilson, D. B. (1996).
%\newblock Exact sampling with coupled {M}arkov chains and applications to statistical mechanics,
%\newblock {\sl Random Structures and Algorithms}, 9:223--252.
% 
%\end{thebibliography}




\end{document}
%:
