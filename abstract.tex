A social network is an example of a phenomenon with complex stochastic dependence 
that is commonly modeled with a class of exponential families called exponential random graph models (ERGM).
Maximum likelihood estimators (MLE) for such  
exponential families
can be difficult to estimate when the likelihood is difficult to compute.
Most methodologies rely on iterated estimates and are sensitive
to the starting value, often unable to converge if started too far from the
solution.  In addition, these methods may require significant trial-and-error to 
work effectively.  
%Markov chain Monte Carlo (MCMC) methods based on the MCMC-MLE algorithm in \citep
%{Geyer:1992} are guaranteed to converge in theory under certain conditions when 
%starting from any value, but in practice such an algorithm may labor to converge when 
%given a poor starting value.  
Even more problematic is that the MLE may not exist, 
a situation that occurs with positive probability for ERGMs.
% for discrete state space exponential families like ERGMs.
In such a case, the MLE 
is actually ``at infinity" in some direction of the parameter space.  
\citet{Geyer:gdor} illustrated an algorithm to detect non-existent MLEs 
and construct one-sided confidence intervals for how close
the parameter is to infinity in the case of generalized linear 
models, where the convex support is known.  

Here we present a simple line search algorithm to find the MLE of a regular exponential 
family when the MLE exists and is unique.  
The algorithm can be started from any 
initial value and avoids trial-and-error experimentation.  
We show convergence of the algorithm for the case where the gradient can be 
calculated exactly.  When it cannot, it has a particularly convenient form that is 
easily estimable with Markov chain Monte Carlo (MCMC), making the algorithm still useful to a practitioner.  
Finally, when the MLE does not exist, our algorithm adapts the 
approach of \citet{Geyer:gdor} to a setting where the convex support is not
known in advance, which is typically the case for network models.  
We are able to then calculate one-sided confidence intervals
for the parameters.