\label{A:Triangle count}
As noted in Section~\ref{S:ERGM setup}, even for an undirected 9-node network, 
there are $2^{{9\choose 2}}$, or about 69 billion different possible graphs.
Counting the network statistics of edges, two-stars, and triangles for this
network is not a trivial calculation and can take an enormous amount of time 
if not coded efficiently (our first straightforward implementation 
entirely in R would have taken over a year).  Thus every effort must 
be made to represent the data as efficiently as possible, 
implementing loops in C and avoiding calculations.

A few quantities are easily obtained: the maximum number of edge is 
${n \choose 2} = {9 \choose 2} = 36$.  The maximum number of triangles is 
${n \choose 3} = {9 \choose 3} = 84$, since a triangle requires 3 of the 9 actors.  
For each triangle, there are 3  different two-stars, so the maximum number 
of two-stars is $3 \cdot 84 = 252$.

%Now, R (and all other programming platforms at the moment) cannot handle $2^{36}$ as an integer; in fact, casting $2^{31}$ as an integer produces an error.  Thus we do not want to write a \texttt{for} loop going from 1 to $2^{36}$.  So what to do?  

The approach we have implemented requires first creating a collection of graphs,
one graph for each possible network structure of interest (triangle or two star),
where each graph is empty except for that particular network structure.  For example,
one graph in this collection is the one representing a triangle between the 1st, 2nd, and 3rd nodes.  In matrix
form, this network has all zeros except for ones in entires of (1,2), (1,3), (2,3), 
(2,1), (3,1), (3,2).
We then compare each of the 69 billion possible graphs to this collection.
Every time a comparison yields a match, that particular configuration is present
and the count for that statistic is incremented by one.  
The reason this is efficient is because we can actually represent the 
a graph as a single binary number and do bitwise comparisons for far greater speed.

We can explain this in further detail with triangles.
We know there are ${9 \choose 3} = 84$ possible triangles, and so we create
84 graphs, each with only one triangle present among 3 actors.  
So, the first element corresponds to a graph with ties present between 
actors 1, 2, and 3, and no other ties present, the second element corresponds 
to a graph with a ties present between actors 1, 2, and 4.  
Because the network is undirected, the adjacency matrix is symmetric and 
can be fully described by just its upper triangle, which in our convention 
of going down vertically and then across is $(1,1,1,0,0,0, \ldots,0)$.  
We can treat this upper triangular vector as a 36-digit number, 
111000000000000000000000000000000000.  If we treat this as a number 
in base 2, we can convert it to base 10 to a number less than $2^{36}$.  
For this first matrix, it is 60,129,542,144.  Proceeding in this manner up 
through 84, we have a set of 84 numbers corresponding to all the possible 
triangle configurations in the network.  

We now turn our attention to iterating through the $2^{36} \approx 69$ billion possible graphs.  
Using the \texttt{long int} representation in C (on a 64-bit system), 
we can in fact iterate from 0 to $2^{36}-1$.  By having the index itself 
correspond to a specific graph,  we can treat its binary form as 
the collapsed upper triangular vector.
For example, the last index corresponding to $2^{36}-1$ has binary form of 
 111111111111111111111111111111111111 which shows every tie present and
 is the complete graph.
   
For each graph index, we can loop through our set of 84 triangle numbers
and perform bitwise logic operations (the \texttt{\&} operator in C) to 
compare the binary form of the graph index to each of the triangle numbers.  
In binary form, if there are ones in all the digits that the triangle number has a 
one in, 
then the graph has this particular triangle present.
For example, the operation $(2^{36}-1)$ \& $60,129,542,144$ would return 
111000000000000000000000000000000000 in binary form, indicating that this 
particular triangle is present.  This is confirmed by comparing the
result of the binary operation back to the original triangle number,
which would return TRUE.  Proceeding through all the other triangle numbers, 
we get a count of how many triangles are present for this graph index.

A similar method is employed for counting two-stars, where we would similarly
first calculate the $3\times84 = 252$ two-stars graph numbers before iterating
through the graph indices.

To count edges, there are well-known tricks to count the number of ones 
in a binary number in C which cleverly use the `\texttt{>>}' shift operator.
By taking the graph index \texttt{>> 1}, it moves all the digits to the 
right by dropping off the rightmost digit and adding a 0 to the 
farthest left digit.  So, 
\begin{align*}
111111111111111111111111111111111111 >> 1
\end{align*}
returns 01111111111111111111111111111111111 (where there are now 35 ones instead of 36).  
In this manner, the shifts can be continued and ones counted 
(done by another bit comparison to '01' using the `\texttt{\&}' again) 
until there are no more ones.  This approach avoids any arithmetic and is thus much faster.  
It should be noted that a computer always stores the \texttt{long int} in binary form 
and so the 36-digit representation used here is entirely for our benefit only.

Finally, we can further speed up the calculations by parallelizing this computation.  
We can count the number of edges, two-stars, and triangles in the 
first $2^{36}/8$ graphs at the same time that we count them in the last $2^{36}/8$ graphs.  
To do this, we simply use the \texttt{mclapply} function in the \texttt{multicore} \citep{multicore:R} 
library in R.
At the time of this writing, we performed this calculation on an 8-CPU 2.9GHz linux box
which took 2 hour 15 minutes to complete.  




