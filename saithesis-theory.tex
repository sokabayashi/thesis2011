All of the parameter estimation issues discussed thus far for ERGMs are relevant to 
the larger class of discrete state space exponential family models.
These models are commonly used to model phenomena with dependent structure, 
where the outcomes of the response variable of interest are in fact dependent on one 
another.  For example, the Ising 
model \citep{Ising,Potts} is an exponential family model that has been used to model 
ferromagnetism, where neighboring 
pixels (representing atoms in a crystal lattice) are more likely to have the same 
color.  
%A realized 
%sample from this model is depicted in Figure~\ref{F:pottsimage},   
%We explore this model further in Section~\ref{S:Examples:Ising}.
Other examples of phenomena with dependent structure modeled with exponential 
families include
plant ecology \citep{Besag:1974,Besag:1975}, DNA fingerprint data \citep{Geyer:1992}
 and the lifetime fitness of plants \citep
{Shaw:2008}.

Although motivated by ERGMs, the methods we propose are rooted in fundamental
exponential family theory and applicable to any finite state space 
exponential family model.  Thus the theory presented in this section is from
the perspective of this more general setting.

\section{Exponential Family Theory}
Much of the fundamental background for exponential families has already been
covered in Section~\ref{S:ERGM setup}; our focus is on regular exponential
families on a finite sample space $\YY$  with log likelihood \eqref{E:loglike}.
As noted earlier, when the sample space $\YY$ is even moderately large,
the cumulant function $c(\eta)$ involves a summation that may be prohibitively 
expensive to evaluate.  For example, the sample space $\YY$ for an Ising model 
defined on a $32\times 32$ square lattice where each entry takes values of 0 or 1 
has $2^{1024} \approx 10^{300}$ elements.  
A loop with this many iterations takes too long no matter how programmed.

A useful property of all exponential families \cite[p.~27]{TPE2} on which we rely 
heavily is that 
\begin{align*}
	\E_\eta(g(Y)) &= \nabla c(\eta)	\\
	\Var_\eta(g(Y)) &= \nabla^2 c( \eta ).
\end{align*}

Thus we can express first and second derivatives of the log likelihood \eqref
{E:loglike} and Fisher information $I(\eta)$ as
\begin{align}
	\nabla \ell( \eta ) &= g(y) - \E_\eta g(Y) \label{E:nabla ell} \\
	\nabla^2 \ell( \eta ) &=  - \Var_\eta g(Y) \label{E:nabla2 ell} \\
	\I(\eta) &= -\E_\eta \nabla^2 \ell (\eta ) = \Var_\eta g(Y) \label{E:FI}
\end{align}
and thereby avoid evaluation of the problematic cumulant function $c$.


By the strict convexity of the log likelihood function ensured by \eqref{E:nabla2 
ell}, the global 
maximum, if it is exists, is attained when $\eta$ is such that $\nabla \ell( \eta ) = 
0$, or, using \eqref{E:nabla ell}, by setting ``expected equal to observed,"
\begin{align}
	\E_\eta g(Y) = g(\yobs). \label{E:Observed-Expected}
\end{align}

%%%%%%%%%%%%%%%%%%%%%%%%%%%%%%%%%%%%%%%%%%%%%%%%%%%%%


\subsection{Mean value parameterization}

An alternative parameterization of an exponential family is the mean value 
parameterization.  For each natural parameter $\eta$, we can define the mean parameter 
$\mu$ such that
\begin{align*}
	\mu = \E_\eta g(Y).
\end{align*}
The space for the mean value parameterization is the same as the convex support of the 
natural statistics and thus lends itself to easier interpretability  \citep
{Handcock:degeneracy, Rinaldo:2009}.  In particular, $\mu = g(\yobs)$ is the MLE in 
the mean value parameterization.  \citeauthor{Handcock:degeneracy} observed that mean 
value parameters located too close to the boundary of the convex support correspond to 
degenerate distributions.  

\textbf{[NONONO]}. \hl{ There exists a one-to-one mapping between a natural parameter and its mean value 
parameter, and one can calculate mean value parameters from natural parameters. In 
general there is no simply way to get the natural parameter value from the mean value 
parameter (if there was, then finding MLEs would be very easy and we wouldn't need all 
these algorithms!).}

%%%%%%%%%%%%%%%%%%%%%%%%%%%%%%%%%%%%%%%%%%%%%%%%%%%%%


\section{Convex Analysis}
The issue of MLE existence in the conventional sense in an exponential family is 
closely tied to the geometric properties of 
the convex support of the model \citep{Barndorff, Geyer:gdor, Rinaldo:2009}.  We 
describe the relevant theory from convex analysis as it pertains to the case of 
exponential families.

A \emph{convex polytope} $C$ is the \emph{convex hull} of a finite set of points $V$, denoted $\con( V )$.
By the Minkowski-Weyl theorem \citep[Theorem 19.1]{Rockafellar:1970}, this convex 
set can equivalently be represented as the intersection of a finite collection of 
closed half-spaces.  These two representations of a convex polytope are referred to 
as the \emph{V-representation} and \emph{H-representation}, respectively.  
%The V-representation of a convex polytope $C$ is the set of all linear combinations
%\begin{align*}
%	\sum_{i \in E \cup I} b_i \alpha_i
%\end{align*}
%where $\alpha_i$ are vectors, $b_i$ are scalars, $E$ and $I$ are disjoint finite sets 
%such that
%\begin{align*}
%	b_i \geq 0, \quad i \in E \cup I
%\end{align*}
%and if $I$ is nonempty
%\begin{align*}
%	\sum_{i \in I} b_i = 1.
%\end{align*}

The H-representation can be expressed as the solution set of a finite set of linear 
equations and inequalities,
\begin{align*}
	C = \{x: Ax \leq b \},
\end{align*}
where $A$ is a matrix and $b$ a vector.

The \emph{relative interior} of a convex set $C$, denoted $\rint C$, is the interior 
relative to its affine hull.  
\textbf{[WHAT ELSE? bd()? ]}

A nonempty \emph{face} of a convex polytope $C$ is a convex subset of $C$ such that 
every line segment in $C$ with a relative interior point in $F$ has both end points in 
$F$ \citep{Rockafellar:1970}.  It is itself a convex polytope.
A \emph{proper} face is a face that is not the empty set or $C$, and 
\emph{facets} are proper faces of the highest dimension.

The \emph{tangent cone}\footnote{Typically, the tangent cone is defined 
as the \emph{closure} of the above set, but
this is unnecessary when $C$ is polyhedral.}
 of a convex polyhedral set $C$ at a point $x \in C$ is
\begin{align*}
	T_C(x) = \{s(w-x):w \in C \text{ and } s \geq 0 \}.
\end{align*}


The \emph{normal cone} of a convex set $C$ in $\RR^d$ at a point $x \in C$ is 
\begin{align*}
	N_C(x) = \{ \delta \in \RR^d: \inner{w-x,\delta} \leq 0 \text{ for all } w \in C 
\}.
\end{align*}

Tangent and normal cones are \emph{polars} of each other, that is, each determines the other.  
The normal cone at $x$ can be defined in terms of the tangent cone at $x$ by
\begin{align*}
	N_C(x) 	&= \{ w \in \RR^d: \inner{ w, v } \leq 0 \text{ for all } v \in T_C(x) \}.
\end{align*}

\hl{DEFINE \emph{direction of recession}, \emph{direction of constancy}.}  \textbf{get from 
\citep{Rockafellar:1970} p 69.}



%%%%%%%%%%%%%%%%%%%%%%%%%%%%%%%%%%%%%%%%%%%%%%%%%%%%%
\section{MLE existence in exponential families}
We now present two equivalent approaches to determine the existence of an MLE
in the conventional sense.  

\subsection{Well-known condition?}
The well-known condition for the existence of the MLE \citep{Barndorff, Brown:1986} 
relates the relative location
of the observed statistic to the boundaries of the convex support
and is formally stated as follows:
\begin{theorem}[Corollary 9.6, Section 9.8(viii) in \citep{Barndorff}] \label{Thm:MLE rint}
Under the conditions [CONDITIONS], the MLE exists in the conventional sense and is 
unique if and only if 
$g(\yobs) \in \rint(C)$.
\end{theorem}

While the above theorem is in fact quite elegant and simple, we note that in order to apply this result, the geometry of $C$ must be fully known.

\subsection{Approach of \citet{Geyer:gdor}}
\citet{Geyer:1990,Geyer:gdor} related MLE existence to
 not only the boundary of the convex support, but also to behavior of the log likelihood function along certain directions.

We retrace the blocks leading to this result, and present simpler proofs where possible:
\begin{theorem}[Theorem 2.2 in \citep{Geyer:1990}]\label{Thm:e_c}
\begin{align*}
e^{c(\eta + s \delta) - bs} &\to 
		\begin{cases} 
			0 									& b > \sigma_c(\delta) \\
			e^{c(\eta)} P_\eta(g(Y) \in H ) 		& b = \sigma_c(\delta) \\
			+\infty								& b < \sigma_c(\delta)
		\end{cases}
& \text{as } s \to +\infty.
\end{align*}
where $\delta$ is a non-zero direction, $C$ the convex support, and
\begin{align*}
	\sigma_C (\delta) &= \sup_{g(y) \in C} \inner{ g(y), \delta} \\
	H &= \set{w: \inner{w, \delta} = \sigma_C(\delta) }.
\end{align*}
\end{theorem}
Here, $H$ is the supporting hyperplane to the set $C$ with normal vector $\delta$.

\begin{proof}
\textbf{Case: $b = \sigma_C(\delta)$.}

Starting with \eqref{E:kappa},
\begin{align*}
	e^{c(\eta)} = \kappa(\eta) = \int  e^{\inner{\eta, g(y)}} \, d\mu(y),
\end{align*}
so that
\begin{align*}
	e^{ c(\eta + s \delta ) - bs } &= \int e^{\inner{\eta + s \delta,g(y)} - bs } \, d\mu(y). \\
					&= \int e^{\inner{\eta, g(y)}  + s [ \inner{ g(y), \delta} - b ] } \, d\mu(y). 
\end{align*}
Multiplying by $\frac{f_\eta(y)}{f_\eta(y)}$,
\begin{align*}
	e^{ c(\eta + s \delta ) - bs } &= \int e^{\inner{\eta,g(y)}  + s [ \inner
{g(y),\delta} - b ] }  \frac{ f_\eta(y) }{ e^{\inner{\eta,g(y)} - c(\eta)} }\, d\mu(y) \\
	&= \int e^{  s [ \inner{g(y),\delta} - b ] + c(\eta) }  f_\eta(y) \, d\mu(y) \\
	&= \E_\eta e^{  s [ \inner{g(Y),\delta} - b ] + c(\eta) }.
\end{align*}
%What happens as $s \to +\infty$?  We would like to reverse the order of taking the 
%limit and expectation.  Fortunately, we have the monotone convergence theorem.  
The monotone convergence theorem can be applied to reverse the order of the limit and expectation
for monotone sequences of random variables.  For $\inner{g(Y), \delta} \leq b$, we 
have a monotonically decreasing
 sequence of random variables and for $\inner{g(Y), \delta} > b$, the sequence is increasing.  Thus, 
\begin{align*}
	\lim_{s\to \infty} \E_\eta e^{  s [ \inner{g(Y),\delta} - b ] + c(\eta) } &= E_
\eta \lim_{s\to \infty} e^{  s [ \inner{g(Y),\delta} - b ] + c(\eta) }. 
\end{align*}
Ignoring the expectation and examining the just limit component of the above,
\begin{align*}
	\lim_{s\to \infty} e^{  s [ \inner{g(Y),\delta} - b ] + c(\eta) } &= 
			\begin{cases} 
			0 								& \inner{g(Y),\delta} < b \\
			e^{c(\eta)} 			 			& \inner{g(Y),\delta} = b \\
			+\infty							& \inner{g(Y),\delta} > b.
		\end{cases}
\end{align*}
In this first case that we consider, $b = \sigma_C(\delta) = \sup_{g(y) \in C}
\inner{g(y),\delta}$, so $\inner{g(Y),\delta}$ can never be greater than $b$ and 
thus the $+\infty$ outcome above is not possible.  We can rewrite the result above 
succinctly as
\begin{align*}
	\lim_{s\to \infty} e^{  s [ \inner{g(Y),\delta} - b ] + c(\eta) } 
		&= I(\inner{g(Y), \delta} = b ) e^{c(\eta)} 
		= I( g(Y) \in H ) e^{c(\eta)}.
\end{align*}
Then returning to the original expression of interest,
\begin{align*}
	\lim_{s\to \infty} e^{c(\eta + s\delta) - bs} 
	= \lim_{s\to \infty} \E_\eta e^{  s [ \inner{g(Y),\delta} - b ] + c(\eta) } 
	&= e^{c(\eta)} P( g(Y) \in H ).
\end{align*}

\textbf{Case: $b > \sigma_C(\delta)$}.
\begin{align*}
	\lim_{s\to \infty} e^{ c(\eta + s \delta ) - bs } &= 
	\lim_{s\to \infty} e^{ c(\eta + s \delta ) - \sigma_C(\delta)s + \sigma_C
(\delta)s - bs }  \\
	&= \left( \lim_{s\to \infty} e^{ c(\eta + s \delta ) - \sigma_C(\delta)s} 
\right ) \left(  \lim_{s\to \infty} e^{ s[ \sigma_C(\delta) - b] } \right )\\
	&= \left (e^{c(\eta) }P(g(Y)\in H) \right ) \cdot 0 = 0.
\end{align*}

\textbf{Case: $b < \sigma_C(\delta)$}.
\begin{align*}
	\lim_{s\to \infty} e^{ c(\eta + s \delta ) - bs } &= 
	\lim_{s\to \infty} e^{ c(\eta + s \delta ) - \sigma_C(\delta)s + \sigma_C
(\delta)s - bs }  \\
	&= \left( \lim_{s\to \infty} e^{ c(\eta + s \delta ) - \sigma_C(\delta)s} 
\right ) \left(  \lim_{s\to \infty} e^{ s[ \sigma_C(\delta) - b] } \right )\\
	&= \left (e^{c(\eta) }P(g(Y)\in H) \right ) \cdot \left ( + \infty \right ) 
= + \infty.
\end{align*}
\end{proof}

We now define directions of constancy and recession in terms of Theorems 1
and 2 of \citep{Geyer:gdor} which we state without complete proofs.

\begin{theorem}[Theorem 1 (Direction of Constancy) from \citet{Geyer:gdor}] \label{Thm:DOC}
For a full exponential family with
\begin{itemize}
\item log likelihood function $\ell(\eta)$ as in \eqref{E:loglike},
\item natural parameter space $\Xi$ as in \eqref{E:paramspace},
\item natural statistic $g(Y)$,
\item observed data $\yobs$ such that $g(\yobs) \in C$, the convex 
support,
\end{itemize}
the following are equivalent:
\begin{enumerate}
\item For all $\eta \in \Xi$, the function $s \mapsto \ell( \eta + s\delta)$ is 
constant on $\RR$ \cite[Theorem 1(b)]{Geyer:gdor}.
\item The parameter values $\eta$ and  $\eta + s\delta$ correspond to the same 
probability distribution for all $\eta \in \Xi$ and all real $s$ \cite[Theorem 1(d)]
{Geyer:gdor}.
\item $\inner{g(Y) - g(\yobs),\delta} = 0$ almost surely for all distributions in the 
family \cite[Theorem 1(f)]{Geyer:gdor}.
\item $\delta \in N_C(g(\yobs))$ and $-\delta \in N_C(g(\yobs))$ \cite[Theorem 1(g)]
{Geyer:gdor}.
\item $\inner{w,\delta} = 0$ for all $w \in T_C(g(\yobs))$ \cite[Theorem 1(h)]
{Geyer:gdor}.
\end{enumerate}
\end{theorem}
Any vector $\delta$ that satisfies any of the conditions above is called a 
\emph{direction of constancy} of the log likelihood.
The set of all directions of constancy is called the \emph{constancy space} of 
the log likelihood, which is a vector subspace.

\begin{theorem}[Direction of Recession: Theorem 3 from \cite{Geyer:gdor}] \label{Thm:DOR}
For a full exponential family with the same setting as Theorem~\ref{Thm:DOC},
%\begin{itemize}
%\item log likelihood function, $\ell(\eta)$, described by \eqref{E:loglike},
%\item natural parameter space, $\Xi$,
%\item natural statistic, $g(Y)$,
%\item observed value of the natural statistic, $g(\yobs)$, such that $g(\yobs) \in C$, the convex 
%support,
%\end{itemize}
the following are equivalent:
\begin{enumerate}
\item For all $\eta \in \Xi$, the function $s \mapsto \ell( \eta + s\delta)$ is 
nondecreasing on $\RR$ \cite[Theorem 3(b)]{Geyer:gdor}.
\item $\inner{g(Y) - g(\yobs),\delta} \leq 0$ almost surely for all distributions in 
the family. \cite[Theorem 3(d)]{Geyer:gdor}.
\item $\delta \in N_C(g(\yobs))$ \cite[Theorem 3(e)]{Geyer:gdor}.
\item $\inner{w,\delta} \leq 0$ for all $w \in T_C(g(\yobs))$ \cite[Theorem 3(f)]
{Geyer:gdor}.
\end{enumerate}
\end{theorem}
Any vector $\delta$ that satisfies any of the conditions above is called a 
\emph{direction of recession} of the log likelihood.  Every direction of constancy
is a direction of recession.

\citeauthor{Geyer:gdor} uses Corollary 2.4.1 in \citep{Geyer:1990} to prove the 
equivalence of conditions 1 and 2 of Theorem~\ref{Thm:DOR}.  Here we present
a more accessible proof without the use of this corollary.  It is the equivalence
of these two conditions the relates the behavior of the log likelihood function
to the convex support of the distribution.
%The proof requires the following lemma, which a PROOF WILL BE NEEDED.
%$\ell(\theta+s\delta)$ is non-dereasing if and only if $\lim_{s \to \infty} \ell
%(\theta+s\delta) > -\infty$.  See Theorem 8.6 in \citep{Rockafellar}
\begin{proof}
\textbf{From $2 \to 1$}:
Assume $\inner{g(Y)-g(\yobs), \delta} \leq 0$.  Take expectations of both sides
with respect to the distribution index by the parameter value $\eta + s \delta$:
\begin{align} \label{E:exp_gy}
\inner{ \E_{\eta + s \delta} g(Y) - g(\yobs), \delta} \leq 0.
\end{align}
Now, taking the derivative of $\ell( \eta + s\delta)$ with respect to $s$,
\begin{align*}
\deriv{\ell( \eta + s \delta)}{s} &= \deriv{}{s} 
			\left ( \inner{g(\yobs), \eta+s \delta} - c(\eta+s\delta)  \right )\\
%	&= \inner{g(\yobs), \delta} - \deriv{}{s} c(\eta+s\delta) \\
	&= \inner{g(\yobs), \delta} - \inner{ \E_{\eta+s\delta}g(Y),\delta }\\
	&= - \inner{ \E_{\eta+s\delta}g(Y) - g(\yobs),\delta },
\end{align*}
which is greater than or equal 0 by \eqref{E:exp_gy}.
Thus $\ell(\eta+s\delta)$ is a non-decreasing function of $s$.

\textbf{From $1 \to 2$}:
\begin{align}
	\ell( \eta+s\delta) &= \inner{g(\yobs), \eta+s\delta} - c(\eta+s\delta) \notag \\ 
	&= \inner{g(\yobs), \eta} + s \inner{g(\yobs),\delta} -bs +bs - c(\eta+s\delta) \notag \\ 
	&= \inner{g(\yobs), \eta} + s [\inner{g(\yobs),\delta} -b]  - \log e^{c(\eta+s\delta) -bs}. \label{E:expanded ell}
\end{align}
%Theorem~8.6 in \citep{Rockafellar:1970} says that $\ell(\theta+s\delta)$ 
%is non-decreasing if and only 
%if $\lim_{s \to \infty} \ell(\theta+s\delta) > -\infty$.  This implies that in the 
%above expression, for the right-most term, $b \geq \sigma_c(\delta)$, and also that

By assumption, $\ell(\eta + s\delta)$ is a non-decreasing function of $s$.  
Then for $\ell(\eta ) > -\infty$, $\ell(\eta + s\delta) \geq \ell(\eta )  > -\infty$
for any $s>0$ and thus $\lim_{s \to \infty} \ell(\eta+s\delta) > -\infty$.
By Theorem~\ref{Thm:e_c}, in order for the righthand side of \eqref{E:expanded ell} 
to be greater than $-\infty$ as $s \to +\infty$ requires
\begin{align*}
	\inner{g(\yobs), \delta} - b \geq 0,
\end{align*}
and
\begin{align*}
	b \geq \sigma_C(\delta). % \sup_{g(y)\in C} \inner{ g(y), \delta }.
\end{align*}
Then
\begin{align*}
	\sigma_C(\delta)  - \inner{g(\yobs), \delta} \leq b - \inner{g(\yobs), \delta}  \leq 0,
%#	\inner{y, \delta} - \inner{Y_{max},\delta } \geq 0.
\end{align*}
and recalling that $\sigma_C(\delta) = \sup_{g(y) \in C} \inner{g(y), \delta}$, 
we conclude that
\begin{align*}
	\inner{g(Y) - g(\yobs),\delta } \leq 0
\end{align*}
almost surely.
\end{proof}
%%%%%%%%%%%%%%%%%%%%%%%%%%%%%%%%%

Theorems~\ref{Thm:DOC} and \ref{Thm:DOR} induce the following criteria about the
existence of the MLE in the conventional sense:

%%%%%%%%%%%%%%%%%%%%%
\begin{theorem}[Theorem 4, \citep{Geyer:gdor}] \label{Thm:MLE existence}
For a full exponential family with the setting as Theorem~\ref{Thm:DOC}, the 
following are equivalent 
\begin{enumerate}
\item the MLE exist.
\item Every direction of recession is a direction of constancy.
\item $N_C(g(\yobs))$ is a vector subspace.
\item $T_C(g(\yobs))$ is a vector subspace.
\end{enumerate}
\end{theorem}

Thus the MLE does not exist in the conventional sense if there exists a vector 
$\delta$ that is a direction of recession but not a direction of constancy.  
If in addition $\delta \in \rint N_C(g(\yobs))$, then $\delta$ is called a 
\emph{generic direction of recession} (GDOR).  \hl{Normal cones are convex sets and
relative interiors of nonempty convex sets are nonempty.}  Thus 
a GDOR exists if and only if the MLE does not exist.

Theorem~\ref{Thm:MLE rint} relates the non-existence of the MLE to the location
of $g(\yobs)$ relative to the boundary of $C$.  Here, Theorems \ref{Thm:DOC}, \ref{Thm:DOR}, and \ref{Thm:MLE existence} relate the non-existence of the MLE
to a strictly increasing log likelihood function.  This in turn
occurs when $g(\yobs)$ is such that $\inner{g(Y) - g(\yobs)} < 0$.  This is
exactly the criteria that puts $g(\yobs)$ on the boundary of the convex support,
thus agreeing with Theorem~\ref{Thm:MLE rint}.

%%%%%%%%%%%%%%%%%%%%%
\begin{corollary}[Corollary 5, \citep{Geyer:gdor}] \label{Cor:strictly increasing}
For a full exponential family with the setting as Theorem~\ref{Thm:DOC}, if $\delta$ is a
direction of recession that is not a direction of constancy, 
then for all $\eta \in \Xi$, the function $s \mapsto \ell(\eta+s\delta)$ is strictly
increasing on the interval where it is finite.
\end{corollary}

This corollary is clearly implied by Theorems \ref{Thm:DOC} and \ref{Thm:DOR}; if
$\delta$ is a direction of recession that is not a direction of constancy, 
the $\ell(\eta+s\delta)$ is nondecreasing with respect to $s$ by 
Theorem \ref{Thm:DOR} but cannot be constant by Theorem \ref{Thm:DOC}.  Therefore,
it must be strictly increasing.


%%%%%%%%%%%%%%%%%%%%%

\subsection{Limiting conditional model}
When the MLE does not exist for an exponential family in the conventional sense, there 
exists a direction of recession that is not a direction of constancy 
along which the log likelihood goes to $+\infty$.  The behavior of the 
density function of the distribution is described in the following theorem:

\begin{theorem}[Theorem 6 from \citet{Geyer:gdor}] \label{Thm:LCM}
For a full exponential family with the setting of Theorem~\ref{Thm:DOC}, and 
additionally,
\begin{enumerate}
\item density function $f_{\eta}(y)$ defined by \eqref{E:ERGM},
\item direction of recession, $\delta$,
\item $H = \{ w \in \RR^d: \inner{ w-g(\yobs), \delta } = 0 \}$,
\item $P( g(Y) \in H) > 0$ for some distribution in the family,
\end{enumerate}
then for all $\eta \in \Xi$
\begin{align} \label{E:LCM}
\lim_{s \to \infty} f_{\eta+s\delta}(y) = 
			\begin{cases} 
			0 								& \inner{g(y) - g(\yobs), \delta} < 0 \\
			\frac{f_\eta(y)}{P_\eta(g(Y) \in H)} 	& \inner{g(y) - g(\yobs),
\delta} = 0 \\
			+\infty							& \inner{g(y) - g(\yobs), \delta} > 0.
		\end{cases}
\end{align}
If $\delta$ is not a direction of constancy, 
then $s \mapsto P_{\eta+s\delta}( g(Y) \in H)$ is continuous and 
strictly increasing, and $P_{\eta+s\delta}( g(Y) \in H) \to 1$ as $s \to \infty$.
\end{theorem}


%%%%% BEGIN PROOF
\begin{proof}
By Theorem~\ref{Thm:DOR}, if $\delta$ is a direction of recession, then 
$\inner{g(Y) - g(\yobs), \delta} \leq 0$ almost surely, which implies 
that $\inner{g(Y), \delta} \leq \inner{g(\yobs), \delta}$.  
So, the largest value that $\inner{g(Y), \delta}$ can take is 
$\inner{g(\yobs), \delta}$.  Using our earlier notation, 
\begin{align*}
\sigma_C(\delta) = \sup_{g(y) \in C}\inner{g(y), \delta} = 
\inner{g(\yobs), \delta}.
\end{align*}

From \eqref{E:ERGM}, we can express $f(_{\eta+s\delta}(y)$ as
\begin{align*}
 f_{\eta+s\delta}(y) &= e^{ \inner{\eta+s\delta,g(y)} - c(\eta+s\delta)  } \\
 &= e^{ \inner{\eta,g(y)} -c(\eta)+c(\eta) + s\inner{\delta,g(y)} - c(\eta+s\delta) } \\
 	&= f_\eta(y) \frac{e^{c(\eta)}}{e^{ c(\eta+s\delta) - \inner{g(y),\delta}
s } }.
\end{align*}
The denominator in the fraction above has the form of the starting expression in 
Theorem~\ref{Thm:e_c} with $\inner{g(y),\delta}$ in place of $b$.
Thus we can take the limit of $f(_{\eta+s\delta}(y)$ as
$s \to +\infty$ by applying Theorem~\ref{Thm:e_c},

\begin{align*}
	f_{\eta+s\delta}(y) = f_\eta(\eta) \frac{e^{c(\eta)}}{e^{ c(\eta+s
\delta) - \inner{g(y),\delta}s } } 
	\to	
			\begin{cases} 
			0 					& \inner{g(y),\delta} < \inner{g(\yobs),\delta} \\
			\frac{f_\eta(y)}{P_\eta(g(Y) \in H)} 	& 
								\inner{g(y),\delta} = \inner{g(\yobs),\delta} \\
			+\infty				& \inner{g(y),\delta} > \inner{g(\yobs),\delta}.
			\end{cases}
\end{align*}
This gets us \eqref{E:LCM}.  We next show the behavior of 
$P_{\eta+s\delta}(g(Y) \in H)$ as $s \to +\infty$ when $\delta$ is a direction of
recession that is not a direction of constancy.

\begin{align*}
 P_{\eta+s\delta}(g(Y) \in H) &= \int_H e^{\inner{g(y), \eta+s\delta} - c(\eta
+s\delta)} \, d\mu(y) \\
		&= \int_H  \frac{e^{c(\eta)}}{e^{c(\eta+s\delta)-\inner{g(y),\delta}s}} 
					f_\eta(\eta) \, d\mu(y)\\
		&= \E_\eta  \left ( I_H \frac{e^{c(\eta)}}{e^{c(\eta+s\delta)-\inner{g(y),\delta}s}} \right )
\end{align*}
Since the indicator function $I_H$ evaluates to zero unless 
$\inner{ g(y), \delta} = \inner{ g(\yobs), \delta} $,  
this expression can pulled out of the expectation along with other constants, so that 
\begin{align*}
		 P_{\eta+s\delta}(g(Y) \in H)
		 &= \frac{e^{c(\eta)} }{ e^{ c(\eta+s\delta) -s\inner{g(\yobs),\delta} } }
		 \E_\eta  I_H   \\
		 &= \frac{e^{c(\eta)} }{ e^{ c(\eta+s\delta) -s\inner{g(\yobs),\delta} } }
		 P_\eta  (g(Y) \in H). 
		 \end{align*}
 
Then taking the limit of $s \to +\infty$ via Theorem~\ref{Thm:e_c} gives
\begin{align*}
 P_{\eta+s\delta}(g(Y) \in H)
		&\to \frac{e^{c(\eta)}}{ e^{c(\eta)} P_\eta (g(Y) \in H) } P_\eta (g(Y)\in H) = 1
 \end{align*}
 as desired.   
\end{proof}

%%%%% END PROOF

%%%%%%%%%%%%%%%%%%%%%%%%%%%%%%%%%%%%%%%%%%%%%%%%%%%%%
%\subsection{Remarks on Theorem~\ref{Thm:LCM}}
There are many implications of Theorem~\ref{Thm:LCM} described in \citep{Geyer:gdor}
which we repeat here, though we go into more detail for further clarity.

The event $\inner{g(Y),\delta} > \inner{g(\yobs),\delta}$ has zero probability by 
Theorem~\ref{Thm:DOR}(2) and thus the $+\infty$ case need not be considered.
The right-hand side of \eqref{E:LCM} can be viewed as a conditional density of a 
distribution with parameter $\eta$ given $g(Y) \in H$, and expressed as $f_{\eta}
(\, \cdot\,  \mid g(y) \in H)$.  This is because the numerator
is in fact a joint density, which can be reasoned as follows.

Let $W = I(g(Y) \in H)$, which has density 
\begin{align*}
	f_\eta(w) &= \begin{cases}
					P_\eta(g(Y) \in H) \quad \text{for $w=1$} \\
					P_\eta(g(Y) \notin H) \quad \text{for $w=0$}.
				\end{cases}
\end{align*}

The conditional density of $W \mid Y$ is
\begin{align*}
	f_\eta(w \mid y) &= \begin{cases}
 			\begin{cases}
			1 	\quad &\text{for $w=0$}\\
			0 	\quad &\text{for $w=1$}\\
			\end{cases} \quad &\text{for } g(y) \notin H\\
 			\begin{cases}
			0 	\quad &\text{for $w=0$}\\
			1 	\quad &\text{for $w=1$}\\
			\end{cases} \quad &\text{for } g(y) \in H.
 		\end{cases}
\end{align*}

Then the conditional density for $Y \mid [g(Y)\in H]$ can be expressed
\begin{align*}
	f_\eta(y \mid g(y) \in H) &= f_\eta(y \mid w=1) \\
					&= \frac{f_\eta(w=1, y)}{f_\eta(w=1)}	
					= \frac{f_\eta(w=1 \mid y) f_\eta(y)}{P_\eta(g(Y) \in H)}  \\%\quad 
%					 \text{for $g(y) \in H$}	\\
	&= \begin{cases}
 			\frac{0 \cdot f_\eta(y)}{P_\eta(g(Y) \in H)}   \quad 
					 \text{for $g(y) \notin H$}	\\
 			\frac{1 \cdot f_\eta(y)}{P_\eta(g(Y) \in H)}   \quad 
					 \text{for $g(y) \in H$}	\\
 		\end{cases}
\end{align*}
which is exactly the right-hand side of \eqref{E:LCM}.

The log likelihood for $f_\eta( \, \cdot\,  \mid g(y) \in H)$ for $\eta \in \Xi$ can be expressed as
\begin{align*}
%	\ell_{LCM}(\eta) &= 
	\inner{\eta, g(\yobs)} - c(\eta) - \log P_\eta(g(Y) \in H) 
%			&= \ell(\eta) - \log P_\eta(g(Y) \in H)
\end{align*}
which clearly has exponential family form with the same natural parameter $\eta$ and 
statistic $g(\yobs)$ as the original exponential family.  The cumulant function is different,
however, and thus the family may not be full.  The full family containing this exponential family is at least as large as
\begin{align*}
\{ f_{\eta}(\, \cdot\,  \mid g(Y) \in H) \text{ for }  \eta + \gamma: \eta \in \Xi \text
{ and } \gamma \in \Gammalim \},
\end{align*}
where $\Gammalim$ is the constancy space of $f_{\eta}( \, \cdot\,  \mid g(Y) \in H)$,
and is called the \emph{limiting conditional model} (LCM) \citep{Geyer:gdor}.  

When $\delta$ is a direction of recession that is not a direction of constnacy, 
we can succinctly summarize the result of Theorem~\ref{Thm:LCM} as
\begin{align*}
\lim_{s \to \infty} f_{\eta+s\delta}(y) = f_{\eta}( y \mid g(y) \in H).
\end{align*}

We can express the log likelihood of the LCM as a function of the 
log likelihood of the original family,
\begin{align} \label{E:LCM ll bound}
 \ell_{LCM}(\eta) = \ell(\eta) - \log P_\eta(g(Y) \in H)
\end{align}
implying
\begin{align*}
	\ell(\eta) < \ell_{LCM}(\eta).	
\end{align*}
The inequality above is strict since the log likelihood is strictly increasing
when $\delta$ is a GDOR by Corollary~\ref{Cor:strictly increasing}
and so $P_\eta(g(Y) \in H)$ cannot be equal to 1.
Thus even though the MLE does not exist for the original family, its log likelihood
$\ell(\eta)$ is bounded by the LCM log likelihood and the function increases strictly
towards $\ell_{LCM}(\eta)$ for a given $\eta$ as $s$ increases.





The following theorems and corollaries provide us with some foundation for
finding both a GDOR and then the MLE in the LCM.
This first theorem provides the linear programming framework through
which we can find a GDOR.  
%%%%%%%%%%%%%%%%%%%%%
\begin{theorem}[Theorem 7 from \citet{Geyer:gdor}] \label{Thm:L-GDOR}
For a full exponential family having polyhedral convex support $C$ and observed value 
of the natural statistic $g(\yobs)$ such that $g(\yobs) \in C$, 
let $T_C(g(\yobs) = \con(V)$, and define
\begin{align*}
	L = \{ v \in V: -v \in T_C(g(\yobs)) \}.
\end{align*}
Then a GDOR exists if and only if $L \neq V$, in which case a vector $\delta$ is a GDOR if and 
only if
\begin{align*}
	\inner{w, \delta} = 0, \quad w \in L \\
	\inner{w, \delta} < 0, \quad w \in V \setminus L \\
\end{align*}
\end{theorem}
It is not immediately obvious why defining $L$ in such a manner is helpful; in fact,
the motivation is rooted in linear programming.  We rely on this theorem heavily in 
Chapter~\ref{Chapter:Linear programming}.

%%%%%%%%%%%%%%%%%%%%%
\begin{corollary}[Corollary 8 from \citet{Geyer:gdor}]
Under the assumptions of the Theorem~\ref{Thm:L-GDOR}, a GDOR is not a direction of constancy.
\end{corollary}

%%%%%%%%%%%%%%%%%%%%%
\begin{corollary}[Corollary 9 from \citet{Geyer:gdor}] \label{Cor:spanL}
Under the assumptions of the Theorem~\ref{Thm:L-GDOR}, suppose $\delta$ is a GDOR.  Then
\begin{align*}
	T_{C \cap H} (g(\yobs)) &= \spanl L \\
	C \cap H &= C \cap (g(\yobs) + \spanl L ). \\
\end{align*}
\end{corollary}

The following theorem (which is new) shows that $C \cap H$ defines the support of the LCM 
in the finite state space setting that is of interest to us.  
%%%%%%%%%%%%%%%%%%%%%
\begin{theorem} \label{Thm:C-H}
For a full exponential family having finite polyhedral convex support $C$ and observed value 
of the natural statistic $g(\yobs)$ such that $g(\yobs) \in C$, suppose $\delta$ is a GDOR.
Then $C \cap H$ defines the convex support of the LCM.
\end{theorem}
\begin{proof}
Theorem~\ref{Thm:LCM} says that if $\delta$ is a GDOR, which is a direction of recession
that is not a direction of constancy, then for the original model,
\begin{align*}
	P_{\eta+s\delta}( g(Y) \in H ) \to 1.
\end{align*}
Since the LCM is the limiting distribution of the original model as $s \to +\infty$, it follows that $P^{LCM}(g(Y) \in H) = 1$.  
This puts 0 probability on points outside of $H$.  

The original model puts 0 probability on any $x \notin C$, including $H \setminus C$.  Because of the pointwise convergence of the original density to the LCM, since the original density puts zero probability on any $x \in H \setminus C$, the limiting model must as well.  Then
\begin{align*}
	P^{LCM}(g(Y) \in H) &= P^{LCM} \left( g(Y) \in ( (H \setminus C)  \cup (H \cap C)   \right ) \\
	 			&= P^{LCM} \left( g(Y) \in H \setminus C \right) 
								+ P^{LCM} \left( g(Y) \in H \cap C \right ) \\
	 			&= 0 + P^{LCM} \left( g(Y) \in H \cap C  \right ) = 1.
\end{align*}

Left to show is that there is no smaller set than $C \cap H$ that 
can be the convex support of the LCM.
%Let $S = g(\YY)$, and define $A = \con( S \cap H )$.  Does $A = C \cap H )$?
%Clearly $ A \subset (C \cap H)$.  But is there an $x \in C\cap H$ where $x \notin A$?
Let $S = g(\YY)$.  The convex support of the LCM by construction is $\con(S \cap H)$, 
which is closed since $S$ is finite.  
%Clearly, $(S \cap H) \subset (C \cap H)$.  
Is $(C \cap H) = \con(S \cap H)$?
If a point $x \in C$, it is a convex combination of a finite set points in $S$,
\begin{align*}
	x = \alpha_1 w_1 + \alpha_2 w_2 + \cdots + \alpha_n w_n
\end{align*}
where $\alpha_i \geq 0$, $\sum_i \alpha_i = 1$, and $w_i \in S$.

We claim for any $x \in C \cap H$, the coefficients of the $w_i$ not in $S \cap H$ must all be zero.  Suppose to get a contradiction that for an $x \in S \cap H$, there exist $\alpha_j > 0$ for $w_j \notin S \cap H$ for $j \in J$.  Then
\begin{align*}
	\inner{x - g(\yobs), \delta } &= \left \inner{ \sum_i \alpha_i w_i - g(\yobs), \delta \right } \\
	&= \left \inner{ \sum_i \alpha_i \left( w_i - g(\yobs) \right), \delta \right }\\
	&=  \left \inner{ \sum_{i \notin J} \alpha_i \left( w_i - g(\yobs) \right), \delta \right }
		+ \left \inner{ \sum_{j\in J} \alpha_j \left( w_j - g(\yobs) \right), \delta \right }
\end{align*}
where the first term is less than or equal to zero by the definition of a direction of
recession, and the second term is strictly less than zero since $w_j \notin H$.  Thus 
$\inner{x - g(\yobs), \delta } < 0$, which means $x \notin H$, which is a contradiction.

Thus $x$ will equal a convex combination of just the points in $S \cap H$.  Since this is 
true for any $x \in C \cap H$, we have that $C \cap H = \con( S\cap H)$.
\end{proof}



Theorem~\ref{Thm:C-H} and Corollary~\ref{Cor:spanL} tells us that by maximizing the 
original log likelihood along a GDOR, we will converge to an LCM which has $C \cap H$ for
its convex support.  Because the tangent cone $T_{C \cap H} (g(\yobs))$ 
is in fact a vector subspace, then by Theorem~\ref{Thm:MLE existence}, the MLE for 
this model must exist.  Corollary~\ref{Cor:spanL} also tells us that when the MLE
does not exist, $g(\yobs)$ lies in the relative interior of $C \cap H$.

The last part of Theorem~\ref{Thm:LCM} says that for the sequence of exponential families with parameter 
value $\eta+s\delta$, the probability of $g(Y)$ occurring on the plane $H$ is strictly 
increasing, going to 1.  We may think of this as 
``probability accumulating on the boundary"---where by boundary 
we  actually mean the face $C \cap H$ on which $g(\yobs)$ sits in the relative interior.

\hl{Thus} the MLE for the LCM maximizes the likelihood in \hl{the family} that is the 
union of the LCM and the original family.  When 
this happens, we say we have found an MLE in the Barndorff-Nielsen completion of the 
original family.


From a practical perspective, Theorem~\ref{Thm:LCM} tells us that when 
$\delta$ is a GDOR, $\etaLCM$ is the MLE of the LCM, and $\gamma \in \Gammalim$, then
\begin{align*}
	\lim_{s \to +\infty} \ell(\etaLCM + \gamma + s\delta) = \sup_{\RR^d} \ell(\eta).
\end{align*}

When no GDOR exists and thus the MLE exists in the conventional sense, then the 
original family is the LCM, corresponding to $\delta=0$ in Theorem~\ref{Thm:LCM}.

%%%%%%%%%%%%%%%%%%%%%%%%%%%%%%%%%%%%%%%%%%%%%%%%%%%%%
%\subsection{Summary}
%By the well-known theorem of MLE existence for exponential families (Theorem~\ref{Thm:MLE rint}), when the observed statistic, $g(\yobs)$, is 
%on the boundary of the convex support $C$, the MLE does not exist in the conventional 
%sense.
%Here, we have expressed this result with more detail: by Theorem~\ref{Thm:MLE existence}, there must exist a GDOR.
%
%According to Theorem~\ref{Thm:LCM}, as we move in the natural parameter space in the 
%direction of a GDOR, the distribution with this parameter value will 
%put increasing probability on points for  
%on the face $H \cap C$ on which $g(\yobs)$ lies in the interior of.  
%This face is orthogonal to $\delta
%$ by construction.  Samples generated 
%from these models will therefore increasingly fall on the face on which $g(\yobs)$ 
%lies as $\eta$ increases in the direction of $\delta$.  Eventually, all generated sample points will fall on this boundary.


%%%%%%%%%%%%%%%%%%%%%%%%%%%%%%%%%%%%%%%%%%%%%%%%%%%%%
\section{One-sided confidence interval}

Consider the probability distributions defined with log likelihood $\ell( \eta + s 
\delta)$ as defined 
by \eqref{E:loglike}, where $s$ is a real scalar, $\delta$ is a GDOR and 
we have found $\etaLCM$, the MLE in the LCM, and $\gamma \in \Gammalim$
(and hence $\etaLCM + \gamma$ also indexes the same LCM MLE distribution).  Then 
as $s$ goes from $-\infty$ to $+\infty$, the probability of observing $g(Y) \in H$ 
goes from zero to one in a strictly increasing manner, per Theorem~\ref{Thm:LCM}.  Thus we can find the unique $s$, call it $\hat{s}$, that makes 
this probability 0.05.  That is, we find the unique $s$ such that
\begin{align*}
	P_{\etaLCM + \gamma + s\delta}\left (g(Y) \in H \right ) = \alpha.
\end{align*}
Then $[\hat{s}, +\infty)$ is a 95\% confidence interval for 
the scalar parameter 
$s$, and, in turn, $[\etaLCM + \gamma + \hat{s}\delta, +\infty)$ gives 
a 95\% confidence interval for $\etaMLE$.

\subsection{Confidence interval for mean value parameterization}
The confidence intervals calculated for the natural parameters can of 
course be converted to those for mean value parameters, where the observed
data, and hence MLE, are known.  

