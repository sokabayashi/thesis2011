All of the parameter estimation issues discussed thus far for ERGMs are relevant to 
the larger class of discrete state space exponential family models.
These models are commonly used to model phenomena with dependent structure, 
where the outcomes of the response variable of interest are in fact dependent on one 
another.  For example, the Ising 
model \citep{Ising,Potts} is an exponential family model that has been used to model 
ferromagnetism.  
%A realized 
%sample from this model is depicted in Figure~\ref{F:pottsimage}, where neighboring 
%pixels (representing atoms in a crystal lattice) are more likely to have the same 
%color.  
%We explore this model further in Section~\ref{S:Examples:Ising}.
Other examples of phenomena with dependent structure modeled with exponential 
families include
plant ecology \citep{Besag:1974,Besag:1975}, DNA fingerprint data \citep{Geyer:1992}
 and the lifetime fitness of plants \citep
{Shaw:2008}.

Although motivated by ERGMs, the algorithm we propose is rooted in fundamental
exponential family theory and applicable to any finite state space 
exponential family model.  Thus the theory presented in this section is from
the perspective of general exponential families.

\section{Exponential Family Theory}
Much of the basic background for exponential families has already been
covered in Section~\ref{S:ERGM setup}; our focus is on regular exponential
families on a finite sample space $\YY$  with log likelihood \eqref{E:loglike}.
As noted earlier, when the sample space $\YY$ is even moderately large,
the cumulant function $c(\eta)$ involves a summation that may be prohibitively 
expensive to evaluate.  For example, the sample space $\YY$ for an Ising model 
defined on a $32\times 32$ square lattice where each entry takes values of 0 or 1 
has $2^{1024} \approx 10^{300}$ elements.  
A loop with this many iterations takes too long no matter how programmed.

A useful property of all exponential families \cite[p.~27]{TPE2} on which we rely 
heavily is that 
\begin{align*}
	\E_\eta(g(Y)) &= \nabla c(\eta)	\\
	\Var_\eta(g(Y)) &= \nabla^2 c( \eta ).
\end{align*}

Thus we can express first and second derivatives of the log likelihood \eqref
{E:loglike} and Fisher information, $I
(\eta)$, as
\begin{align}
	\nabla \ell( \eta ) &= g(y) - \E_\eta g(Y) \label{E:nabla ell} \\
	\nabla^2 \ell( \eta ) &=  - \Var_\eta g(Y) \label{E:nabla2 ell} \\
	\I(\eta) &= -\E_\eta \nabla^2 \ell (\eta ) = \Var_\eta g(Y) \label{E:FI}
\end{align}
and thereby avoid evaluation of the problematic cumulant function $c$.


By the strict convexity of the log likelihood function ensured by \eqref{E:nabla2 
ell}, the global 
maximum, if it is exists, is attained when $\eta$ is such that $\nabla \ell( \eta ) = 
0$, or, by \eqref{E:nabla ell},
\begin{align}
	\E_\eta g(Y) = g(\yobs). \label{E:Observed-Expected}
\end{align}


\section{Convex Analysis}
The issue of MLE existence in the conventional sense in an exponential family is 
closely tied to the geometric properties of 
the convex support of the model \citep{Barndorff, Geyer:gdor, Rinaldo:2009}.  We 
describe the relevant theory from convex analysis as it pertains to the case of 
exponential families.

A \emph{convex polytope} $C$ is the convex hull of a finite set of points $V$,
\begin{align*}
	C = \con( V ).
\end{align*}
where [\textbf{DEFINE pos().}]

  By the Minkowski-Weyl theorem \citep[Theorem 19.1]{Rockafellar:1970}, this convex 
set can equivalently be represented as the intersection of a finite collection of 
closed half-spaces.  These two representations of a convex polyhedron are referred to 
as the \emph{V-representation} and \emph{H-representation}, respectively.  
%The V-representation of a convex polyhedron $C$ is the set of all linear combinations
%\begin{align*}
%	\sum_{i \in E \cup I} b_i \alpha_i
%\end{align*}
%where $\alpha_i$ are vectors, $b_i$ are scalars, $E$ and $I$ are disjoint finite sets 
%such that
%\begin{align*}
%	b_i \geq 0, \quad i \in E \cup I
%\end{align*}
%and if $I$ is nonempty
%\begin{align*}
%	\sum_{i \in I} b_i = 1.
%\end{align*}

The H-representation can be expressed as the solution set of a finite set of linear 
equations and inequalities,
\begin{align*}
	C = \{x: Ax \leq b \},
\end{align*}
where $A$ is a matrix and $b$ a vector.

The \emph{relative interior} of a convex set $C$, denoted $\rint C$, is the interior 
relative to its affine hull.  
\textbf{[WHAT ELSE? bd()?  int()?  pos()?]}

A nonempty \emph{face} of a convex polyhedron $C$ is a convex subset of $C$ such that 
every line segment in $C$ with a relative interior point in $F$ has both end points in 
$F$ \citep{Rockafellar:1970}.  It is itself a convex polyhedron.
A \emph{proper} face is a face that is not the empty set or $C$, and 
\emph{facets} are proper faces of the highest dimension.

The \emph{tangent cone} of a convex set $C$ at a point $x \in C$ is
\begin{align*}
	T_C(x) = \cl\{s(w-x):w \in C \text{ and } s \geq 0 \},
\end{align*}
where $\cl$ denotes the closure operation \citep[Theorem 6.9]{Rockafellar}.  

The \emph{normal cone} of a convex set $C$ in $\RR^d$ at a point $x \in C$ is 
\begin{align*}
	N_C(x) = \{ \delta \in \RR^d: \inner{w-x,\delta} \leq 0 \text{ for all } w \in C 
\}.
\end{align*}

Tangent and normal cones are polars of each other, that is, each determines the other.  
The normal cone at $x$ can be defined in terms of the tangent cone at $x$ by
\begin{align*}
	N_C(x) 	&= \{ w \in \RR^d: \inner{ w, v } \leq 0 \text{ for all } v \in T_C(x) \}.
\end{align*}

DEFINE \emph{direction of recession}, \emph{direction of constancy}.  \textbf{get from 
\citep{Rockafellar:1970}.}



\section{MLE existence in exponential families}
We now present two equivalent approaches to determine the existence of an MLE
in the conventional sense.  

The well-known condition for the existence of the MLE \citep{Barndorff, Brown:1986} 
relates the relative location
of the observed statistic to the boundaries of the convex support
and is formally stated as follows:
\begin{theorem} \label{Thm:MLE rint}
Under the conditions [CONDITIONS], the MLE exists in the conventional sense and is 
unique if and only if 
$g(\yobs) \in \rint(C)$.
\end{theorem}

From this one can conclude whether or not the MLE exists in the conventional sense
if $C$ is known.

\citet{Geyer:1990,Geyer:gdor} present theory that relates MLE existence to not only
the boundary of the convex support, but also to how the log likelihood behaves along
directions orthogonal to the boundaries.
CLEAN THIS UP.

In order to do so, we revisit some of these results with simplified proofs:
Adapted here using the notation from \citep{Geyer:gdor},
\begin{theorem}[Theorem 2.2 in \citep{Geyer:1990}]
\begin{align*}
e^{c(\theta + s \delta) - bs} &\to 
		\begin{cases} 
			0 								& b > \sigma_c(\delta) \\
			e^{c(\theta)} P_\theta(Y \in H ) 	& b = \sigma_c(\delta) \\
			+\infty							& b < \sigma_c(\delta)
		\end{cases}
& \text{as } s \to +\infty.
\end{align*}
where $\delta$ is a non-zero direction, $C$ the convex support, and
\begin{align*}
	\sigma_C (\delta) = \sup_{y \in C} \inner{ y, \delta} \\
	H_\delta = \set{w: \inner{w, \delta} = \sigma_C(\delta) }.
\end{align*}
\end{theorem}
$H_\delta$ is the supporting hyperplane to the set $C$ with normal vector $\delta$.

\begin{proof}
\textbf{Case: $b = \sigma_C(\delta)$.}

Starting with density of the exponential family with parameter $\theta$, we know that
\begin{align*}
	e^{c(\theta)} = \int h(y) e^{\inner{y,\theta}} \, dy.
\end{align*}
So,
\begin{align*}
	e^{ c(\theta + s \delta ) - bs } &= \int h(y) e^{\inner{y,\theta + s \delta - 
bs} } \, dy. \\
									&= \int h(y) e^{\inner{y,\theta}  + s [ \inner
{y,\delta} - b ] } \, dy. 
\end{align*}
Multiply by $\frac{p_\theta(y)}{p_\theta(y)}$ to get
\begin{align*}
	e^{ c(\theta + s \delta ) - bs } &= \int h(y) e^{\inner{y,\theta}  + s [ \inner
{y,\delta} - b ] }  \frac{ h(y)e^{\inner{y, \theta} - c(\theta)} }{ h(y)e^{\inner{y, 
\theta} - c(\theta)} }\, dy \\
	&= \int e^{  s [ \inner{y,\delta} - b ] + c(\theta) }  h(y)e^{\inner{y, \theta} - 
c(\theta)} \, dy \\
	&= \E_\theta e^{  s [ \inner{Y,\delta} - b ] + c(\theta) }
\end{align*}
What happens as $s \to +\infty$?  We would like to reverse the order of taking the 
limit and expectation.  Fortunately, we have the monotone convergence theorem.  For $
\inner{y, \delta} \leq b$, we have a monotonically decreasing sequence of random 
variables.  For $\inner{y, \delta} > b$, the sequence is increasing.  Thus,
\begin{align*}
	\lim_{s\to \infty} \E_\theta e^{  s [ \inner{Y,\delta} - b ] + c(\theta) } &= E_
\theta \lim_{s\to \infty} e^{  s [ \inner{Y,\delta} - b ] + c(\theta) }. 
\end{align*}
Ignoring the expectation and examining the just limit component of the above,
\begin{align*}
	\lim_{s\to \infty} e^{  s [ \inner{Y,\delta} - b ] + c(\theta) } &= 
			\begin{cases} 
			0 								& \inner{Y,\delta} < b \\
			e^{c(\theta)} 		 			& \inner{Y,\delta} = b \\
			+\infty							& \inner{Y,\delta} > b.
		\end{cases}
\end{align*}
In the case that we are considering, however, $b = \sigma_C(\delta) = \sup_{y \in C}
\inner{y,\delta}$, so $\inner{Y,\delta}$ can never be greater than $b$, so the third 
case in the limit above is not possible.  Thus we can rewrite the result above 
succinctly as
\begin{align*}
	\lim_{s\to \infty} e^{  s [ \inner{Y,\delta} - b ] + c(\theta) } &= I(\inner{Y,
\delta} = b ) e^{c(\theta)} 
	= I( Y \in H_\delta ) e^{c(\theta)}.
\end{align*}
Then returning to the full expression,
\begin{align*}
	\lim_{s\to \infty} \E_\theta e^{  s [ \inner{Y,\delta} - b ] + c(\theta) } &= e^
{c(\theta)} P( Y \in H_\delta ).
\end{align*}

\textbf{Case: $b > \sigma_C(\delta)$}.
\begin{align*}
	\lim_{s\to \infty} e^{ c(\theta + s \delta ) - bs } &= 
	\lim_{s\to \infty} e^{ c(\theta + s \delta ) - \sigma_C(\delta)s + \sigma_C
(\delta)s - bs }  \\
	&= \left( \lim_{s\to \infty} e^{ c(\theta + s \delta ) - \sigma_C(\delta)s} 
\right ) \left(  \lim_{s\to \infty} e^{ s[ \sigma_C(\delta) - b] } \right )\\
	&= \left (e^{c(\theta) }P(Y\in H_\delta) \right ) \cdot 0 = 0.
\end{align*}

\textbf{Case: $b < \sigma_C(\delta)$}.
\begin{align*}
	\lim_{s\to \infty} e^{ c(\theta + s \delta ) - bs } &= 
	\lim_{s\to \infty} e^{ c(\theta + s \delta ) - \sigma_C(\delta)s + \sigma_C
(\delta)s - bs }  \\
	&= \left( \lim_{s\to \infty} e^{ c(\theta + s \delta ) - \sigma_C(\delta)s} 
\right ) \left(  \lim_{s\to \infty} e^{ s[ \sigma_C(\delta) - b] } \right )\\
	&= \left (e^{c(\theta) }P(Y\in H_\delta) \right ) \cdot \left ( + \infty \right ) 
= + \infty.
\end{align*}
\end{proof}


\begin{theorem}[Theorem 1 from \citet{Geyer:gdor}] \label{Thm:DOC}
For a full exponential family with
\begin{itemize}
\item log likelihood function, $\ell(\eta)$, as in \eqref{E:loglike},
\item natural parameter space, $\Xi$, as in \eqref{E:paramspace},
\item natural statistic, $g(Y)$,
\item observed data, $\yobs$, such that $g(\yobs) \in C$, the convex 
support,
\end{itemize}
the following are equivalent:
\begin{enumerate}
\item $\delta$ is a direction of constancy of the log likelihood.
\item For all $\eta \in \Xi$, the function $s \mapsto \ell( \eta + s\delta)$ is 
constant on $\RR$ \cite[Theorem 1 (b)]{Geyer:gdor}.
\item The parameter values $\eta$ and  $\eta + s\delta$ correspond to the same 
probability distribution for all $\eta \in \Xi$ and all real $s$ \cite[Theorem 1 (d)]
{Geyer:gdor}.
\item $\inner{g(Y) - g(\yobs),\delta} = 0$ almost surely for all distributions in the 
family \cite[Theorem 1 (f)]{Geyer:gdor}.
\item $\delta \in N_C(g(\yobs))$ and $-\delta \in N_C(g(\yobs))$ \cite[Theorem 1 (g)]
{Geyer:gdor}.
\item $\inner{w,\delta} = 0$ for all $w \in T_C(g(\yobs))$ \cite[Theorem 1 (h)]
{Geyer:gdor}.
\end{enumerate}
\end{theorem}
The set of all directions of constancy is called the \emph{constancy space} of the log 
likelihood.

\begin{theorem}[DOR: Theorem 3 from \cite{Geyer:gdor}] \label{Thm:DOR}
For a full exponential family with the same setting as Theorem~\ref{Thm:DOC},
%\begin{itemize}
%\item log likelihood function, $\ell(\eta)$, described by \eqref{E:loglike},
%\item natural parameter space, $\Xi$,
%\item natural statistic, $g(Y)$,
%\item observed value of the natural statistic, $g(\yobs)$, such that $g(\yobs) \in C$, the convex 
%support,
%\end{itemize}
the following are equivalent:
\begin{enumerate}
\item $\delta$ is a direction of recession of the log likelihood.
\item For all $\eta \in \Xi$, the function $s \mapsto \ell( \eta + s\delta)$ is 
nondecreasing on $\RR$ \cite[Theorem 3 (b)]{Geyer:gdor}.
\item $\inner{g(Y) - g(\yobs),\delta} \leq 0$ almost surely for all distributions in 
the family. \cite[Theorem 3 (d)]{Geyer:gdor}.
\item $\delta \in N_C(g(\yobs))$ \cite[Theorem 3 (e)]{Geyer:gdor}.
\item $\inner{w,\delta} \leq 0$ for all $w \in T_C(g(\yobs))$ \cite[Theorem 3 (f)]
{Geyer:gdor}.
\end{enumerate}
\end{theorem}


%%%%%%%%%%%%%%%%%%%%%%% 1/26/11 - Added from Algorithm.tex
\subsection{Theorem 3(a) and (c), \citet[p. 270]{Geyer:gdor}}
So what do we this theorem for?  It relates when $\delta$ is a direction of recession 
or not.  Theorem 3 in GDOR has says the following are equivalent:
\begin{itemize}
\item (a) There exists a $\theta \in \Theta$ such that the function $s \mapsto \ell
(\theta+s\delta)$ is nondecreasing on $\RR$.
\item (c) $\inner{Y-y, \delta} \leq 0$ almost surely for all distributions in the 
family.
\end{itemize}

The $\delta$ in (a) in the above is a direction of recession.  

The proof requires the following lemma, which a PROOF WILL BE NEEDED.
$\ell(\theta+s\delta)$ is non-dereasing if and only if $\lim_{s \to \infty} \ell
(\theta+s\delta) > -\infty$.

\textbf{From $(c) \to (a)$}:

Take $\E_{\theta + s \delta}()$ of $\inner{Y-y, \delta} \leq 0$.  Then
\begin{align*}
\inner{ \E_{\theta + s \delta} Y-y, \delta} \leq 0.
\end{align*}
Now, take the derivative of $\ell( \theta + s\delta)$ with respect to $s$.
\begin{align*}
\deriv{\ell( \theta + s \delta}{s}) &= \deriv{}{s} \left (   \inner{y, \theta+s
\delta} - c(\theta+s\delta)  \right )\\
	&= \inner{y, \delta} - \deriv{}{s} c(\theta+s\delta) \\
	&= \inner{y, \delta} - \inner{ \E_{\theta+s\delta}Y,\delta }\\
	&= - \inner{ \E_{\theta+s\delta}Y - y,\delta }.
\end{align*}
Applying the above from the assumption, we get that 
\begin{align*}
\deriv{\ell( \theta + s \delta)}{s} \geq 0.
\end{align*}
Thus $\ell(\theta+s\delta)$ is a non-decreasing function of $s$.

\textbf{From $(a) \to (c)$}:
\begin{align*}
	\ell( \theta+s\delta) &= \inner{y, \theta+s\delta} - c(\theta+s\delta) \\
	&= \inner{y, \theta} + s \inner{y,\delta} -bs +bs - c(\theta+s\delta) \\
	&= \inner{y, \theta} + s [\inner{y,\delta} -b]  - \log e^{c(\theta+s\delta) -bs}.
\end{align*}
We know from the above lemma that $\ell(\theta+s\delta)$ is non-dereasing if and only 
if $\lim_{s \to \infty} \ell(\theta+s\delta) > -\infty$.  This implies that in the 
above expression, for the right-most term, $b \geq \sigma_c(\delta)$, and also that
\begin{align*}
	\inner{y, \delta} - b \geq 0,
\end{align*}
or 
\begin{align*}
	b - \inner{y, \delta}  \leq 0.
\end{align*}
If the above is true, then 
\begin{align*}
	\sigma_C(\delta)  - \inner{y, \delta} \leq b - \inner{y, \delta}  \leq 0,
%#	\inner{y, \delta} - \inner{Y_{max},\delta } \geq 0.
\end{align*}
and recalling that $\sigma_C(\delta) = \sup_{y \in C} \inner{y, \delta}$, then
\begin{align*}
	\inner{Y - y,\delta } \leq 0.
\end{align*}
(with probability one---do I even need to add this).
%%%%%%%%%%%%%%%%%%%%%%%%%%%%%%%%%

From the two theorems above, it is clear that every direction of constancy is a 
direction of recession.  These induce the following criteria about the existence of 
the MLE in the conventional sense:

\begin{corollary}[Theorem 4, Corollary 5 \citep{Geyer:gdor}] \label{Cor:MLE DNE}
For a full exponential family with the setting as Theorem~\ref{Thm:DOC}, if $\delta$ 
is a 
direction of recession that is not a direction of constancy, then  
\begin{enumerate}
\item the MLE does not exist.
\item $N_C(g(\yobs))$ is a vector subspace.
\item $T_C(g(\yobs))$ is a vector subspace.
\item for all $\eta \in \Xi$, the function $s \mapsto \ell(\eta + s \delta)$ is 
strictly increasing on 
the interval where it is finite.
\end{enumerate}
\end{corollary}

We call $\delta$ a \emph{generic direction of recession} 
(GDOR) if $\delta \in \rint N_C(g(\yobs))$.  By construction it is not a direction of 
constancy.  By Corollary~\ref{Cor:MLE DNE}, a GDOR exists if and only if the MLE does 
not exist.


When the MLE does not exist for an exponential family in the conventional sense, there 
exists a GDOR along which the log likelihood goes to $+\infty$.  The behavior of the 
density of the distribution is described in the following theorem:

\begin{theorem}[Theorem 6 from \citet{Geyer:gdor}] \label{Thm:LCM}
For a full exponential family with the setting as Theorem~\ref{Thm:DOC}, and 
additionally,
\begin{enumerate}
\item density function $f_{\eta}(y)$ defined by \eqref{E:ERGM},
\item direction of recession, $\delta$,
\item $H = \{ w \in \RR^d: \inner{ w-g(\yobs), \delta } = 0 \}$,
\item $P( g(Y) \in H) > 0$ for some distribution in the family,
\end{enumerate}
then for all $\eta \in \Xi$
\begin{align} \label{E:LCM}
\lim_{s \to \infty} f_{\eta+s\delta}(y) = 
			\begin{cases} 
			0 								& \inner{g(Y) - g(\yobs), \delta} < 0 \\
			\frac{f_\eta(y)}{P_\eta(g(Y) \in H)} 	& \inner{g(Y) - g(\yobs),
\delta} = 0 \\
			+\infty							& \inner{g(Y) - g(\yobs), \delta} > 0.
		\end{cases}
\end{align}
If $\delta$ is not a direction of constancy, then $s \mapsto P_{\eta+s\delta}( g(Y) 
\in H)$ is continuous and strictly increasing, and 
$P_{\eta+s\delta}( g(Y) \in H) \to 
1$ as $s \to \infty$.
\end{theorem}


The right-hand side of \eqref{E:LCM} can be viewed as a conditional density of a 
distribution with parameter $\eta$ given $g(Y) \in H$ and hence expressed as $f_{\eta}
(\, \cdot\,  | g(Y) \in H)$ (the set that maps to $+\infty$ has probability zero by 
Theorem~\ref{Thm:DOR}).  Because it still has the same functional form as the 
exponential family density, $f_\eta(y)$, the family of distributions
\begin{align} \label{E:f conditional}
\{ f_{\eta}(\, \cdot\,  | g(Y) \in H) \text{ for } \eta \in \Xi \}
\end{align}
is itself an exponential family with the same natural parameters and statistics as the 
original family.  
%Then we can succinctly summarize the result of Theorem~\ref{Thm:LCM} as
%\begin{align*}
%\lim_{s \to \infty} f_{\eta+s\delta}(y) = f_{\eta}( y | g(Y) \in H).
%\end{align*}
It may not be full, however; the full family containing this exponential family,
\begin{align*}
\{ f_{\eta}(\, \cdot\,  | g(Y) \in H) \text{ for }  \eta + \gamma: \eta \in \Xi \text
{ and } \gamma \in \Gammalim \},
\end{align*}
where $\Gammalim$ is the constancy space of \eqref{E:f conditional},
is called the \emph{limiting conditional model} (LCM) \citep{Geyer:gdor}.  
\textbf{Even though the MLE does not exist, $\ell(\eta)$ is bounded by the LCM log 
likelihood.}
An MLE for this new model must exist, since $g(\yobs)$ cannot lie on the boundary of 
the convex support which was itself determined by $g(\yobs)$ lying in the interior of 
it.  

Before going on to the proof, let's think about what this means.  Suppose $\delta$ is 
a DOR and not a DOC, because the latter is not interesting.  So, the MLE does not 
exist because the log-likelihood, despite being strictly convex, is strictly 
increasing when we go in the direction of $\delta$.  That's what DOR means.  So then 
what?

Well, we need to think first about $H$ again, this plane that is perpendicular to $
\delta$ on which the observed data $y$ lies.  Looking first at the last sentence of 
this theorem, it says that for the sequence of exponential families with parameter 
value $\theta+s\delta$, the probability of $Y$ occurring on the plane $H$ is strictly 
increasing, and in fact, going to 1.  This is what Charlie is referring to when he 
says things like ``the probability is accumulating on the boundary".  So, $H$ 
actually coincides with the boundary of the convex hull, the observed data $y$ sits 
is on this boundary (hence the MLE does not exist, by the usual MLE existence 
theorem, and hence a DOR exists).  When we take the parameter value further and 
further along in the direction of the DOR, the samples generated from that model will 
increasingly cluster around that boundary face of the convex hull on which $y$ 
sits.

So how about the rest of the theorem?  We saw in Theorem 3 that $\inner{Y-y,\delta} 
\leq 0$ so the set where the above is $+\infty$ has probability zero.  I think the 
set for which $\inner{Y-y,\delta} = 0$ corresponds to a DOC, doesn't it?  Hmm, 
something's not right with my reasoning.

Charlie's paper says: the right side (excluding the 0 probability part) is the 
density of the conditional distribution given the event $Y \in H$ of the distribution 
having parameter value $\theta$, so we could denote the RHS by $f_\theta(\omega | Y 
\in H)$.  The exponential family with the \emph{full} natural parameter space (I'm 
glossing over some stuff) is the limiting condition model (LCM).  ``If an MLE exists 
for the LCM, then it maximizes the likelihood in the family that is the union of the 
LCM and the original family, and it maximizes the likelihood in the family that is 
the set of all limits of sequences of distributions in the original family.  When 
this happens, we say we have found an MLE in the Barndorff-Nielsen completion of the 
original family."

\begin{proof}
Hmm, I'm concerned about the different forms of $H$: are they actually all the same 
thing?  Well, carry on.

By Theorem 3, if $\delta$ is a DOR, then $\inner{Y-y, \delta} \leq 0$ which implies 
that $\inner{Y, \delta} \leq \inner{y, \delta}$.  So, the largest value that $\inner
{Y, \delta}$ can take on is $\inner{y, \delta}$.  Then $\sigma_C(\delta) = \inner{y, 
\delta}$.

Focus on the form of the density for $f_{\theta+s\delta}(\omega)$.  According the 
form of the relative density previously defined with respect to $\psi$,
\begin{align*}
 f_{\theta+s\delta}(\omega) &= e^{ \inner{y,\theta+s\delta - \psi} - c(\theta+s
\delta) + c(\psi)  } \\
 	&= e^{ \inner{y,\theta - \psi} -c(\theta) + c(\theta) + s \inner{y,\delta} - c
(\theta+s\delta) + c(\psi)  } \\
 	&= e^{ \inner{y,\theta - \psi} -c(\theta) + c(\psi) - c(\theta+s\delta) + s 
\inner{y,\delta} + c(\theta)   } \\
 	&= f_\theta(\omega) \frac{e^{c(\theta)}}{e^{ c(\theta+s\delta) - \inner{y,\delta}
s } }.
\end{align*}
This form is starting to look familiar, in particular, the denominator is from 
Theorem 2.2 from \citet{Geyer:1990} except with $\inner{y,\delta}$ instead of $b$ 
(where this 
$y$ is the dummy variable in the density, not the observed data).  So what happens as 
$s \to \infty$?  

It depends on how $\inner{Y(\omega),\delta}$ relates to $\sup_{y \in C}\inner{y,
\delta} = \inner{y,\delta}$, for the observed data $y$.  (our notation is getting a 
little confusing since $y$ can denote the observed data $y$ or the dummy variable in 
the density).  
\begin{align*}
	f_{\theta+s\delta}(\omega) = f_\theta(\omega) \frac{e^{c(\theta)}}{e^{ c(\theta+s
\delta) - \inner{y,\delta}s } } 
	\to	
			\begin{cases} 
			0 								& \inner{Y(\omega),\delta} < \inner{y,
\delta} \\
			\frac{f_\theta(\omega)}{P_\theta(Y \in H)} 	& \inner{Y(\omega) ,
\delta} = \inner{y,\delta} \\
			+\infty							& \inner{Y(\omega),\delta} > \inner{y,
\delta}.
	\end{cases}
\end{align*}
For $\delta$ a DOR but not DOC, we need to show that the $P_{\theta+s\delta}(Y \in H) 
\to 1$ 
as $s \to \infty$, where $H = \set{w \in \RR^p: \inner{w-\yobs,d}=0}$.
\begin{align*}
 P_{\theta+s\delta}(Y \in H) &= \int_H e^{\inner{y, \theta+s\delta - \psi} - c(\theta
+s\delta) + c(\psi)} \, d\psi \\
		&= \E_\psi \left \{ I_H e^{\inner{y, \theta - \psi} +c(\psi) - [c(\theta+s
\delta) -s\inner{y,\delta}]} \right 
\}\\
		&= \E_\psi \left \{ I_H  \frac{e^{c(\theta)} }{ e^{ c(\theta+s\delta) -s\inner
{y,\delta} } } 
		e^{\inner{y, \theta - \psi} -c(\theta)+c(\psi)}  \right \}.
\end{align*}
Since the indicator function, $I_H$, fixes $\inner{ y, \delta}$ to be a constant, it 
can pulled out of the expectation, so that 
\begin{align*}
		 P_{\theta+s\delta}(Y \in H)
		 &= \frac{e^{c(\theta)} }{ e^{ c(\theta+s\delta) -s\inner{y,\delta} } }
		 \E_\psi \left \{ I_H   
		e^{\inner{y, \theta - \psi} -c(\theta)+c(\psi)}  \right \}, \\
		 &= \frac{e^{c(\theta)} }{ e^{ c(\theta+s\delta) -s\inner{y,\delta} } } 
		 P_\theta ( Y \in H ).
		 \end{align*}
 
Then applying Theorem 2.2 from Geyer:1990,
\begin{align*}
 P_{\theta+s\delta}(Y \in H)
		&\to \frac{1}{P_\theta(Y \in H)}  P_{\theta}(Y \in H) = 1
 \end{align*}
 as $s \to \infty$.  
 
 That this probability is strictly increasing follows directly from the DOR property 
which requires $\ell( \theta+s
\delta)$ to be strictly increasing.   
\end{proof}


\subsection{log likelihood is bounded by LCM}
\subsection{summary}



In summary, by Theorem~\ref{Thm:MLE rint}, when the observed statistic, $g(\yobs)$, is 
on the boundary of the convex support $C$, the MLE does not exist in the conventional 
sense.
Then by Corollary~\ref{Cor:MLE DNE}, there must exist a direction of recession, $
\delta$, that is not a direction of constancy.  
According to Theorem~\ref{Thm:LCM}, as we move in the natural parameter space in the 
direction of $\delta$, the distribution with this parameter value will put 
increasingly probability on points for which the natural statistics, $g(Y)$, land in 
the hyperplane $H$ on which $g(\yobs)$ lies.  This hyperplane is orthogonal to $\delta
$ by construction and contains the boundary of the convex support.  Samples generated 
from these models will therefore increasingly fall on the face on which $g(\yobs)$ 
lies as $\eta$ increases in the direction of $\delta$.  Theorem~\ref{Thm:LCM} says 
that eventually, all generated sample points will fall on this boundary.


\subsection{Mean value parameterization}

An alternative parameterization of an exponential family is the mean value 
parameterization.  For each natural parameter $\eta$, we can define the mean parameter 
$\mu$ such that
\begin{align*}
	\mu = \E_\eta g(Y).
\end{align*}
The space for the mean value parameterization is the same as the convex support of the 
natural statistics and thus lends itself to easier interpretability  \citep
{Handcock:degeneracy, Rinaldo:2009}.  In particular, $\mu = g(\yobs)$ is the MLE in 
the mean value parameterization.  \citeauthor{Handcock:degeneracy} observed that mean 
value parameters located too close to the boundary of the convex support correspond to 
degenerate distributions.  

There exists a one-to-one mapping between a natural parameter and its mean value 
parameter, and one can calculate mean value parameters from natural parameters. In 
general there is no simply way to get the natural parameter value from the mean value 
parameter (if there was, then finding MLEs would be very easy and we wouldn't need all 
these algorithms!).

%%%%%%%%%%%%%%%%%%%%%%%%%%%%%%%%%%%%%%%%%%%%%%%%%%%%%
\subsection{Linearity}\label{S:linearity}
\citet{Geyer:gdor} defines the \emph{linearity} of a set of points $V$ to be
\begin{align*}
	L = \{ w \in V: -w \in \con( \pos V) \}.
\end{align*}
It is the subspace with bases determined by $V$.

We use this to find the face of the polyhedral convex support on which the observed 
statistic lies in the interior of.  The function \texttt{linearity} in the \texttt
{rcdd} package can take as input the rays directed from observed statistic to sample 
points generated via MCMC from the distribution with parameter value of interest.  
From these, it determines the rays that form the basis of a vector subspace.  Of 
course we are not interested in the vector subspace; we need only take the points 
corresponding to these bases as our empirical face $F$.  By construction, the observed 
statistic lies in the interior of $F$.

%Let $\widetilde{T}_C(x)$ be the empirical tangent cone at the point $x$ constructed 
%from a set of points $W$ all in $C$.  That is,
%\begin{align*}
%	\widetilde{T}_C(x) = \{ w - x: w \in W\}.
%\end{align*}
%Then for $V = \widetilde{T}_C(g(\yobs))$, the linearity corresponds to the set of 
%points from $W$ (shifted by $g(\yobs)$) that form what we refer to as the empirical 
%face $F$.  
%In our algorithm, we use the MCMC sample points for $W$.  \citet{Geyer:gdor} provides 
%the function \texttt{linearity} in the R package \texttt{rcdd} to do this computation.

%Consider the case when $g(\yobs) \in \rintr C$.  Then the MLE exists, and the 
%linearity will correspond to all the points $W \in C$.


%%%%%%%%%%%%%%%%%%%%%%%%%%%%%%%%%%%%%%%%%%%%%%%%%%%%%
\subsection{One-sided confidence interval}

Consider the probability distributions defined with log likelihood $\ell( \eta + s 
\delta)$ as defined 
by \eqref{E:loglike}, where $s$ is a real scalar and $\delta$ is a direction of 
recession.  Then 
as $s$ goes from $-\infty$ to $+\infty$, the probability of observing $g(Y) \in H$ 
goes from zero to one.  This 
function is strictly increasing by Theorem~\ref{Thm:LCM}.  Thus we can find the unique 
$s$, call it $\hat{s}$, that makes 
this probability 0.05.  Then $[\hat{s}, +\infty)$ is a 95\% confidence interval for 
the scalar parameter 
$s$, and, in turn, $[\hat{\eta} + \hat{s}\delta, +\infty)$ gives a 95\% confidence 
interval for $\etaMLE
$.

%%%%%%%%%%%%%%%%%%%%%%%%%%%%%%%%%%%%%%%%%%%%%%%%%%%%%
\subsection{Subsampling}
%Once our algorithm has determined that the observed data is on the boundary and thus 
%the MLE does not exist in the conven
Our algorithm depends on being able to to subsample from the MC sample points when 
optimizing in the LCM, restricting to the sample points that comprise the support of 
this model.  The subsampling operation itself is straightforward since we determined 
empirically the face that forms this support.  

However, it is not obvious that when we draw $g(Y_1), \ldots, g(Y_m)$ from the 
original model with parameter value $\eta$ that the subsample restricted to the 
support is in fact a sample from the LCM with parameter value $\eta$.  This is a 
convenient consequence of Theorem~\ref{Thm:LCM}: by this stage in the algorithm, the 
majority of sample points (greater than 60\%) are on the face.  Then $s$ in the 
expression $P_{\eta + s \delta}(g(Y) \in H)$ is a large value.  But since the LCM 
density is the limit of the density of the original model as $s$ goes to $+\infty$, 
the distribution of the LCM is accordingly well approximated.  (\textbf{BUT WELL 
ENOUGH?})
