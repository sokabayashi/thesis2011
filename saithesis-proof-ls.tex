Our algorithm minimizes the objective function $f$ by performing repeated one-
dimensional updates.  We need the 
following lemma to transfer global properties of the objective function to the 
objective function restricted to a 
search direction.

%%%%%%%%%%% BEGIN LEMMA %%%%%%%%%%%%%%
\begin{lemma} \label{Lemma:f min} 
Suppose a function $f:\RR^n \to \RR$ is proper, lower semicontinuous, and strictly 
convex.  Then the minimum for $f$ 
exists and is unique if and only if every nonempty level set $\lev_{\leq \alpha} f$ is 
bounded.
\end{lemma}

%%%%%%%%%%% PROOF - LEMMA %%%%%%%%%%%%%%

\begin{proof}[Proof of Lemma~\ref{Lemma:f min}]
Assume every non-empty level set is bounded.  Then by Theorem~1.9 in \citet
{Rockafellar}, the minimum of $f$ is finite 
and thus exists.  By strict convexity, this minimum is also unique.

Now assume the minimum for $f$ exists, denoted by $\min f$.  By assumption, the level 
set $\lev_{\leq \min f} f$ 
contains exactly one point.  By Corollary~8.7.1 in \citet{Rockafellar:1970}, the level 
sets $\lev_{\leq \alpha} f$ are 
bounded for every $\alpha$.  
\end{proof}

%%%%%%%%%%% PROOF - LINE SEARCH %%%%%%%%%%%%%%
\begin{proof}[Proof of Theorem~\ref{Thm:Line Search}]
The objective function $f$ is bounded below, strictly convex, and lower semicontinuous 
by assumption, and so by Lemma~
\ref{Lemma:f min}, the global minimum exists.  Then by Lemma~\ref{Lemma:f min}, all 
level sets of type $\lev_{\leq a} f$, $a \in \RR$ are bounded in $\RR^n$.  Restricting the set to be along a search 
direction $p_k$ maintains the 
boundedness of these sets.  By Lemma~\ref{Lemma:f min} again, the minimum in this 
restriction exists and is unique.     

Then, unless $\nabla f( x_k ) = 0$ in which case $x_k$ is already the solution, for 
each $k$, we can uniquely define $\alpha_{c_k}$ and $\alpha_{min_k}$ as follows: 
\begin{align}
%	\theta_k &= \cos^{-1} \left( \frac{ -\nabla f_k^T p_k }{ ||\nabla f_k|| \, ||
%p_k||} \right) \label{E:cosine} \\
	\nabla f( x_k + \alpha_{c_k} p_k)^T p_k &= c \nabla f(x_k)^T p_k \label{E:alphac} \\
	\nabla f( x_k + \alpha_{min_k} p_k)^T p_k &= 0. \label{E:alphamin} 
\end{align}
The point $\alpha_{c_k}$ is uniquely defined because it is the minimizer of 
$\alpha \mapsto f( x_k + \alpha p_k) - \alpha c \nabla f( x_k )^T p_k$.
These values appear on the $\alpha$-axis in Figure \ref{F:Wolfe-mod}.
%%%%%%%%%%% FIGURE %%%%%%%%%%%%%%
\begin{figure}
\centering
\scalebox{.4}{\input{Figures/Wolfe-mod.pdf_t}}
\caption{The acceptable region for $\alpha$ according to the curvature condition 
\eqref{E:Wolfe-mod} when restricting 
$f$ to direction $p_k$.}
\label{F:Wolfe-mod}
\end{figure}

%These values are illustrated on the $\alpha$-axis in Figure \ref{F:Wolfe-mod}.  
%Equation \eqref{E:alphamin} defines $\alpha_{min_k}$ to be the step size that would make the 
%gradient at $x_{k+1}$ equal to zero and hence minimizes $f(x_{k+1})$, equation \eqref{E:alphac} 
%defines \alpha_{c_k} to be the step size that would make the gradient at $x_{k+1}$ equal to 

By the strict convexity of $f$ and Theorem~2.14 in \citet{Rockafellar}, 
\begin{align}
	f( x_k + \alpha_{c_k} p_k ) &< f(x_k) +  \bigl[ \nabla f(x_k + \alpha_{c_k} p_k) 
\bigr]^T \alpha_{c_k} p_k. \notag 
\\
	\intertext{Applying \eqref{E:alphac} to the right hand side of the above gives}
	f( x_k + \alpha_{c_k} p_k ) &< f(x_k) + \alpha_{c_k} c \nabla f(x_k)^T p_k. \label
{E:b-less-a}
	\end{align}	
(See points $a$ and $b$ in Figure \ref{F:Wolfe-mod}).

The subproblem $\alpha \mapsto f(x_k + \alpha p_k)$ is strictly convex and hence 
monotonically decreasing at $\alpha_k$ 
such that $\alpha_{c_k} \leq \alpha_k \leq \alpha_{min_k}$ (in Figure \ref{F:Wolfe-
mod}, see points $b$ and $c$).  That 
is,
\begin{align}
	f( x_k + \alpha_{min}p_k) &\leq f( x_k + \alpha_k p_k) \leq f( x_k + \alpha_{c_k} 
p_k). \label{E:f-sandwich}
\end{align}
	
Combining the second inequality of \eqref{E:f-sandwich} with \eqref{E:b-less-a}, we 
have	
\begin{align}
	f( x_k + \alpha_k p_k ) &< f(x_k) + \alpha_{c_k} c \nabla f(x_k)^T p_k,  \label
{E:decrease}
	\intertext{which can be rearranged as}
	f(x_k)-f( x_k + \alpha_k p_k ) &>  -\alpha_{c_k} c \nabla f(x_k)^T p_k. \label
{E:f-lb}
\end{align}
This last inequality \eqref{E:f-lb} expresses a lower bound for the amount of decrease 
in our objective function at 
each step (the right-hand side is positive since $\nabla f(x_k)^T p_k < 0$ by 
assumption that $p_k$ is a descent 
direction).  It is this lower bound that we will use to cover the distance to the 
minimum of the objective function.  

We now turn our attention to \eqref{E:alphac}.  Define $x_{c_k} = x_k + \alpha_{c_k} 
p_k$.  Then
\begin{align}
	\nabla f( x_{c_k} )^T p_k &= c \nabla f(x_k)^T p_k. \notag
\end{align}
Subtracting $\nabla f(x_k)^T p_k$ from both sides gives
\begin{align}
	\left( \nabla f( {x_{c_k}} ) - \nabla f(x_k) \right )^T p_k &= ( c - 1 ) \nabla f
(x_k)^T p_k.  \label{E:c-1}
\end{align}

%\textbf{NEW: SAI 12/05/10}\\
%Since we are considering a finite state space for $| \nabla^2 \ell(\eta) | \leq K$ for some constant K for all $\eta$ 
%\textbf{By Theorems 9.2 and 9.7 in \citet{Rockafellar}, since $\nabla^2 f(x)$ is assumed to be finite, 
%$\nabla f(x)$ is Lipschitz continuous relative to the convex set $\RR^n$.}

By Corollary~25.5.1 in \citet{Rockafellar:1970}, since $f$ is convex and 
differentiable on the open convex set $\NN$, 
it is actually continuously differentiable on $\NN$.  It is then Lipschitz 
continuously differentiable relative to any 
compact subset of $\NN$. 

Applying this to the compact level set $\lev_{\leq f(x_k)} f$, which is contained in $
\NN$ by assumption, there exists 
a constant $L < \infty$ such that
	\begin{align}
		|| \nabla f(x) - \nabla f(\tilde{x}) || \leq L || x - \tilde{x} || \quad \text
{for all $x, \tilde{x} \in \lev_
{\leq f(x_0)} f$}. \label{E:Lipschitz}
	\end{align} 

Applying \eqref{E:Lipschitz} to $x_{c_k}$ and $x_k$, we have
\begin{align}
|| \nabla f(x_{c_k}) - \nabla f(x_k) || &\leq L || x_{c_k} - x_k || \notag
\intertext{or}
|| \nabla f(x_{c_k}) - \nabla f(x_k) || &\leq L || \alpha_{c_k} p_k ||. \notag	
\end{align}

Multiplying both sides by $\lVert p_k \rVert$ gives
\begin{align}
|| \nabla f(x_{c_k}) - \nabla f(x_k) || \cdot ||p_k || &\leq \alpha_{c_k} L || p_k ||
^2 \notag \\
\intertext{and by Cauchy-Schwarz this implies}
( \nabla f(x_{c_k}) - \nabla f(x_k) )^T p_k & \leq || \nabla f(x_{c_k}) - \nabla f
(x_k) || \cdot ||p_k || \label
{E:from-Lipschitz} \\ 
	&\leq \alpha_{c_k} L || p_k ||^2. \notag
\end{align}
Substituting \eqref{E:c-1} into the left-hand side of this last inequality \eqref
{E:from-Lipschitz} gives
\begin{align}
( c - 1 ) \nabla f(x_k)^T p_k &\leq \alpha_{c_k} L || p_k ||^2 \notag\\
\intertext{or}
-\alpha_{c_k} &\leq \frac{( 1 - c )}{L} \frac{ \nabla f(x_k)^T p_k}{ || p_k ||^2}. 
\label{E:-alpha}
\end{align}

%%%%%%
%%%%%%
Write out the first $k+1$ inequalities of \eqref{E:f-lb}:
\begin{align}
\begin{split}
	f( x_1 ) &< f(x_0) + \alpha_{c_0} c \nabla f(x_0)^T p_0 \\
	f( x_2 ) &< f(x_1) + \alpha_{c_1} c \nabla f(x_1)^T p_1  \\
	\ldots  \\
	f( x_{k} ) &< f(x_{k-1}) + \alpha_{c_{k-1}} c \nabla f(x_{k-1})^T p_{k-1}  \\
	f( x_{k+1} ) &< f(x_k) + \alpha_{c_k} c \nabla f(x_k)^T p_k 
\end{split}
	\label{E:to telescope}
\end{align}
Telescoping the right-hand side of \eqref{E:to telescope},
\begin{align*}
	f( x_{k+1} ) &< f(x_0) + c \sum_{j=0}^{k} \alpha_{c_j} \nabla f(x_j)^T p_j. \notag
\end{align*}
Noting that $\nabla f(x_j)^T p_j < 0$, we can substitute our upper bound \eqref{E:-
alpha} for $-\alpha_{c_j}$ in the 
right-hand side above,
\begin{align}
	f( x_{k+1} ) &< f(x_0) - c \sum_{j=0}^{k} \frac{( 1 - c )}{L} 
	\frac{ \nabla f(x_j)^T p_j}{ || p_j ||^2 } \nabla f(x_j)^T p_j \notag \\
	\intertext{which simplies to}
	f( x_{k+1} ) &< f(x_0) - c \sum_{j=0}^{k} \frac{( 1 - c )}{L} 
	\frac{ (\nabla f(x_j)^T p_j)^2}{ || p_j ||^2 }. 
\notag
\end{align}

Because $f(x)$ is bounded below by assumption, there exists some $M < \infty$ such 
that $f(x_0) - f(x_{k+1}) < M$ for 
all $k$. Then rearranging the above yields,

\begin{align*}
	\frac{c( 1 - c )}{L} \sum_{j=0}^{k}   \frac{ ( \nabla f(x_j)^T p_j )^2}{ || p_j ||^2 } &< M < \infty.
 \end{align*}
The angle $\theta_j$ between the search direction $p_k$ and steepest descent direction 
$-\nabla f_k$ can be expressed 
by $\cos \theta_j = \frac{ -\nabla f(x_j)^T p_j}{||\nabla f_j|| \cdot || p_j||}$.  
Substituting this into the equation 
above and taking $k \to \infty$,
\begin{align*}
	\frac{c( 1 - c )}{L} \sum_{j=0}^{\infty}  \cos^2 \theta_j ||\nabla f(x_j)||^2 &< \infty.
\end{align*}
Since $0 < c < 1$,
\begin{align}
	\sum_{j=0}^{\infty}  \cos^2 \theta_j ||\nabla f(x_j)||^2 &< \infty. \label{E:Z's}
\end{align}

The convergent series in \eqref{E:Z's} implies that 
\begin{align*}
	\cos^2 \theta_k || \nabla f(x_k) ||^2 &\to 0 \text{ as } k \to \infty.
\end{align*}
With the additional restriction on the search direction $p_k$ such that $\cos \theta_k \geq \delta > 0$ for some choice 
of $\delta$, for all choices of $k$, we get the desired convergence result of
\begin{align*}
	\lim_{k \to \infty} || \nabla f(x_k) || &= 0.
\end{align*}
\end{proof}
%%%%%%%%%%% END PROOF %%%%%%%%%%%%%%
The inequality \eqref{E:Z's} has been referred to as \emph{Zoutendijk's condition} 
\citep{NW}, though we arrive at this 
result via different assumptions.












\newpage
\section{Proof of Exponential family log likelihood maximization}
\begin{proof}[Proof of Theorem~\ref{Thm:log like max}]
Let $f(\cdot)$ represent the negative log likelihood $- \ell(\cdot)$, the objective 
function to be minimized.  We proceed from the perspective of a minimization of a 
function $f(\cdot)$ since this is the convention in the optimization literature \citep
{NW,Rockafellar}.  \citet{Okabayashi:longrange} prove a special case when the minimum 
exists, and this proof mimics that one.

The negative log likelihood function $-\ell(\cdot)$ is strictly convex by 
\eqref{E:nabla2 ell}, and continuous since it is infinitely differentiable by LEHMAN.

It is bounded below by the negative LCM by Corollary CHARLIE.

The objective function $f$ is bounded below, strictly convex, and lower semicontinuous 
by assumption, and so by Lemma~
\ref{Lemma:f min}, the global minimum exists.  
Then by Lemma~\ref{Lemma:f min}, all level sets of type $\lev_{\leq a} f$, $a \in \RR$ are bounded in $\RR^n$.  Restricting the set to be along a search 
direction $p_k$ maintains the 
boundedness of these sets.  By Lemma~\ref{Lemma:f min} again, the minimum in this 
restriction exists and is unique.     

Then, unless $\nabla f( x_k ) = 0$ in which case $x_k$ is already the solution, for 
each $k$, we can uniquely define $
\alpha_{c_k}$ and $\alpha_{min_k}$ as follows: 
\begin{align}
%	\theta_k &= \cos^{-1} \left( \frac{ -\nabla f_k^T p_k }{ ||\nabla f_k|| \, ||
%p_k||} \right) \label{E:cosine} \\
	\nabla f( x_k + \alpha_{c_k} p_k)^T p_k &= c \nabla f(x_k)^T p_k \label{E:alphac} 
\\
	\nabla f( x_k + \alpha_{min_k} p_k)^T p_k &= 0. \label{E:alphamin} 
\end{align}
The point $\alpha_{c_k}$ is uniquely defined because it is the minimizer of $\alpha 
\mapsto f( x_k + \alpha p_k) - 
\alpha c \nabla f( x_k )^T p_k$.
These values appear on the $\alpha$-axis in Figure \ref{F:Wolfe-mod}.
%%%%%%%%%%% FIGURE %%%%%%%%%%%%%%
\begin{figure}
\centering
\scalebox{.4}{\input{Figures/Wolfe-mod.pdf_t}}
\caption{The acceptable region for $\alpha$ according to the curvature condition 
\eqref{E:Wolfe-mod} when restricting 
$f$ to direction $p_k$.}
\label{F:Wolfe-mod}
\end{figure}

%These values are illustrated on the $\alpha$-axis in Figure \ref{F:Wolfe-mod}.  
%Equation \eqref{E:alphamin} defines $\alpha_{min_k}$ to be the step size that would 
%make the 
%gradient at $x_{k+1}$ equal to zero and hence minimizes $f(x_{k+1})$, equation \eqref
%{E:alphac} 
%defines \alpha_{c_k} to be the step size that would make the gradient at $x_{k+1}$ 
%equal to 

By the strict convexity of $f$ and Theorem~2.14 in \citet{Rockafellar}, 
\begin{align}
	f( x_k + \alpha_{c_k} p_k ) &< f(x_k) +  \bigl[ \nabla f(x_k + \alpha_{c_k} p_k) 
\bigr]^T \alpha_{c_k} p_k. \notag 
\\
	\intertext{Applying \eqref{E:alphac} to the right hand side of the above gives}
	f( x_k + \alpha_{c_k} p_k ) &< f(x_k) + \alpha_{c_k} c \nabla f(x_k)^T p_k. 
	\label{E:b-less-a}
	\end{align}	
(See points $a$ and $b$ in Figure \ref{F:Wolfe-mod}).

The subproblem $\alpha \mapsto f(x_k + \alpha p_k)$ is strictly convex and hence 
monotonically decreasing at $\alpha_k$ 
such that $\alpha_{c_k} \leq \alpha_k \leq \alpha_{min_k}$ (in Figure \ref{F:Wolfe-mod}, see points $b$ and $c$).  That is,
\begin{align}
	f( x_k + \alpha_{min}p_k) &\leq f( x_k + \alpha_k p_k) \leq f( x_k + \alpha_{c_k} p_k). \label{E:f-sandwich}
\end{align}
	
Combining the second inequality of \eqref{E:f-sandwich} with \eqref{E:b-less-a}, we 
have	
\begin{align}
	f( x_k + \alpha_k p_k ) &< f(x_k) + \alpha_{c_k} c \nabla f(x_k)^T p_k,  
	\label{E:decrease}
	\intertext{which can be rearranged as}
	f(x_k)-f( x_k + \alpha_k p_k ) &>  -\alpha_{c_k} c \nabla f(x_k)^T p_k. 
	\label{E:f-lb}
\end{align}
This last inequality \eqref{E:f-lb} expresses a lower bound for the amount of decrease 
in our objective function at 
each step (the right-hand side is positive since $\nabla f(x_k)^T p_k < 0$ by 
assumption that $p_k$ is a descent 
direction).  It is this lower bound that we will use to cover the distance to the 
minimum of the objective function.  

We now turn our attention to \eqref{E:alphac}.  Define $x_{c_k} = x_k + \alpha_{c_k} p_k$.  Then
\begin{align}
	\nabla f( x_{c_k} )^T p_k &= c \nabla f(x_k)^T p_k. \notag
\end{align}
Subtracting $\nabla f(x_k)^T p_k$ from both sides gives
\begin{align}
	\left( \nabla f( {x_{c_k}} ) - \nabla f(x_k) \right )^T p_k &= ( c - 1 ) 
	\nabla f(x_k)^T p_k.  \label{E:c-1}
\end{align}

%\textbf{NEW: SAI 1/17/11}\\
\textbf{
By \eqref{E:nabla2 ell}, $\nabla^2 \ell(\eta)$ is bounded for finite state space $g(\YY)$, which is true by assumption.  Thus $| \nabla^2 f(x) | \leq K$ for some 
constant K for all $x$. 
Then by Theorems 9.2 and 9.7 in \citet{Rockafellar}, 
%since $\nabla^2 f(x)$ is assumed to be finite, 
$\nabla f(x)$ is Lipschitz continuous relative to the convex set $\RR^n$.}

%By Corollary~25.5.1 in \citet{Rockafellar:1970}, since $f$ is convex and 
%differentiable on the open convex set $\NN$, 
%it is actually continuously differentiable on $\NN$.  It is then Lipschitz 
%continuously differentiable relative to any 
%compact subset of $\NN$. 

Applying this to the compact level set $\lev_{\leq f(x_k)} f$, 
%which is contained in $\NN$ by assumption, 
there exists a constant $L < \infty$ such that
	\begin{align}
		|| \nabla f(x) - \nabla f(\tilde{x}) || \leq L || x - \tilde{x} || \quad \text
{for all $x, \tilde{x} \in \lev_
{\leq f(x_0)} f$}. \label{E:Lipschitz}
	\end{align} 

Applying \eqref{E:Lipschitz} to $x_{c_k}$ and $x_k$, we have
\begin{align}
|| \nabla f(x_{c_k}) - \nabla f(x_k) || &\leq L || x_{c_k} - x_k || \notag
\intertext{or}
|| \nabla f(x_{c_k}) - \nabla f(x_k) || &\leq L || \alpha_{c_k} p_k ||. \notag	
\end{align}

Multiplying both sides by $\lVert p_k \rVert$ gives
\begin{align}
	|| \nabla f(x_{c_k}) - \nabla f(x_k) || \cdot ||p_k || &\leq \alpha_{c_k} L || p_k ||^2 
	\notag \\
\intertext{and by Cauchy-Schwarz this implies}
	( \nabla f(x_{c_k}) - \nabla f(x_k) )^T p_k & \leq || 
	\nabla f(x_{c_k}) - \nabla f(x_k) || \cdot ||p_k || \label{E:from-Lipschitz} \\ 
	&\leq \alpha_{c_k} L || p_k ||^2. \notag
\end{align}
Substituting \eqref{E:c-1} into the left-hand side of this last inequality 
\eqref{E:from-Lipschitz} gives
\begin{align}
( c - 1 ) \nabla f(x_k)^T p_k &\leq \alpha_{c_k} L || p_k ||^2 \notag\\
\intertext{or}
-\alpha_{c_k} &\leq \frac{( 1 - c )}{L} \frac{ \nabla f(x_k)^T p_k}{ || p_k ||^2}. 
\label{E:-alpha}
\end{align}

%%%%%%
%%%%%%
Write out the first $k+1$ inequalities of \eqref{E:f-lb}:
\begin{align}
\begin{split}
	f( x_1 ) &< f(x_0) + \alpha_{c_0} c \nabla f(x_0)^T p_0 \\
	f( x_2 ) &< f(x_1) + \alpha_{c_1} c \nabla f(x_1)^T p_1  \\
	\ldots  \\
	f( x_{k} ) &< f(x_{k-1}) + \alpha_{c_{k-1}} c \nabla f(x_{k-1})^T p_{k-1}  \\
	f( x_{k+1} ) &< f(x_k) + \alpha_{c_k} c \nabla f(x_k)^T p_k 
\end{split}
	\label{E:to telescope}
\end{align}
Telescoping the right-hand side of \eqref{E:to telescope},
\begin{align*}
	f( x_{k+1} ) &< f(x_0) + c \sum_{j=0}^{k} \alpha_{c_j} \nabla f(x_j)^T p_j. \notag
\end{align*}
Noting that $\nabla f(x_j)^T p_j < 0$, we can substitute our upper bound 
\eqref{E:-alpha} for $-\alpha_{c_j}$ in the right-hand side above,
\begin{align}
	f( x_{k+1} ) &< f(x_0) - c \sum_{j=0}^{k} \frac{( 1 - c )}{L} 
	\frac{\nabla f(x_j)^T p_j}{ || p_j ||^2 } \nabla f(x_j)^T p_j \notag \\
	\intertext{which simplies to}
	f( x_{k+1} ) &< f(x_0) - c \sum_{j=0}^{k} \frac{( 1 - c )}{L} 
	\frac{ (\nabla f(x_j)^T p_j)^2}{ || p_j ||^2 }. 
\notag
\end{align}

Because $f(x)$ is bounded below by assumption, there exists some $M < \infty$ such 
that $f(x_0) - f(x_{k+1}) < M$ for 
all $k$. Then rearranging the above yields,

\begin{align*}
	\frac{c( 1 - c )}{L} \sum_{j=0}^{k}   
	\frac{ ( \nabla f(x_j)^T p_j )^2}{ || p_j ||^2 } &< M < \infty.
 \end{align*}
The angle $\theta_j$ between the search direction $p_k$ and steepest descent direction 
$-\nabla f_k$ can be expressed 
by $\cos \theta_j = \frac{ -\nabla f(x_j)^T p_j}{||\nabla f_j|| \cdot || p_j||}$.  
Substituting this into the equation 
above and taking $k \to \infty$,
\begin{align*}
	\frac{c( 1 - c )}{L} \sum_{j=0}^{\infty}  \cos^2 \theta_j ||\nabla f(x_j)||^2 &< \infty.
\end{align*}
Since $0 < c < 1$,
\begin{align}
	\sum_{j=0}^{\infty}  \cos^2 \theta_j ||\nabla f(x_j)||^2 &< \infty. \label{E:Z's}
\end{align}

The convergent series in \eqref{E:Z's} implies that 
\begin{align*}
	\cos^2 \theta_k || \nabla f(x_k) ||^2 &\to 0 \text{ as } k \to \infty.
\end{align*}
With the additional restriction on the search direction $p_k$ such that $\cos \theta_k 
\geq \delta > 0$ for some choice 
of $\delta$, for all choices of $k$, we get the desired convergence result of
\begin{align*}
	\lim_{k \to \infty} || \nabla f(x_k) || &= 0.
\end{align*}

\end{proof}


%%%%%%%%%%% BEGIN PROOF - Line Search Works  %%%%%%%%%%%%%%
Theorem 3.1 shows that the gradient of the objective function converges to 0.  The 
proof for Theorem 3.2 is concerned 
with the conditions for mapping this convergence to the convergence of the iterated 
parameter estimates $\eta_k$ to the 
unique MLE.  In particular, the mapping from $\eta_k$ to the gradient must be globally 
invertible.

\begin{proof}[Proof of Theorem~\ref{Thm:Line Search works}]

The Fisher information for a regular exponential family is non-singular by \eqref
{E:FI} and thus invertible.  If we 
consider the map defined by
\begin{align*}
	h(\eta) = \nabla c(\eta)
\end{align*}
where $c$ is the cumulant function \eqref{E:cumulant}, its first derivative matrix is
\begin{align}
	\nabla h(\eta) = \nabla^2 c(\eta) = I(\eta) \label{E:nabla h eta}
\end{align}
which is again non-singular.  Since this is true for any $\eta$, by the inverse 
function theorem, $h$ is everywhere 
locally invertible.

In fact, $h$ is globally invertible. For any $\mu$ in the range of $h$, consider the 
function
\begin{align*}
	q(\eta) = \mu^T\eta - c(\eta).
\end{align*}
Since $\nabla^2 q(\eta) = - I(\eta)$ by \eqref{E:nabla h eta}, $q$ is strictly 
concave.  Therefore, a maximizer for $q$, call it $\hat{\eta}$, is unique if it exists and satisfies the first-order condition
\begin{align*}
	\nabla q( \hat{\eta} ) = 0.
\end{align*}
This in turn implies that
\begin{align*}
	\mu - h(\hat{\eta}) = 0
\end{align*} 
or
\begin{align*}
	\mu = h( \hat{\eta} ). \label{E:eta hat}
\end{align*}
Because of the assumption that $\mu$ is in the range of $h$, this means that 
$\hat{\eta}$ in fact exists, and by the 
strict concavity of $q$, is unique.  This implies that $h$ must be one-to-one and 
hence globally invertible.


Since $c$ is infinitely differentiable by Theorem~2.7.1 in \citet{TSH}, so is $h$, and 
by the inverse function theorem, 
so is $h^{-1}$ (even if we do not know the form of $h^{-1}$).  The first derivative of 
$h^{-1}$ can be expressed as
\begin{align*}
	\nabla h^{-1}(\mu) = \left [ \nabla h(\eta) \right ]^{-1} = \left [ I(\eta) 
\right ]^{-1}
\end{align*}
when $\mu = h(\eta)$ and is thus non-singular everywhere, including at the MLE of 
$\eta$, $\etaMLE$.

Thus our algorithm, which concludes that $ || \nabla \ell( \eta_k) ||  = || g(y) - h
(\eta_k) || \to 0$, implies that 
\begin{align*}
	\mu_k = h(\eta_k) \to g(y), 
\end{align*}
or
\begin{align*}
	h^{-1}(\mu_k)  \to h^{-1}\left (g(y) \right),
\end{align*}
or
\begin{align*}
	\eta_k  \to  \etaMLE.
\end{align*}

\end{proof}
%%%%%%%%%%% END PROOF %%%%%%%%%%%%%%

%\chapter{Proofs of Chapter \ref{sec:gibbs} Results}\label{sec:app1}
%\input{GibbsProofs}
%
%\chapter{Proofs of Chapter \ref{sec:examples} Results}\label{sec:app2}
%\input{ExampleProofs}

